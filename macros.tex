% -*- coding: utf-8 -*-

%% standalone subfiles

\let\byebye = \relax
\csname loaded\endcsname
\let\loaded = \endinput
\let\byebye = \bye

%% macros for The TeXbook

\input manmac

%% compatable with manmac.tex

\catcode `\^=7

%% font definitions

\def\song{SimSun}
\def\hei{SimHei}
\def\kai{KaiTi}
\def\fang{FangSong}
\def\yuan{YouYuan:embolden=3;extend=0.8}
%\def\yuan{FZY3JW--GB1-0:embolden=3;extend=0.95}
%
\input xeCJK-base
\xeCJKenablechecksingle
\normalspacedchars{-}
%
\def\letfont{\let\CJKfont}
\font\zhtenrm="\song" at 10pt
\let\CJKfont\zhtenrm
%
\def\setecfont#1#2#3#4{
  \expandafter\font\csname en#1\endcsname = #2 at #4pt
  \expandafter\font\csname zh#1\endcsname = "#3" at #4pt
  \expandafter\def\csname#1\endcsname{\csname en#1\endcsname\expandafter\letfont\csname zh#1\endcsname}
}

\setecfont{titlefont}{cmssdc10}{\yuan}{40}
%
\setecfont{tenrm}{cmr10}{\song}{10}
\setecfont{ninerm}{cmr9}{\song}{9}
\setecfont{eightrm}{cmr8}{\song}{8}
\setecfont{sevenrm}{cmr7}{\song}{7}
\setecfont{sixrm}{cmr6}{\song}{6}
\setecfont{fiverm}{cmr5}{\song}{5}
%
\setecfont{tenbf}{cmbx10}{\hei}{10}
\setecfont{ninebf}{cmbx9}{\hei}{9}
\setecfont{eightbf}{cmbx8}{\hei}{8}
\setecfont{sevenbf}{cmbx7}{\hei}{7}
\setecfont{sixbf}{cmbx6}{\hei}{6}
\setecfont{fivebf}{cmbx5}{\hei}{5}
%
\setecfont{tentt}{cmtt10}{\fang}{10}
\setecfont{ninett}{cmtt9}{\fang}{9}
\setecfont{eighttt}{cmtt8}{\fang}{8}
\setecfont{seventt}{cmtt7}{\fang}{7}
\setecfont{sixtt}{cmtt6}{\fang}{6}
\setecfont{fivett}{cmtt5}{\fang}{5}
%
\setecfont{tensl}{cmsl10}{\kai}{10}
\setecfont{ninesl}{cmsl9}{\kai}{9}
\setecfont{eightsl}{cmsl8}{\kai}{8}
\setecfont{sevensl}{cmsl7}{\kai}{7}
\setecfont{sixsl}{cmsl6}{\kai}{6}
\setecfont{fivesl}{cmsl5}{\kai}{5}
%
\setecfont{tenit}{cmti10}{\kai}{10}
\setecfont{nineit}{cmti9}{\kai}{9}
\setecfont{eightit}{cmti8}{\kai}{8}
\setecfont{sevenit}{cmti7}{\kai}{7}
\setecfont{sixit}{cmti6}{\kai}{6}
\setecfont{fiveit}{cmti5}{\kai}{5}

\def\ST#1{\font\tempfont="\song" at #1pt\letfont\tempfont}
\def\HT#1{\font\tempfont="\hei" at #1pt\letfont\tempfont}
\def\KT#1{\font\tempfont="\kai" at #1pt\letfont\tempfont}
\def\FS#1{\font\tempfont="\fang" at #1pt\letfont\tempfont}
\def\YY#1{\font\tempfont="\yuan" at #1pt\letfont\tempfont}

%% some redefinition of manmac.tex

\font\titlefonta=cmssdc10 at 20pt

\def\tenpoint{\def\rm{\fam0\tenrm}%
  \textfont0=\tenrm \scriptfont0=\sevenrm \scriptscriptfont0=\fiverm
  \textfont1=\teni \scriptfont1=\seveni \scriptscriptfont1=\fivei
  \textfont2=\tensy \scriptfont2=\sevensy \scriptscriptfont2=\fivesy
  \textfont3=\tenex \scriptfont3=\tenex \scriptscriptfont3=\tenex
  \def\it{\fam\itfam\tenit}%
  \textfont\itfam=\tenit
  \def\sl{\fam\slfam\tensl}%
  \textfont\slfam=\tensl
  \def\bf{\fam\bffam\tenbf}%
  \textfont\bffam=\tenbf \scriptfont\bffam=\sevenbf
   \scriptscriptfont\bffam=\fivebf
  \def\tt{\fam\ttfam\tentt}%
  \textfont\ttfam=\tentt
  \tt \ttglue=.5em plus.25em minus.15em
  \normalbaselineskip=15pt
  \def\MF{{\manual META}\-{\manual FONT}}%
  \let\sc=\eightrm
  \let\big=\tenbig
  \setbox\strutbox=\hbox{\vrule height8.5pt depth3.5pt width\z@}%
  \normalbaselines\rm}
%
\def\ninepoint{\def\rm{\fam0\ninerm}%
  \textfont0=\ninerm \scriptfont0=\sixrm \scriptscriptfont0=\fiverm
  \textfont1=\ninei \scriptfont1=\sixi \scriptscriptfont1=\fivei
  \textfont2=\ninesy \scriptfont2=\sixsy \scriptscriptfont2=\fivesy
  \textfont3=\tenex \scriptfont3=\tenex \scriptscriptfont3=\tenex
  \def\it{\fam\itfam\nineit}%
  \textfont\itfam=\nineit
  \def\sl{\fam\slfam\ninesl}%
  \textfont\slfam=\ninesl
  \def\bf{\fam\bffam\ninebf}%
  \textfont\bffam=\ninebf \scriptfont\bffam=\sixbf
   \scriptscriptfont\bffam=\fivebf
  \def\tt{\fam\ttfam\ninett}%
  \textfont\ttfam=\ninett
  \tt \ttglue=.5em plus.25em minus.15em
  \normalbaselineskip=14pt
  \def\MF{{\manual hijk}\-{\manual lmnj}}%
  \let\sc=\sevenrm
  \let\big=\ninebig
  \setbox\strutbox=\hbox{\vrule height8pt depth3pt width\z@}%
  \normalbaselines\rm
  \ST{9}}
%
\def\eightpoint{\def\rm{\fam0\eightrm}%
  \textfont0=\eightrm \scriptfont0=\sixrm \scriptscriptfont0=\fiverm
  \textfont1=\eighti \scriptfont1=\sixi \scriptscriptfont1=\fivei
  \textfont2=\eightsy \scriptfont2=\sixsy \scriptscriptfont2=\fivesy
  \textfont3=\tenex \scriptfont3=\tenex \scriptscriptfont3=\tenex
  \def\it{\fam\itfam\eightit}%
  \textfont\itfam=\eightit
  \def\sl{\fam\slfam\eightsl}%
  \textfont\slfam=\eightsl
  \def\bf{\fam\bffam\eightbf}%
  \textfont\bffam=\eightbf \scriptfont\bffam=\sixbf
   \scriptscriptfont\bffam=\fivebf
  \def\tt{\fam\ttfam\eighttt}%
  \textfont\ttfam=\eighttt
  \tt \ttglue=.5em plus.25em minus.15em
  \normalbaselineskip=12pt
  \def\MF{{\manual opqr}\-{\manual stuq}}%
  \let\sc=\sixrm
  \let\big=\eightbig
  \setbox\strutbox=\hbox{\vrule height7pt depth2pt width\z@}%
  \normalbaselines\rm}

\def\leftheadline{\hbox to \pagewidth{%
    \vbox to 10pt{}% strut to position the baseline
    {\tenbf\folio}\hfil% folio to left of text
    \rhead% running head flush left
    }}
\def\rightheadline{\hbox to \pagewidth{%
    \vbox to 10pt{}% strut to position the baseline
    \rhead% running head flush right
    \hfil{\tenbf\folio}% folio to right of text
    }}

\def\beginchapter#1 #2#3. #4\par{\global\exno=0
  \subsecno=0
  \def\chapno{#2#3}
%  \ifodd\pageno
%    \errmessage{You had too much text on that last page; I'm backing up}
%    \advance\pageno by-1 \fi
  \titlepage
  \def\\{ } % \\'s in the title will be treated as spaces
  \message{#1 #2#3:} % show the chapter title on the terminal
  \bookmark{1}{#2#3. #4}
  \def\MF{{\manual 89:;<=>:}} % slant the logo
  \xdef\rhead{#2#3: #4\unskip}
  {\def\TeX{T\kern-.2em\lower.5ex\hbox{E}\kern-.06em X}
    \def\MF{{\vbox to30pt{}\manual ()*+,-.*}}
    \def\\{#3}
    \ifx\empty\\ \rightline{\inchhigh #2\kern-.04em}
    \else\rightline{\inchhigh #2\kern-.06em#3\kern-.04em}\fi
    \vskip 1.75pc
    \baselineskip 36pt \lineskiplimit \titlelsl \lineskip 12pt
    \let\\=\cr % now the \\'s are line dividers
    \halign{\line{\titlefont\hfil##}\\#4\unskip\\}
    \vskip 4pc
    \special{pdf:image width 348pt height 0in depth 348pt (artwork/#1\chapno.png)}
    \vfill\eject} % output the chapter title page
  \tenpoint}

\def\beginappendix#1 #2#3. #4\par{\global\exno=0
  \subsecno=0
  \def\chapno{#2#3}
%  \ifodd\pageno
%    \errmessage{You had too much text on that last page; I'm backing up}
%    \advance\pageno by-1 \fi
  \titlepage
  \def\\{ } % \\'s in the title will be treated as spaces
  \message{#1 #2#3:} % show the chapter title on the terminal
  \bookmark{1}{#2#3. #4}
  \def\MF{{\manual 89:;<=>:}} % slant the logo
  \xdef\rhead{#2#3: #4\unskip}
  {\def\TeX{T\kern-.2em\lower.5ex\hbox{E}\kern-.06em X}
    \def\MF{{\vbox to30pt{}\manual ()*+,-.*}}
    \def\\{#3}
    \ifx\empty\\ \rightline{\inchhigh #2\kern-.04em}
    \else\rightline{\inchhigh #2\kern-.06em#3\kern-.04em}\fi
    \vskip 1.75pc
    \baselineskip 36pt \lineskiplimit \titlelsl \lineskip 12pt
    \let\\=\cr % now the \\'s are line dividers
    \halign{\line{\titlefont\hfil##}\\#4\unskip\\}
    \vfill\eject} % output the chapter title page
  \tenpoint}

\outer\def\subsection#1. {\medbreak\advance\subsecno by 1
  \bookmark{2}{\the\subsecno. #1}
  \noindent{\bf \the\subsecno.\enspace#1.\enspace}}

\def\beginlines{\par\begingroup\baselineskip=12pt\nobreak\medskip\parindent\z@ \obeylines
  \hrule\kern1pt\nobreak \everypar{\strut}}

\def\pdfdestx#1{%
  \special{pdf:dest (#1) [@thispage /XYZ 0 @ypos null]}%
}
\def\pdflinkx#1#2{%
  \special{pdf:bann << /Type/Annot/Subtype/Link  /Border [0 0 0] /A << /S/GoTo/D (#1) >> >>}%
  \special{pdf:bc [0 0 1]}#2\special{pdf:ec}%
  \special{pdf:eann}%
}
\def\pdfexno#1#2{\pdfdestx{ex-#1-#2}\pdflinkx{ans-#1-#2}{#1.#2}}
\def\pdfansno#1#2{\pdfdestx{ans-#1-#2}\pdflinkx{ex-#1-#2}{#1.#2.}}

\outer\def\exercise{\medbreak
  \global\advance\exno by 1
  \noindent\llap{\manual\char'170\rm\kern.15em}% triangle in margin
  {\ninebf 练习~\bf\pdfexno{\chapno}{\the\exno}}\par\nobreak\noindent}
\def\dexercise{\global\advance\exno by 1
  \llap{\manual\char'170\rm\kern.15em}% triangle in indented space
  {\eightbf 练习~\bf\pdfexno{\chapno}{\the\exno}}\hfil\break}

\def\ansno#1.#2:{\medbreak\noindent
  \hbox to\parindent{\bf\hss\pdfansno{#1}{#2}\enspace}\ignorespaces}

%\let\oldanswer = \answer
%\outer\def\answer{\par\noindent\hbox to\parindent{\ninebf 答案\hfil\quad}\ninepoint}

\proofmodefalse % this should be false when making camera-ready copy

\def\frac#1/#2{\leavevmode\kern.1em
  \raise.5ex\hbox{\the\scriptfont0 #1}\kern-.1em
  /\kern-.15em\lower.25ex\hbox{\the\scriptfont0 #2}}

%% from old bookall.tex
\baselineskip=14pt
\lineskip=2pt
\tenpoint

%% paper dimensions

\pdfpagewidth=7in \pdfpageheight=10in

%% pdf bookmarks

\def\gbmark#1{\special{pdf: out 1 << /Title <FEFF#1> /Dest [ @thispage /XYZ 0 @ypos null ] >>}}

\special{pdf: docview << /PageMode /UseOutlines /OpenAction [ 0 /Fit ] >>}
\def\bookmark#1#2{%
  \begingroup%
  \def\linebreak{}\def\TeX{TeX}\def\noindent{}\def\\{}\def\hfil{}\def\quad{}%
  \special{pdf:out #1 << /Title (#2) /Dest [ @thispage /XYZ 0 @ypos null ] >>}%
  \endgroup%
}

%% add original page numbers

\newcount\origpageno

\def\strutdepth{\dp\strutbox}
\def\1{\strut\vadjust{\kern-\strutdepth\pagenotext}%
  \pdfdestx{page\the\origpageno}\global\advance\origpageno by 1}
\def\pagenotext{\vtop to \strutdepth{
  \baselineskip\strutdepth\vss\llap{\tt[\the\origpageno]\qquad}\null}}

%% compatable with manmac.tex

\catcode `\^=13
