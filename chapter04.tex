% -*- coding: utf-8 -*-

\input macros

%\beginchapter Chapter 4. Fonts\\of Type
\beginchapter Chapter 4. 字体风格

\origpageno=13

%Occasionally you will want to change from one ^{typeface} to another, for
%example if you wish to be {\bf ^{bold}} or to {\sl emphasize\/} something.
%\TeX\ deals with sets of up to 256 characters called ``^{fonts}'' of type,
%and control sequences are used to select a particular font. For example,
%you could specify the last few words of the first sentence above
%in the following way, using the plain \TeX\ format of Appendix~B:
%\begintt
%to be \bf bold \rm or to \sl emphasize \rm something.
%\endtt
%Plain \TeX\ provides the following control sequences for changing fonts:
%\begindisplay
%^|\rm| switches to the normal ``roman'' typeface:&Roman\cr
%^|\sl| switches to a slanted roman typeface:&\sl Slanted\cr
%^|\it| switches to italic style:&\it Italic\cr
%^|\tt| switches to a typewriter-like face:&\tt Typewriter\cr
%^|\bf| switches to an extended boldface style:&\bf Bold\cr
%\enddisplay
%^^{typewriter type}^^{face}
%At the beginning of a run you get ^{roman type} (|\rm|) unless you specify
%otherwise.
\1有时候,你想把字体从一种变成另一种,例如当你要 {\bf {bold}}~%
或 {\sl emphasize\/} 某些内容时。%
\TeX\ 能处理 256 个字符的集合——所谓的字体,
并且用控制系列来选择特殊的字体。%
例如,你可以用下列方法来指定几个词的字体(采用附录 B 的 plain \TeX\ 格式):
\begintt
to be \bf bold \rm or to \sl emphasize \rm something.
\endtt
\begindisplay
to be \bf bold \rm or to \sl emphasize \rm something.
\enddisplay
Plain \TeX\ 提供了下列控制系列来改变字体:
\begindisplay
|\rm| 转变为标准``roman''字体:&Roman\cr
|\sl| 转变为斜 roman 字体:&\sl Slanted\cr
|\it| 转变为 italic 字体:&\it Italic\cr
|\tt| 转变为类似 typewriter 的字体:&\tt Typewriter\cr
|\bf| 转变为 bold 的字体:&\bf Bold\cr
\enddisplay
在开始排版前,如果你不定义为其它字体,将使用 roman 字体 (|\rm|)。

%Notice that two of these faces have an ``^{oblique}'' slope for emphasis:
%{\sl ^{Slanted type} is essentially the same as roman, but the letters are
%slightly skewed, \it while the letters in ^{italic type} are drawn in a
%different style.} \ (You can perhaps best appreciate the difference between
%the roman and italic styles by contemplating {\tenu letters that are
%in an unslanted italic face.}) \ Typographic conventions are presently
%in a state of transition, because new technology has made it possible to
%do things that used to be prohibitively expensive; people are wrestling
%with the question of how much to use their new-found typographic freedom.
%Slanted roman type was introduced in the 1930s, but it first became widely
%used as an alternative to the conventional italic during the late 1970s.
%It can be bene\-ficial in mathematical texts, since slanted
%letters are distinguishable from the italic letters in math formulas.
%The double use of italic type for two different purposes---for example,
%when statements of theorems are italicized as well as the names of variables in
%those theorems---has led to some confusion, which can now be
%avoided with slanted type. People are not generally agreed about the relative
%merits of slanted versus italic, but slanted type is rapidly becoming a
%favorite for the titles of books and journals in bibliographies.
注意,为了起强调作用,这些类型中的 roman 和 {\sl Slanted} 字体之间有一个倾斜,
而它们本质上是同一个字体,仅仅字母有一些歪斜;而 {\it italic} 字体%
与它们在本质上不是一个字体。%
(通过观察 {\tenu unslanted italic} 字体的字母,你可能会更好地体会到 roman 和%
~italic 字体之间的差别。)
目前排版业行规处于变革时代,因为新技术把过去不可能的工作变得可能;
人们正致力于解决使用新排版技术的成本问题。%
Slanted roman 字体出现在 1930 年代,但是直到 1970 年代后期,它才作为传统 italic~%
字体的替代品而被广泛使用。%
在数学文稿中使用它是有益的,因为 slanted 字母可以与数学公式中的 italic 字母%
区别开。%
在两个不同目的下使用同一 italic 字体——例如,定理的陈述和定理中%
变量的名称都使用 italic——将产生冲突,现在可以用 slanted 类型字体来避免了。%
人们一般对 slanted 相对于 italic 字体的用处不怎么认可,但是 slanted 类型%
正成为参考文献中的书名和杂志名的受欢迎的字体。

%Special fonts are effective for emphasis, but not for sustained reading;
%your eyes would tire if long portions of this manual were entirely set in
%a bold or slanted or italic face. Therefore roman type accounts for the
%bulk of most typeset material. But it's a nuisance to say `|\rm|' every
%time you want to go back to the roman style, so \TeX\ provides an easier
%way to do it, using ``^{curly brace}^^{brace}'' symbols: You can switch
%fonts inside the special symbols |{| and |}|, without affecting the fonts
%outside. For example, the displayed phrase at the beginning of this
%chapter is usually rendered
%\begintt
%to be {\bf bold} or to {\sl emphasize} something.
%\endtt
%This is a special case of the general idea of ``^{grouping}'' that we shall
%discuss in the next chapter. It's best to forget about the first way of
%changing fonts, and to use grouping instead; then your \TeX\ manuscripts
%will look more natural, and you'll probably
%never\footnote*{Well \dots, hardly ever.} have to type `|\rm|'.
特殊字体有利于强调,但是不利于阅读;
如果本手册的很长一部分完全设置为 bold 或 slanted 或 italic 字体,
你的眼睛会很累。%
因此,roman 字体占据了大部分排版内容。%
但是,当要回到 roman 字体时每次都键入`|\rm|'太费事,
于是 \TeX\ 给出了一个简单的办法,利用``大括号''符号:
你可以在特殊符号 |{| 和 |}| 中转换字体而不影响括号外的字体。%
例如,表达本章上面的语句一般用
\begintt
to be {\bf bold} or to {\sl emphasize} something.
\endtt
这是一般编组思想的特殊例子,我们将在下一章讨论它。%
最好把改变字体的前一种方法忘掉,而使用编组;
\1这样,你的 \TeX\ 文稿看起来更舒服,并且可能从不(或几乎从不)要键入`|\rm|'。

%\exercise Explain how to type the bibliographic reference `Ulrich ^{Dieter},
%{\sl Journal f\"ur die reine und angewandte Mathematik\/ \bf201} (1959),
%37--70.' [Use grouping.]
%\answer |Ulrich Dieter, {\sl Journal f\"ur die reine und angewandte|\parbreak
%        |Mathematik\/ \bf201} (1959), 37--70.|\par\nobreak\smallskip\noindent
%It's convenient to use a single group for both |\sl| and |\bf| here. The
%`|\/|' is a refinement that you might not understand until you read the
%rest of Chapter~4.
\exercise 看看怎样键入参考文献%
`Ulrich {Dieter},
{\sl Journal f\"ur die reine und angewandte Mathematik\/ \bf201} (1959),
37--70.' [\ 利用编组。]
\answer |Ulrich Dieter, {\sl Journal f\"ur die reine und angewandte|\parbreak
        |Mathematik\/ \bf201} (1959), 37--70.|\par\nobreak\smallskip\noindent
这里将 |\sl| 和 |\bf| 放在同一个编组中更方便。
`|\/|' 是个细微改良,你读完本章其他内容就会明白它。

%We have glossed over an important aspect of quality in the preceding
%discussion. Look, for example, at the {\it italicized} and {\sl slanted}
%words in this sentence. Since italic and slanted styles slope to the right,
%the d's stick into the spaces that separate these words from the roman
%type that follows; as a result, the spaces appear to be too skimpy,
%although they are correct at the base of the letters. To equalize the
%effective white space, \TeX\ allows you to put the special control sequence
%`^|\/|' just before switching back to unslanted letters. When you type
%\begintt
%{\it italicized\/} and {\sl slanted\/} words
%\endtt
%you get {\it italicized\/} and {\sl slanted\/} words that look better.
%The `|\/|' tells \TeX\ to add an\break % makes the line tighter, to be fair
%``{\sl^{italic correction}\/}'' to the
%previous letter, depending on that letter; this correction is about four
%times as much for an `$f$' as for a `$c$', in a typical italic font.
在上面的讨论中,我们已经阐述了排版质量的一个重要方面。%
例如,观察本句中的 {\it italicized} and 和 {\sl slanted} words 两组词。%
因为 italic 和 slanted 字体向右倾斜,字母 d 刺进了它与随后 roman 字体的%
单词之间的空格;
因此,间距看起来就不够了,虽然字母都是正确的。
为了补偿损失的间距,\TeX\ 允许就在转换回 unslanted 字母前放一个控制系列`|\/|'。
当你键入
\begintt
{\it italicized\/} and {\sl slanted\/} words
\endtt
时,就得到 {\it italicized\/} and 和 {\sl slanted\/} words, 更好看了。%
`|\/|'告诉 \TeX\ 要按不同字母对前一字母作一个``{\sl {italic correction}\/}'';
对标准的 italic 字体,这个校正对 $f$ 是对 $c$ 的 4 倍。

%Sometimes the italic correction is not desirable, because other factors take
%up the visual slack. The standard rule of thumb is to use |\/| just before
%switching from slanted or italic to roman or bold, unless the next
%character is a period or comma. For example, type
%\begintt
%{\it italics\/} for {\it emphasis}.
%\endtt
%Old manuals of style say that the ^{punctuation} after a word should be in the
%{\it same\/} font as that {\it word;\/} but an italic semicolon often looks
%wrong, so this convention is changing. When an italicized word occurs
%just before a semicolon, the author recommends typing `|{\it word\/};|'.
有时候,不想要 italic 校正,因为其它的因素使得在视觉上不感松散。%
标准的上策是只在从 slanted 或 italic 转换到 roman 或 bold 前使用 |\/|,
除非下一个字符是句点或逗号。%
例如,键入
\begintt
{\it italics\/} for {\it emphasis}.
\endtt
在以前的手册格式中,单词后的标点与单词的字体相同;
但是 italic 的分号一般很难看,所以这个约定被改过来了。%
当一个 italicized 单词正好出现在分号前时,作者建议使用 `|{\it word\/};|'。

%\exercise {\it Explain how to typeset a\/ {\rm roman} word in the midst
%of an italicized sentence.}
%\answer |{\it Explain ... typeset a\/ {\rm roman} word ... sentence.}|
%Note the position of the italic correction in this case.
\exercise {\it Explain how to typeset a\/ {\rm roman} word in the midst
of an italicized sentence.}(说明怎样在 italicized 句子中间键入一个 roman 单词)。
\answer |{\it Explain ... typeset a\/ {\rm roman} word ... sentence.}|
注意在此例子中倾斜校正的位置。

%\danger Every letter of every font has an italic correction, which you can
%bring to life by typing |\/|. The correction is usually zero in unslanted
%styles, but there are exceptions: To typeset a bold `{\bf f\/}' in quotes,
%you should say |a| |bold| \hbox{|`{\bf f\/}'|}, lest you get a bold `{\bf f}'.
\danger 每种字体的每个字母都有 italic 校正,你可通过键入 |\/| 来激活它。%
在 unslanted 字体中校正一般是\hbox{零,} 但是有一些例外:
为了排版引号中的 bold `{\bf f\/}', 你应该键入 |bold| \hbox{|`{\bf f\/}'|},
以免得到的是\hbox{bold `{\bf f}'。}

%\ddangerexercise Define a control sequence |\ic| such that `|\ic c|' puts the
%italic correction of character $c$ into \TeX's register |\dimen0|.
%\answer |\def\ic#1{\setbox0=\hbox{#1\/}\dimen0=\wd0|\parbreak
%|\setbox0=\hbox{#1}\advance\dimen0 by -\wd0}|.
\ddangerexercise 定义一个控制系列 |\ic| 使得`|\ic c|'把字符 $c$ 的 italic 校正%
赋值给 \TeX\ 的记数器 |\dimen0|。
\answer |\def\ic#1{\setbox0=\hbox{#1\/}\dimen0=\wd0|\parbreak
|\setbox0=\hbox{#1}\advance\dimen0 by -\wd0}|。

%\ddanger The primitive control sequence ^|\nullfont| stands for a font that
%has no characters. This font is always present, in case you haven't
%specified any others.
\ddanger 原始控制系列 |\nullfont| 代表无字符的一种字体。%
如果你没有给定其它字体,这种字体总会\hbox{出现。}

%Fonts vary in size as well as in shape. For example, the font you are now
%reading is called a ``10-point'' font, because certain features of its
%design are 10 ^{points} apart, when measured in printers' units. \ (We
%will study the point system later; for now, it should suffice to point out
%that the parentheses around this sentence are exactly 10 points tall---and
%the em-dash is just 10 points wide.) \ The ``^{dangerous bend}''
%sections of this manual are set in 9-point type, the foot\-notes in 8-point,
%^{subscripts} in 7-point or 6-point, sub-subscripts in 5-point.
\1字体的外形随字体大小而变。%
例如,你现在所看到的是所谓``10-point'字体,因为在用打印机的单位测量时,
它的设计上的某些特征分别是 10 points。%
(后面我们将阐述 point system; 现在,只需指出,本句的括号的高度正好是 10 points%
---且单破折号的宽度正好是 10 points。)
本手册的``危险''标识的段落设置为 9-point 字体,脚注为 8-point,
上标为 7-point 或 6-point, 上上标为 5-point。

%Each font used in a \TeX\ manuscript is associated with a control sequence;
%for example, the 10-point font in this paragraph is called ^|\tenrm|, and
%the corresponding 9-point font is called ^|\ninerm|. The slanted fonts that
%match |\tenrm| and |\ninerm| are called ^|\tensl| and ^|\ninesl|. These
%control sequences are not built into \TeX, nor are they the actual names
%of the fonts; \TeX\ users are just supposed to make up convenient names,
%whenever new fonts are introduced into a manuscript. Such control
%sequences are used to change typefaces.
在 \TeX\ 文稿中使用的每个字体都伴随一个控制系列;
例如,本段的 10-point 字体叫做 |\tenrm|, 而且相应的 9-point 字体叫做 |\ninerm|。%
与 |\tenrm| 和 |\ninerm| 对应的 slanted 字体叫做 |tensl| 和 |ninesl|。%
这些控制系列不是 \TeX\ 内建的,也不是它们的真实字体名;
当需要在文稿中引入新字体时,只建议 \TeX\ 用户使用便利的名字即可。%
这样的控制系列的作用是改变字体。

%When fonts of different sizes are used simultaneously, \TeX\ will line the
%letters up according to their ``^{baseline}s.'' For example, if you type
%\begintt
%\tenrm smaller \ninerm and smaller
%\eightrm and smaller \sevenrm and smaller
%\sixrm and smaller \fiverm and smaller \tenrm
%\endtt
%the result is {smaller \ninerm and smaller \eightrm and smaller
%\sevenrm and smaller \sixrm and smaller \fiverm and smaller}. Of course
%this is something that authors and readers aren't accustomed to, because
%printers couldn't do such things with traditional lead types. Perhaps
%poets who wish to speak in {\fiverm a still small voice} will cause future
%books to make use of frequent font variations, but nowadays it's only
%an occasional font freak {\fiverm(like the author of this manual)} who
%likes such experiments. One should not get too carried away by the prospect
%of font switching unless there is good reason.
当同时使用不同大小的字体时,\TeX\ 将按照它们的``基线''把字母排成一行。%
例如,如果你键入
\begintt
\tenrm smaller \ninerm and smaller
\eightrm and smaller \sevenrm and smaller
\sixrm and smaller \fiverm and smaller \tenrm
\endtt
得到的是 {smaller \ninerm and smaller \eightrm and smaller
\sevenrm and smaller \sixrm and smaller \fiverm and smaller}。%
当然,这是作者和读者都不习惯的样子,因为打印机不能象传统的铅字排版那样。%
可能希望用 {\fiverm a still small voice} 诵读的诗人会为未来的诗作而采用%
频繁变幻的字体,但是现在仅仅是 {\fiverm(like the author of this manual)} 做实验%
来使用这样古怪的字体。%
要没有更好的理由,你不应该因追求字体转换的效果而忘记主要任务。

%An alert reader might well be confused at this point because we started out
%this chapter by saying that `|\rm|' is the command that switches to roman
%type, but later on we said that `|\tenrm|' is the way to do it. The truth
%is that both ways work. But it has become customary to set things up so that
%|\rm| means ``switch to roman type in the current size'' while |\tenrm| means
%``switch to roman type in the 10-point size.'' In plain \TeX\ format, nothing
%but 10-point fonts are provided, so |\rm| will always get you |\tenrm|; but
%in more complicated formats the meaning of\/ |\rm| will change in different
%parts of the manuscript. For example, in the format used by the author to
%typeset this manual, there's a control sequence `^|\tenpoint|' that causes
%|\rm| to mean |\tenrm|, |\sl| to mean |\tensl|, and so on, while
%`^|\ninepoint|' changes the definitions so that |\rm| means |\ninerm|,
%etc. There's another control sequence used to introduce the quotations at
%the end of each chapter; when the quotations are typed, |\rm| and |\sl|
%temporarily stand for {\eightss 8-point unslanted sans-serif type} and
%{\eightssi 8-point slanted sans-serif type}, respectively. This device of
%constantly redefining the abbreviations |\rm| and |\sl|, behind the
%scenes, frees the typist from the need to remember what size or style of
%type is currently being used.
小心的读者对这点可能会感到困惑,因为在本章开头我们说`|\rm|'是转换到 roman 字体%
的命令,但是后面我们又说`|\tenrm|'也是如此。%
事实上两种方法都可以。%
但在习惯设置上,~|\rm| 的意思是`转换到当前大小的 roman 字体';
而 |\tenrm| 的意思是``转换到 10-point 的 roman 字体''。%
在 plain \TeX\ 格式中,只提供了 10-point 字体,所以 |\rm| 总是等价于 |\tenrm|;
但在更复杂的格式中,|\rm| 的意思在文稿的不同部分是不同的。%
例如,在作者排版本手册所用的格式中,有一个控制系列`|\tenpoint|',
它导致了 |\rm| 等于 |\tenrm|, |\sl| 等于 |\tensl|, 等等;
而`|\ninepoint|'也改变了上述定义使得 |\rm| 等于 |\ninerm| 等等。%
还有另外一个控制系列,它用在每章的结尾的引用语上;
当引用语出现时,|\rm| 和 |\sl| 临时分别变成 {\eightss 8-point
unslanted sans-serif type} 和%
~{\eightssi 8-point slanted sans-serif type}。%
暗中常常重新定义简写 |\rm| 和 |\sl| 的这种方法使得排版者不需要记住%
现在需要使用的字体及其大小。

%\exercise Why do you think the author chose the names `|\tenpoint|' and
%`|\tenrm|', etc., instead of `|\10point|' and `|\10rm|'\thinspace?
%\answer Control word names are made of letters, not digits.
\exercise \1你知道为什么作者选用`|\tenpoint|'和`|\tenrm|'等等这样的名字,
而不使用`|\10point|'和`|\10rm|'\allowbreak\hbox{呢?}
\answer \1控制词的名称只能由字母组成,不能包含数字。

%\dangerexercise Suppose that you have typed a manuscript using slanted type for
%emphasis, but your editor suddenly tells you to change all the slanted to
%italic.  What's an easy way to do this?
%\answer Say |\def\sl{\it}| at the beginning, and delete other definitions
%of\/ |\sl| that might be present in your format file (e.g., there might be
%one inside a |\tenpoint| macro).
\dangerexercise 假设你在一个文稿中用 slanted 字体表示强调,但是编辑却突然%
告诉你把所有的 slanted 改成 italic。怎样用简单的方法来实现它?
\answer 在开头写上 |\def\sl{\it}|,然后删除你的格式文件中可能包含的
对 |\sl| 的其他定义(比如在 |\tenpoint| 宏里面可能有一个)。

%\danger Each font has an external name that identifies it with respect to
%all other fonts in a particular library. For example, the font in this
%sentence is called `|cmr9|', which is an abbreviation for ``^{Computer
%Modern} Roman 9~point.'' ^^{cm fonts} In order to prepare \TeX\ for
%using this font, the command
%\begintt
%\font\ninerm=cmr9
%\endtt
%appears in Appendix E\null. In general you say `^|\font||\cs=|\<external
%font name>' to load the information about a particular font into \TeX's
%memory; afterwards the control sequence |\cs| will select that font for
%typesetting.  Plain \TeX\ makes only sixteen fonts available initially (see
%Appendix~B and Appendix~F\null), but you can use |\font| to access
%anything that exists in your system's font library.
\danger 每个字体都有一个外部名字以区别于这个特殊库中的所有其它字体。%
例如,本句中的字体(英文版)称为`|cmr9|', 它是``{Computer
Modern} Roman 9~point''的缩写。%
为了让 \TeX\ 使用本字体,在附录 E 中出现了下列命令
\begintt
\font\ninerm=cmr9
\endtt
一般地,你用`|\font||\cs=|\<external
font name>'把一个特殊字体的有关信息载入到内存中;
然后控制系列 |\cs| 将在排版时选择那个字体。%
Plain \TeX\ 最初只定义了 16 个字体(见附录 B 和 E),
但是你可以用 |\font| 把你的系统字体库中的任何字体为 \TeX\ 所用。

%\danger It is often possible to use a font at several different sizes, by
%magnifying or shrinking the character images. Each font has a so-called
%^{design size}, which reflects the size it normally has by default; for
%example, the design size of |cmr9| is 9~points. But on many systems there is
%also a range of sizes at which you can use a particular font, by scaling its
%dimensions up or down. To load a scaled font into \TeX's memory, you
%simply say `|\font\cs=|\<external font name> ^|at| \<desired size>'.
%For example, the command
%\begintt
%\font\magnifiedfiverm=cmr5 at 10pt
%\endtt
%brings in 5-point Computer Modern Roman at twice its normal size. \ (Caution:
%Before using this `|at|' feature, you should check to make sure that your
%typesetter supports the font at the size in question; \TeX\ will accept any
%\<desired size> that is positive and less than 2048 points, but the final
%output will not be right unless the scaled font really is available on your
%printing device.)
\danger 通过放大或缩小字符的图像,通常一种字体可以有几种不同的大小。%
每种字体有一个所谓设计尺寸,它表示其正常尺寸的缺省值;
例如,|cmr9| 的设计尺寸是 9 points。%
但是在许多系统中,通过放大和缩小尺寸,在一个大小范围内也可以使用某一特定字体。%
为了把缩放后的字体载入 \TeX\ 的内存,你只需直接使用%
`|\font\cs=|\<external font name> |at| \<desired size>'。%
例如,命令
\begintt
\font\magnifiedfiverm=cmr5 at 10pt
\endtt
把两倍于正常尺寸的 5-point Computer Modern Roman 载入。%
(注意:在使用这个`|at|'属性时,你应该确保你的排字机支持所讨论的那个尺寸的字体;
\TeX\ 容许小于 2048 points 的正值的任意 \<desired size>,
但是如果你的打印设备不支持缩放的字体,输出结果就不会正确。

%\danger What's the difference between |cmr5| |at| |10pt| and the normal
%10-point font, |cmr10|? Plenty; a well-designed font will be drawn
%differently at different point sizes, and the letters will often have
%different relative heights and widths, in order to enhance readability.
%\begindisplay
%\tenrm Ten-point type is different from%
%  \magnifiedfiverm\ magnif{}ied f{}ive-point type.
%\enddisplay
%It is usually best to scale fonts only slightly with respect to
%their design size, unless the final product is going to be photographically
%reduced after \TeX\ has finished with it, or unless you are trying for an
%unusual effect.^^{magnification}^^{reduction}
\danger 在 |cmr5| |at| |10pt| 和正常的 10-point 字体 |cmr10| 之间有什么差别呢?
很多呀;设计精良的字体在不同尺寸的图像是不同的,并且%
为了提高观感,字母通常有不同的相对高度和宽度。
\begindisplay
\tenrm Ten point type is different from%
  \magnifiedfiverm\ magnif{}ied f{}ive-point type.
\enddisplay
通常最好只相对于设计尺寸微微放大字体,除非在 \TeX\ 输出完毕后最后输出图像要%
重新缩减回去,或者你在实验一种特殊效果。

%\danger Another way to magnify a font is to specify a scale factor that is
%relative to the design size. For example, the command
%\begintt
%\font\magnifiedfiverm=cmr5 scaled 2000
%\endtt
%is another way to bring in the font ^|cmr5| at double size. The scale factor
%is specified as an integer that represents a magnification ratio times~1000.
%Thus, a scale factor of 1200 specifies magnification by 1.2, etc.
\danger 放大字体的另一种方法是相对于设计尺寸给出放大比例因子。%
例如,命令
\begintt
\font\magnifiedfiverm=cmr5 scaled 2000
\endtt
\1是得到 |cmr5| 字体两倍尺寸的另一种方法。%
放大比例因子规定是整数,等于放大比例乘以 1000。%
因此,放大因子为 1200 就是放大 1.2 倍,等等。

%\dangerexercise State two ways to load font |cmr10| into \TeX's memory
%at half its normal size.
%\answer |\font\squinttenrm=cmr10 at 5pt|\parbreak
%        |\font\squinttenrm=cmr10 scaled 500|
\dangerexercise 用两种方法把 |cmr10| 的正常尺寸的一半大小的字体载入 \TeX\ 内存中。
\answer |\font\squinttenrm=cmr10 at 5pt|\parbreak
        |\font\squinttenrm=cmr10 scaled 500|

%\font\onerm=cmr10 scaled\magstep1
%\font\onett=cmtt10 scaled\magstep1
%\font\tworm=cmr10 scaled\magstep2
%\font\twott=cmtt10 scaled\magstep2
%%\font\threerm=cmr10 scaled\magstep3 % such large magnifications look ugly
%%\font\threett=cmtt10 scaled\magstep3 % in a book context!
%\danger At many computer centers it has proved convenient to supply fonts
%at magnifications that grow in geometric ratios---something like equal-tempered
%tuning on a ^{piano}. The idea is to have all fonts available at their true
%size as well as at magnifications 1.2 and~1.44 (which is $1.2\times1.2$);
%perhaps also at magnification~1.728 ($=1.2\times1.2\times1.2$) and even
%higher. Then you can magnify an entire document by 1.2 or~1.44 and still
%stay within the set of available fonts. Plain \TeX\ provides the
%abbreviations ^|\magstep||0| for a scale factor of 1000, |\magstep1| for a
%scaled factor of 1200, |\magstep2| for 1440, and so on up to |\magstep5|.
%You say, for example,
%\begintt
%\font\bigtenrm=cmr10 scaled\magstep2
%\endtt
%to load font |cmr10| at $1.2\times1.2$ times its normal size.
%\begindisplay \lineskip5pt
%\tenrm\llap{``}This is {\tentt cmr10} at normal size
%  ({\tentt \char`\\magstep0}).''\cr
%\onerm\llap{``}This is {\onett cmr10} scaled once by 1.2
%  ({\onett \char`\\magstep1}).''\cr
%\tworm\llap{``}This is {\twott cmr10} scaled twice by 1.2
%  ({\twott \char`\\magstep2}).''\cr
%%\threerm\llap{``}This is {\threett cmr10} scaled by
%%  {\threett \char`\\magstep3}.''\cr
%\enddisplay
%(Notice that a little magnification goes a long way.) \
%There's also ^|\magstephalf|, which magnifies by $\sqrt{1.2}$,
%i.e., halfway between steps 0 and~1.
\font\onerm=cmr10 scaled\magstep1
\font\onett=cmtt10 scaled\magstep1
\font\tworm=cmr10 scaled\magstep2
\font\twott=cmtt10 scaled\magstep2
%\font\threerm=cmr10 scaled\magstep3 % such large magnifications look ugly
%\font\threett=cmtt10 scaled\magstep3 % in a book context!
\danger 在许多计算机中心,已表明按照几何比例使用放大的字体是很有益的——%
有点象钢琴上精确的调\hbox{音。}%
这个思想使所有字体在正常尺寸,1.2 倍和 1.44($1.2\times1.2$) 倍尺寸都可以使用;
可能还可以放大到 1.728($=1.2\times1.2\times1.2$) 倍或更大。%
因此,你可以把整个书稿放大 1.2 或 1.44 倍而仍然得到精美的\hbox{字体。}%
Plain \TeX\ 提供了一个缩略命令,|\magstep||0| 是放大 1000 比例因子,
|\magstep1| 是放大 1200 比例因子,|\magstep2| 是 1440, 如此直到 |\magstep5|。%
例如,使用
\begintt
\font\bigtenrm=cmr10 scaled\magstep2
\endtt
就载入了 |cmr10| 字体的正常尺寸的 $1.2\times1.2$ 倍。
\begindisplay \lineskip5pt
\tenrm\llap{``}This is {\tentt cmr10} at normal size
  ({\tentt \char`\\magstep0}).''\cr
\onerm\llap{``}This is {\onett cmr10} scaled once by 1.2
  ({\onett \char`\\magstep1}).''\cr
\tworm\llap{``}This is {\twott cmr10} scaled twice by 1.2
  ({\twott \char`\\magstep2}).''\cr
%\threerm\llap{``}This is {\threett cmr10} scaled by
%  {\threett \char`\\magstep3}.''\cr
\enddisplay
(注意:略微放大这种方法用处很大。)
还有一个命令 |\magstephalf|, 它放大 $\sqrt{1.2}$ 倍,即介于 |\magstep0| 和%
~|\magstep1| 之间。

%\danger Chapter~10 explains how to apply magnification to an entire
%document, over and above any magnification that has been specified when
%fonts are loaded. For example, if you have loaded a font that is scaled
%by |\magstep1| and if you also specify ^|\magnification||=\magstep2|, the
%actual font used for printing will be scaled by |\magstep3|. Similarly, if
%you load a font scaled by |\magstephalf| and if you also say
%|\magnification=\magstephalf|, the printed results will be
%scaled by |\magstep1|.
\danger 除了字体载入时已经被%
给定的放大之外,第10章阐述了怎样在整个书稿上使用放大方法。%
例如,如果你已按 |\magstep1| 载入一个字体,并且你还规定 |\magnification||=\magstep2|,
那么在打印时所用的实际字体就按 |\magstep3| 放大。%
类似地,如果载入字体的放大是 |\magstephalf| 且 |\magnification=||\magstephalf|,
那么打印结果放大 |\magstep1|。

\endchapter

Type faces---like people's faces---have distinctive features
indicating aspects of character. % I don't think he was kidding
\author MARSHALL ^{LEE}, {\sl Bookmaking\/} (1965) % page 83

\bigskip

This was the Noblest Roman of them all.
\author WILLIAM ^{SHAKESPEARE}, {\sl The Tragedie %
  of Julius C\ae sar\/} (1599) % Act V, Scene 5, line 68
  % For Shakespeare I'm using the spelling from First Folio (1623)
  % (titles not from the contents page, but the running heads in the plays)
  % but act/line numbers from The Riverside Shakespeare (throughout)

\vfill\eject\byebye
