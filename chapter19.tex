% -*- coding: utf-8 -*-

\input macros

%\beginchapter Chapter 19. Displayed Equations
\beginchapter Chapter 19. 陈列公式

\origpageno=185

%By now you know how to type mathematical formulas so that \TeX\ will handle
%them with supreme elegance; your knowledge of math typing is nearly complete.
%But there is one more part to the story, and the purpose of this chapter
%is to present the happy ending. We have discussed how to deal with individual
%formulas; but ^{displays} often involve a whole bunch of different formulas,
%or different pieces of a huge formula, and it's a bit of a problem to
%lay them out so that they line up properly with each other. Fortunately,
%large displays generally fall into a few simple patterns.
\1现在你已经知道怎样输入数学公式以得到漂亮的排版了;
数学排版方面的知识基本差不多了。%
但是还有部分内容,本章将要讨论它。%
我们已经讨论过怎样排版单个公式;
但是陈列公式常常包括一整组不同的公式,
或者一个大公式的不同片段,
并且在把它们正确地对齐时会出现一点问题。%
幸运的是,大量陈列公式一般可以分为几个简单的类型。

%\subsection One-line displays. Before plunging into the general question
%of display layout, let's recapitulate what we have already covered.  If
%you type `|$$|\<formula>|$$|', \TeX\ will display the formula
%^^{dollardollar} in flamboyant display style, centering it on a line by
%itself. We have also noted in Chapter~18 that it's possible to display two
%short formulas at once, by typing
%`|$$|\<formula$_1$>^|\qquad|\<formula$_2$>|$$|'; this reduces the
%two-formula problem to a one-formula problem. You get the two formulas
%separated by two quads of space, the whole being centered on a line.
\subsection 单行陈列公式.
在讨论陈列公式排版的一般问题前,我们先简要地复习一下。%
如果输入`|$$|\<formula>|$$|', 那么 \TeX\ 将把公式用精美的陈列样式显示出来,
自己单独占一行,并且居中。%
我们还注意到,在第十八章,可以同时显示两个短公式,只要输入%
`|$$|\<formula$_1$>|\qquad|\<formula$_2$>|$$|'即可;
它把两个公式的问题简化为一个公式的问题。%
你所得到的两个公式用两个 quad 的间距隔开,而它们整体行居中。

%Displayed equations often involve ordinary text. Chapter~18 explains
%how to get roman type into formulas without leaving math mode, but the
%best way to get text into a display is to put it into an ^|\hbox|.
%There needn't even be any math at all; to typeset
%$$\hbox{Displayed Text}$$
%you can simply say `|$$\hbox{Displayed Text}$$|'. But here's a more interesting
%example:
%$$X_n=X_k \qquad\hbox{if and only if}\qquad
%    Y_n=Y_k \quad\hbox{and}\quad Z_n=Z_k.$$
%Formulas and text were combined in this case by typing
%\begintt
%$$X_n=X_k \qquad\hbox{if and only if}\qquad
%    Y_n=Y_k \quad\hbox{and}\quad Z_n=Z_k.$$
%\endtt
%Notice that |\qquad| appears around `if and only if', but a single ^|\quad|
%surrounds `and'; this helps to indicate that the $Y$ and~$Z$ parts of the
%display are related more closely to each other than to the $X$~part.
陈列公式常常包含普通文本。%
第十八章讨论了怎样在不离开数学模式的情况下在公式中输入 roman 字体,
但是在陈列公式中排版文本的最好方法是把它放在 |\hbox| 中。%
这样甚至不需要任何数学内容;
为了得到
$$\hbox{要显示的文本}$$
可以只输入`|$$\hbox{要显示的文本}$$|'。%
但是更好的例子在下面:
$$X_n=X_k \qquad\hbox{if and only if}\qquad
    Y_n=Y_k \quad\hbox{and}\quad Z_n=Z_k.$$
在这种情况下,公式和文本都被组合起来,只要输入
\begintt
$$X_n=X_k \qquad\hbox{if and only if}\qquad
    Y_n=Y_k \quad\hbox{and}\quad Z_n=Z_k.$$
\endtt
注意,|\qquad| 出现在`if and only if'两边,但是在`and'两边只有一个 |\quad|;
这清楚地表明了陈列公式的 $Y$ 和 $Z$ 的关系比它们与 $X$ 的关系更近。

%Consider now the display
%$$Y_n=X_n\bmod p \quad\hbox{and}\quad Z_n=X_n\bmod q
%    \qquad\hbox{for all }n\ge0.$$
%Can you figure out how to type this? One solution is
%\begintt
%$$Y_n=X_n\bmod p \quad\hbox{and}\quad Z_n=X_n\bmod q
%    \qquad\hbox{for all }n\ge0.$$
%\endtt
%Notice that a space has been left after `|all|' in the hbox here, since spaces
%disappear when they are out in formula-land. But there's a simpler and more
%logical way to proceed, once you get used to \TeX's idea of modes: You can type
%\begintt
%... \qquad\hbox{for all $n\ge0$.}$$
%\endtt
%Wow---that's math mode inside of horizontal mode inside of display
%math mode. But in this way your manuscript mirrors what you are trying to
%accomplish, while the previous solution (with the space after `|all|')
%looks somewhat forced.
现在看看陈列公式
$$Y_n=X_n\bmod p \quad\hbox{and}\quad Z_n=X_n\bmod q
    \qquad\hbox{for all }n\ge0.$$
    你知道怎样输入吗?一种方法是
\begintt
$$Y_n=X_n\bmod p \quad\hbox{and}\quad Z_n=X_n\bmod q
    \qquad\hbox{for all }n\ge0.$$
\endtt
注意,这里 hbox 中的`|all|'后面有一个空格,
因为当空格出现在公式区时会去掉。%
但是一旦你理解了 \TeX\ 的模式思想,就有更好和更合理的方法:
可以输入
\begintt
... \qquad\hbox{for all $n\ge0$.}$$
\endtt
\1噢——是陈列数学模式中的水平模式中的数学模式。%
但是用这种方法,文稿正好反映出你要表达的意思,
而前一种方法(`|all|'后面加空格)看起来有点不得已。

%\exercise Typeset the following four displays (one at a time):
%$$\openup1\jot\displaylines{
%\sum_{n=0}^\infty a_nz^n\qquad\hbox{converges if}\qquad
%  \vert z\vert<\Bigl(\limsup_{n\to\infty}
%    \root n\!\of{\vert a_n\vert}\,\Bigr)^{-1}.\cr
%{f(x+\Delta x)-f(x)\over\Delta x}\to f'(x)\qquad\hbox{as $\Delta x\to0$.}\cr
%\noalign{\vskip2pt}
%\Vert u_i\Vert=1,\qquad u_i\cdot u_j=0\quad\hbox{if $i\ne j$.}\cr
%\it\hbox{The confluent image of}\quad\left\{
%  \matrix{\hbox{an arc}\hfill\cr\hbox{a circle}\hfill\cr
%    \hbox{a fan}\hfill\cr}
%  \right\}\quad\hbox{is}\quad\left\{
%  \matrix{\hbox{an arc}\hfill\cr\hbox{an arc or a circle}\hfill\cr
%    \hbox{a fan or an arc}\hfill\cr}\right\}.\cr
%\noalign{\vskip-8pt}}$$
%^^|\Delta| ^^|\Vert|
%% the last example comes from Proc AMS 55 (1976), 410, with typos corrected
%\answer |$$\sum_{n=0}^\infty a_nz^n\qquad\hbox{converges if}\qquad|\parbreak
%|  |\||z|\||<\Bigl(\limsup_{n\to\infty}\root n\!\of{|\||a_n|\|^^|\root|
%  |}\,\Bigr)^{-1}.$$|\kern-.33pt\par
%\smallskip
%|$${f(x+\Delta x)-f(x)\over\Delta x}\to f'(x)|\parbreak
%|    \qquad\hbox{as $\Delta x\to0$.}$$|\par
%\smallskip
%|$$\|\||u_i\|\||=1,\qquad u_i\cdot u_j=0\quad\hbox{if $i\ne j$.}$$|\par
%\smallskip
%|$$\it\hbox{The confluent image of}\quad\left\{|\parbreak
%|    \matrix{\hbox{an arc}\hfill\cr\hbox{a circle}\hfill\cr|\parbreak
%|      \hbox{a fan}\hfill\cr}|\parbreak
%|    \right\}\quad\hbox{is}\quad\left\{|\parbreak
%|    \matrix{\hbox{an arc}\hfill\cr|\parbreak
%|      \hbox{an arc or a circle}\hfill\cr|\parbreak
%|      \hbox{a fan or an arc}\hfill\cr}\right\}.$$|\par
%\smallskip\noindent
%The first example includes |\!| and |\,| to give slightly refined spacing;
%but the point of the problem was to illustrate the hbox, not to fuss over
%such extra details.
%The last example can be done much more simply using the ideas of
%Chapter~22, if you don't mind descending to the level of \TeX\ primitives;
%for example, the first matrix could be replaced by ^^|\halign|
%\begintt
%\,\vcenter{\halign{#\hfil\cr an arc\cr a circle\cr a fan\cr}}\,
%\endtt
%and the second is similar.
\exercise 输入下列四个陈列公式(一次一个):
$$\openup1\jot\displaylines{
\sum_{n=0}^\infty a_nz^n\qquad\hbox{converges if}\qquad
  \vert z\vert<\Bigl(\limsup_{n\to\infty}
    \root n\!\of{\vert a_n\vert}\,\Bigr)^{-1}.\cr
{f(x+\Delta x)-f(x)\over\Delta x}\to f'(x)\qquad\hbox{as $\Delta x\to0$.}\cr
\noalign{\vskip2pt}
\Vert u_i\Vert=1,\qquad u_i\cdot u_j=0\quad\hbox{if $i\ne j$.}\cr
\it\hbox{The confluent image of}\quad\left\{
  \matrix{\hbox{an arc}\hfill\cr\hbox{a circle}\hfill\cr
    \hbox{a fan}\hfill\cr}
  \right\}\quad\hbox{is}\quad\left\{
  \matrix{\hbox{an arc}\hfill\cr\hbox{an arc or a circle}\hfill\cr
    \hbox{a fan or an arc}\hfill\cr}\right\}.\cr
\noalign{\vskip-8pt}}$$
\answer |$$\sum_{n=0}^\infty a_nz^n\qquad\hbox{converges if}\qquad|\parbreak
|  |\||z|\||<\Bigl(\limsup_{n\to\infty}\root n\!\of{|\||a_n|\|^^|\root|
  |}\,\Bigr)^{-1}.$$|\kern-.33pt\par
\smallskip
|$${f(x+\Delta x)-f(x)\over\Delta x}\to f'(x)|\parbreak
|    \qquad\hbox{as $\Delta x\to0$.}$$|\par
\smallskip
|$$\|\||u_i\|\||=1,\qquad u_i\cdot u_j=0\quad\hbox{if $i\ne j$.}$$|\par
\smallskip
|$$\it\hbox{The confluent image of}\quad\left\{|\parbreak
|    \matrix{\hbox{an arc}\hfill\cr\hbox{a circle}\hfill\cr|\parbreak
|      \hbox{a fan}\hfill\cr}|\parbreak
|    \right\}\quad\hbox{is}\quad\left\{|\parbreak
|    \matrix{\hbox{an arc}\hfill\cr|\parbreak
|      \hbox{an arc or a circle}\hfill\cr|\parbreak
|      \hbox{a fan or an arc}\hfill\cr}\right\}.$$|\par
\smallskip\noindent
第一个例子用 |\!| 和 |\,| 给出精细点的间距;
但此问题的要点是要说明 hbox,而非关注太多细节。
如果不\1介意用 \TeX\ 原始命令,
利用第 22 章的想法你可以大大简化最后一个例子;
比如第一个矩阵可以改写为^^|\halign|
\begintt
\,\vcenter{\halign{#\hfil\cr an arc\cr a circle\cr a fan\cr}}\,
\endtt
而第二个矩阵类似。

%\dangerexercise Sometimes display style is too grandiose, when the formula
%being displayed is
%$$y={1\over2}x$$
%or something equally simple. One day B. L. ^{User} tried to remedy this by
%typing it as `|$$y={\scriptstyle1\over\scriptstyle2}x$$|', but the
%resulting formula
%$$y={\scriptstyle1\over\scriptstyle2}x$$
%wasn't at all what he had in mind. What's the right way to get ^^{one half}
%^^{1/2--unslashed form} simply `$y={1\over2}x$' when you don't want big
%^{fractions in displays}?
%\answer |$$\textstyle y={1\over2}x$$|. \ (Switching to text style is
%especially common in multiline formulas. For example, you will probably
%find occasions to use ^|\textstyle| on both sides of the |&|'s within
%an ^|\eqalign|.)
\dangerexercise 当要排版的公式为
$$y={1\over2}x$$
或者类似这样的简单东西时,有时候陈列样式太大了。
有一天^{用户笨笨}希望通过输入
`|$$y={\scriptstyle1|\allowbreak|\over\scriptstyle2}x$$|'
来弥补一下,结果得到的
$$y={\scriptstyle1\over\scriptstyle2}x$$
根本不是他想要的。在不想要陈列公式中的大分式,
而仅仅得到 `$y={1\over2}x$' 时,正确的方法是什么?
\answer |$$\textstyle y={1\over2}x$$|。%
(切换到文本样式是多行公式中相当常见的用法。例如,
你将发现在 ^|\eqalign| 的 |&| 的两边有时会用到 ^|\textstyle|。)

%\dangerexercise What difference, if any, is there between the result of
%typing `|$$|\<formula>|$$|' and the result of typing
%`|$$\hbox{$|\<formula>|$}$$|'\thinspace?
%\answer The latter formula will be in text style, not display style.
%And even if you do type `|$$\hbox{$\displaystyle{|\<formula>|}$}$$|', the
%results are not quite the same, as we will see later: \TeX\ will compress
%the glue in `|$$|\<formula>|$$|' if the formula is too wide to fit on
%a line at its natural width, but the glue inside |\hbox{...}| is frozen
%at its natural width. The |\hbox| version also invokes |\everymath|.
\dangerexercise 输入 `|$$|\<formula>|$$|' 和 `|$$\hbox{$|\<formula>|$}$$|'
所得到的结果之间有什么差别么?
\answer 后一个公式将用文本样式,而不是陈列样式显示。
而且即使你确实输入了 `|$$\hbox{$\displaystyle{|\<formula>|}$}$$|',
结果也不会完全一样,这是我们稍后会看到的:
当公式太宽而无法以自然宽度放在一行时,
\TeX\ 将会压缩 `|$$|\<formula>|$$|' 的粘连,
但|\hbox{...}| 内的粘连总是固定为它的自然宽度。
|\hbox| 版本还调用了 |\everymath|。

%\dangerexercise You may have noticed that most of the displays in this
%manual are not centered; displayed material is usually aligned at the
%left with the paragraph indentation, as part of the book design, because
%this is an unusual book. Explain how you could typeset a formula like
%$$\leftline{\indent$\displaystyle
%    1-{1\over2}+{1\over3}-{1\over4}+\cdots=\ln2$}$$
%^^{displays, non-centered}
%that is off-center in this way.
%\answer One solution is to put the formula in an hbox that occupies a full line:
%\begintt
%$$\leftline{\indent$\displaystyle
%    1-{1\over2}+{1\over3}-{1\over4}+\cdots=\ln2$}$$
%\endtt
%But this takes a bit of typing. If you make the definitions
%\begintt
%\def\leftdisplay#1$${\leftline{\indent$\displaystyle{#1}$}$$}
%\everydisplay{\leftdisplay}
%\endtt
%you can type `|$$|\<formula>|$$|' as usual, and the formatting will be
%inserted automatically. \ (This doesn't work with equation numbers;
%Appendix~D illustrates how to handle them as well.)
\dangerexercise 你可能已经注意到了,本手册的大部分陈列公式都不是居中的;
因为本书不是一般的书,作为书籍设计的一部分,陈列公式一般与段落左边的缩进对齐。
看看怎样排版出下面这样不居中的公式:
$$\leftline{\indent$\displaystyle
    1-{1\over2}+{1\over3}-{1\over4}+\cdots=\ln2$}$$
\answer 其中一种解法是将公式放在一个占用整行的 hbox 中:
\begintt
$$\leftline{\indent$\displaystyle
    1-{1\over2}+{1\over3}-{1\over4}+\cdots=\ln2$}$$
\endtt
但这将需要多键入一些字符。如果你使用下面的定义
\begintt
\def\leftdisplay#1$${\leftline{\indent$\displaystyle{#1}$}$$}
\everydisplay{\leftdisplay}
\endtt
你就可以像往常一样键入 `|$$|\<formula>|$$|',而公式的格式化将被自动插入。%
(这不能用于编号公式;附录 D 说明如何同时处理这种公式。)

%If you've had previous experience typing mathematical papers, you probably
%have been thinking, ``What about ^{equation numbers}? When is this book
%going to talk about them?'' Ah yes, now is the time to discuss those sneaky
%little labels that appear off to the side of displays. If you type
%\begindisplay
%|$$|\<formula>|\eqno|\<formula>|$$|
%\enddisplay
%\TeX\ will display the first formula and it will also put an equation number
%(the second formula) at the right-hand margin. For example,
%\begintt
%$$x^2-y^2 = (x+y)(x-y).\eqno(15)$$
%\endtt
%^^|\eqno|
%will produce this:
%$$x^2-y^2 = (x+y)(x-y).\eqno(15)$$
%You can also get equation numbers at the left-hand margin, with ^|\leqno|.
%For example,
%\begintt
%$$x^2-y^2 = (x+y)(x-y).\leqno(16)$$
%\endtt
%will produce this:
%$$x^2-y^2 = (x+y)(x-y).\leqno(16)$$
%Notice that you always give the equation number second, even when it is going
%to appear at the left. Everything from the |\eqno| or |\leqno| command to
%the |$$| that ends the display is the equation number. Thus, you're not
%allowed to have two equation numbers in the same display; but there's a
%way to get around that restriction, as we'll see later.
如果你有以前排版数学文章的经验,可能会想,
``方程编号放在哪里?
本书什么时候讨论它?''
的确,现在是该讨论在陈列公式外边好远才偷偷摸摸出现的小标记了。%
如果输入
\begindisplay
|$$|\<formula>|\eqno|\<formula>|$$|
\enddisplay
\1\TeX\ 就陈列显示出第一个公式,并且还把方程编号(第二个公式)放在右页边处。%
例如,
\begintt
$$x^2-y^2 = (x+y)(x-y).\eqno(15)$$
\endtt
得到的是:
$$x^2-y^2 = (x+y)(x-y).\eqno(15)$$
也可以把方程编号放在左页边处,用的是 |\leqno|。%
例如,
\begintt
$$x^2-y^2 = (x+y)(x-y).\leqno(16)$$
\endtt
得到的是:
$$x^2-y^2 = (x+y)(x-y).\leqno(16)$$
注意,方程编号总是放在第二位,即使要出现在左边。%
从 |\eqno| 到 |\leqno| 命令到结束陈列公式的 |$$| 的所有东西都是方程编号。%
因此,不允许在同一陈列公式中出现两个方程编号;
但是有一种方法可以绕过这个限制,我们后面要讨论它。

%\danger Nowadays people are using right-hand equation numbers more and more,
%because a display most often comes at the end of a sentence or clause, and
%the right-hand convention keeps the number from intruding into the clause.
%Furthermore, it's often possible to save space when a displayed equation
%follows a short text line, since less space is needed above the display;
%such savings are not possible with |\leqno|, because there's no room for
%overlap. For example, there is less space above display~(15) than there is
%above~(16) in our illustrations of\/ |\eqno| and |\leqno|, although the
%formulas and text are otherwise identical.
\danger 现在,人们越来越多地使用方程的右边编号,
因为陈列公式常常出现在句子或条款的结尾处,
右边约定使得编号与条款不混淆。%
还有,当陈列公式跟在一个短行后面时,还常常可以节约空间,
因为在陈列公式上面需要的间距更少;
用 |\leqno| 时就不能这样节约空间,
因为没有可以叠起来的空白。%
例如,在我们关于 |\eqno| 和 |\leqno| 的示例中,公式 (15) 上面的空间比公式 (16)~%
要少,虽然公式和文本这些其它东西都是一样的。

%\danger If you look closely at (15) and (16) above, you can see that the
%displayed formulas have been centered without regard to the presence of
%the equation numbers. But when a formula is large, \TeX\ makes sure that
%it does not interfere with its number; the equation number may even be
%placed on a line by itself.
\danger 如果你仔细观察上面的 (15) 和 (16), 就会发现陈列公式已经居中了,
而没有计及方程编号的出现。%
但是当公式很大时,
 \TeX\ 要确保方程与编号之间互不干扰;
方程编号甚至可以单独放一行。

%\exercise How would you produce the following display?
%$$\prod_{k\ge0}{1\over(1-q^kz)}=
%    \sum_{n\ge0}z^n\bigg/\!\!\prod_{1\le k\le n}(1-q^k).\eqno(16')$$
%\answer |$$\prod_{k\ge0}{1\over(1-q^kz)}=|\parbreak
%        |   \sum_{n\ge0}z^n\bigg/\!\!\prod_{1\le k\le n}(1-q^k).\eqno(16')$$|
\exercise 怎样单独下列陈列公式?
$$\prod_{k\ge0}{1\over(1-q^kz)}=
    \sum_{n\ge0}z^n\bigg/\!\!\prod_{1\le k\le n}(1-q^k).\eqno(16')$$
\answer |$$\prod_{k\ge0}{1\over(1-q^kz)}=|\parbreak
        |   \sum_{n\ge0}z^n\bigg/\!\!\prod_{1\le k\le n}(1-q^k).\eqno(16')$$|

%\dangerexercise Equation numbers are math formulas, typeset in text style.
%So how can you get an equation number like `\hbox{(3--1)}'
%(with an ^{en-dash})?
%\answer |\eqno\hbox{(3--1)}|.
\dangerexercise 公式编号是用文本样式排版的数学公式。
因此,怎样才能得到象`\hbox{(3--1)}'(有一个连接号)这样的公式编号?
\answer |\eqno\hbox{(3--1)}|。

%\ddangerexercise B. L. ^{User} tried typing `|\eqno(*)|' and `|\eqno(**)|',
%and he was pleased to discover that this produced the equation numbers
%`$(*)$' and `$(**)$'. \ [He had been a bit worried that they would come out
%`(*)' and `(**)' instead.] \ But then a few months later he tried
%`|\eqno(***)|' and got a surprise. What was it?
%\answer When you type an ^{asterisk} in math mode, plain \TeX\ considers
%|*| to be a binary operation. In the cases `|(*)|' and `|(**)|', the
%binary operations are converted to type~Ord, because they don't appear in
%a binary context; but the middle asterisk in `|(***)|' remains of type~Bin.
%So the result was `$(***)$'. To avoid the extra medium spaces, you can
%type `|\eqno(*{*}*)|'; or you can change ^|\mathcode||`*|, if you never use
%|*| as a binary operation.
\ddangerexercise ^{用户笨笨}试着输入了`|\eqno(*)|' 和 `|\eqno(**)|',
他高兴地发现,得到的方程编号是`$(*)$'和`$(**)$'。%
[ 他为结果不是 `(*)' 和 `(**)' 而感到有点困惑。]
但是,几个月后,他输入 `|\eqno(***)|' 却得到意想不到的结果。这个结果是什么?
\answer 当你在数学模式中输入^{星号}时,plain \TeX\ 将 |*| 视为二元运算符。
在 `|(*)|' 和 `|(**)|' 的情形,因为它们没出现在二元运算环境中,
二元运算符被转换为 Ord 类型;但 `|(***)|' 中间的星号将保留为 Bin 类型。
所以结果将为 `$(***)$'。为避免多余的空白,你可以输入 `|\eqno(*{*}*)|';
如果你从不将 |*| 用作二元运算符,你也可以改变 ^|\mathcode||`*|。

%\ddanger Somewhere in this manual there ought to be a description of exactly
%how \TeX\ displays formulas; i.e., how it centers them, how it places the
%equation numbers, how it inserts extra space above and below, and so on.
%Well, now is the time for those rules to be stated. They are somewhat
%complex, because they interact with things like |\parshape|, and because
%they involve several parameters that haven't been discussed yet. The purpose
%of the rules is to explain exactly what sorts of boxes, glue, and
%penalties are placed onto the current ^{vertical list} when a display occurs.
\ddanger \1应该在本手册的某些地方讨论一下 \TeX\ 到底是怎样排版陈列公式的:
即,怎样把它们居中,怎样放方程编号,怎样在上下插入额外间距等等。%
嗯,现在该讨论它们了。%
它们是有些复杂,因为它们与象 |\parshape| 这样的东西互相影响,
并且包括了几个还未讨论的参数。%
这些规则的目的就是当出现陈列公式时,要把什么样的盒子,粘连和惩罚%
放在当前垂直列中。

%\ddanger If a display occurs after, say, four lines of a paragraph, \TeX's
%internal register called ^|\prevgraf| will be equal to~4 when the display
%starts.  The display will be assumed to take three lines, so |\prevgraf|
%will become~7 when the paragraph is resumed at the end of the display
%(unless you have changed |\prevgraf| in the meantime). \TeX\ assigns
%special values to three \<dimen> parameters immediately after the opening
%|$$| is sensed:  ^|\displaywidth| and ^|\displayindent| are set to the
%line width~$z$ and the shift amount~$s$ for line number |\prevgraf|${}+2$,
%based on the current paragraph shape or hanging indentation. \ (Usually
%|\displaywidth| is the same as ^|\hsize|, and |\displayindent| is zero,
%but the paragraph shape can vary as described in Chapter~14.) \ Furthermore,
%^|\predisplaysize| is set to the effective width~$p$ of the line preceding
%the display, as follows: If there was no previous line (e.g., if the |$$|
%was preceded by ^|\noindent| or by the closing |$$| of another display),
%$p$~is set to $-16383.99999\pt$ (i.e., to the smallest legal dimension,
%$-$^|\maxdimen|).  Otherwise \TeX\ looks inside the hbox that was formed
%by the previous line, and sets $p$ to the position of the right edge of
%the rightmost box inside that hbox, plus the indentation by which the
%enclosing hbox has been moved right, plus two ems in the current font.
%However, if this value of~$p$ depends on the fact that glue in that hbox
%was stretching or shrinking---for example, if the ^|\parfillskip| glue is
%finite, so that the material preceding it has not been set at its natural
%width---then $p$~is set to |\maxdimen|. \ (This doesn't happen
%often, but it keeps \TeX\ machine independent, since $p$~never depends on
%quantities that may be rounded differently on different computers.)  \
%Notice that |\displaywidth| and |\displayindent| are not
%affected by |\leftskip| and |\rightskip|, but
%|\predisplaysize| is. The values of\/ |\displaywidth|,
%|\displayindent|, and |\predisplaysize|
%will be used by \TeX\ after the displayed formula has been
%read, as explained below; your program can examine them and/or change
%them, if you want the typesetting to be done differently.
\ddanger 如果陈列公式出现在段落的第 4 行后面,
在陈列公式开始时, \TeX\ 的内部寄存器 |\prevgraf| 就等于 4。%
陈列公式设置的行数为 3, 因此当在陈列公式结束处恢复到段落时,
~|\prevgraf| 就变成 7~(除非在其间你改动过 |\prevgraf|)。%
在读入开符号 |$$| 后, \TeX\ 紧接着就指定了三个 \<dimen> 参数的特殊值:
|\displaywidth| 设定的是行的宽度 $z$, |\displayindent| 按照当前段落的形状%
和悬挂缩进设定了第 |\prevgraf|${}+2$ 行的偏移量 $s$。%
(一般情况下,|\displaywidth| 与 |\hsize| 是一样的,
而 |\displayindent| 等于零,
但是段落形状可以象第十四章讨论的那样变化。)
还有,|\predisplaysize| 设定了陈列公式前面行的有效宽度 $p$, 如下:
如果前面没有行(比如,如果 |$$| 前面是 |\noindent| 或者是另一个陈列公式的闭符号 |$$|),
~$p$ 就设定为 $-16383.99999\pt$(即,可用的最小尺寸,$-$|\maxdimen|)。%
否则, \TeX\ 要根据前面的行所形成的 hbox 把 $p$ 设定为此 hbox 中最右边%
盒子的右边界的位置,加上把封装盒子向右移的缩进,再加上当前字体的 2 em。%
但是,如果 $p$ 的这个值由此盒子中粘连的伸缩所确定——例如,
如果 |\parfillskip| 粘连是有限的,使得在其前面的内容不是其自然宽度——%
那么 $p$ 就设定为 |\maxdimen|。%
(这不是常常出现,但是它保持了 \TeX\ 系统的独立性,
因为 $p$ 不依赖于不同计算机的四舍五入得到的量。)
注意,|\displaywidth| 和 |\displayindent| 不受 |\leftskip| 和 |\rightskip| 影响,
但是 |\predisplaysize| 却受其影响。%
在陈列公式读入后, \TeX\ 才用到 |\displaywidth|, %
|\displayindent| 和 |\predisplaysize| 的值,其讨论见下面;
如果要得到不同的排版,可用检验和/或改变它们。

%\ddanger After a display has been read, \TeX\ converts it from a math list
%to a horizontal list~$h$ in display style, as explained in
%Appendix~G\null.  An equation number, if present, is processed in text
%style and put into an hbox~$a$ with its natural width. Now the fussy
%processing begins: Let $z$, $s$, and~$p$ be the current values of\/\
%|\displaywidth|, |\displayindent|, and
%|\predisplaysize|. Let $q$ and~$e$ be zero if there is no equation
%number; otherwise let~$e$ be the width of the equation number, and let~$q$
%be equal to $e$~plus one quad in the symbols font (i.e., in
%^|\textfont||2|). Let $w_0$ be the natural width of the displayed
%formula~$h$. If $w_0+q\le z$, list~$h$~is packaged in an hbox~$b$ having
%its natural width~$w_0$.  But if $w_0+q>z$ (i.e., if the display is too
%wide to fit at its natural width), \TeX\ performs the following ``^{squeeze
%routine}'':  If $e\ne0$ and if there is enough shrinkability in the
%displayed formula~$h$ to reduce its width to $z-q$, then list $h$ is
%packaged in an hbox~$b$ of width~$z-q$. Otherwise $e$ is set to zero, and
%list~$h$ is packaged in a (possibly overfull) hbox~$b$ of width $\min(w_0,z)$.
\ddanger 在读入陈列公式后, \TeX\ 把它从数学列转换到陈列样式的水平列 $h$,
就象在附录 G 中讨论的那样。%
如果出现方程编号,就把它按文本样式处理,并且放在宽度为自然宽度的 hbox 中。%
现在进行烦琐的处理:
设 $z$, ~$s$ 和 $p$ 是 |\displaywidth|, |\displayindent| 和 |\predisplaysize|~%
的当前值。%
如果没有方程编号,就设 $q$ 和 $e$ 为零;
否则,设 $e$ 为方程编号的宽度,
并且设 $q$ 等于 $e$ 加上符号字体(即,|\textfont||2|)的一个 quad。%
设 $w_0$ 是陈列公式 $h$ 的自然宽度。%
如果 $w_0+q\le z$, 就把列 $h$ 包装进宽度为自然宽度 $w_0$ 的 hbox~~$b$ 中。%
但是如果 $w_0+q>z$~(即,如果陈列公式太宽按照自然宽度放不下),
那么 \TeX\ 执行下列``挤压程序'':
如果 $e\ne0$ 并且在陈列公式 $h$ 中有足够的收缩性把其宽度减小到 $z-q$,
那么列 $h$ 就包装进宽度为 $z-q$ 的 hbox~~$b$ 中。%
否则,把 $e$ 设为零,并且把列 $h$ 包装进宽度为 $\min(w_0,z)$ 的 hbox~~$b$ 中%
(可能溢出)。

%\ddanger (Continuation.) \ \TeX\ tries now to center the display without
%regard to the equation number. But if such centering would make it too close
%to that number (where ``too close'' means that the space between them is less
%than the width~$e$), the equation is either centered in the remaining space
%or placed as far from the equation number as possible. The latter alternative
%is chosen only if the first item on list~$h$ is glue, since \TeX\ assumes that
%such glue was placed there in order to control the spacing precisely.
%But let's state the rules more formally: Let~$w$ be the width of box~$b$.
%\TeX\ computes a displacement~$d$, to be used later when positioning box~$b$,
%by first setting $d={1\over2}(z-w)$. If $e>0$ and if $d<2e$, then $d$~is
%reset to ${1\over2}(z-w-e)$ or to zero, where zero is chosen if list~$h$
%begins with a glue item.
\ddanger (续) 现在 \TeX\ 试着把陈列公式居中,而不管方程编号。%
但是如果这样的居中使得它与此编号太近(``太近''就是它们之间的间距小于宽度 $e$),
\1那么方程在剩下的间距上居中,或者放在离方程编号尽可能远的地方。%
只有当列 $h$ 上的第一个项目是粘连时,才选择后一种方法,
因为 \TeX\ 假定这样的粘连放在那里是为了精确地控制间距。%
但是我们要正式给出规则:
设 $w$ 是盒子 $b$ 的宽度。%
通过首先设 $d={1\over2}(z-w)$ 来计算位移量 $d$——当放置盒子 $b$ 时要用到它。%
如果 $e>0$ 并且 $d<2e$, 那么 $d$ 被重新设定为 ${1\over2}(z-w-e)$ 或者零,
其中只有列 $h$ 的开头为一个粘连项目时才选择零。

%\ddanger (Continuation.) \ \TeX\ is now ready to put things onto the current
%vertical list, just after the material previously constructed for the
%paragraph-so-far. First comes a ^{penalty} item, whose cost is an integer
%parameter called ^|\predisplaypenalty|. Then comes glue. If $d+s\le p$,
%or if there was a left equation number (|\leqno|), \TeX\ sets $g_a$ and
%$g_b$ to glue items specified by the parameters ^|\abovedisplayskip|
%and ^|\belowdisplayskip|, respectively; otherwise $g_a$ and $g_b$ become
%glue items corresponding to ^|\abovedisplayshortskip| and
%^|\belowdisplayshortskip|. \ [Translation: If the predisplaysize is short
%enough so that it doesn't overlap the displayed formula, the glue above and
%below the display will be ``short'' by comparison with the glue that is
%used when there is an overlap.] \ If $e=0$ and if there is an |\leqno|,
%the equation number is appended as an hbox by itself, shifted right~$s$ and
%preceded by interline glue as usual; an infinite penalty is also appended,
%to prevent a page break between this number and the display. Otherwise
%a glue item~$g_a$ is placed on the vertical list.
\ddanger (续) 现在 \TeX\ 要把这些东西放在当前垂直列,紧接在前面段落所%
构造的内容后面。%
首先出现的是一个惩罚项目,它的成本是整数参数 |\predisplaypenalty|。%
接着出现的是粘连。%
如果 $d+s\le p$, 或者如果方程编号在左边(|\leqno|), 那么 \TeX\ 分别把 $g_a$~%
和 $g_b$ 设置为给定的粘连项目参数 |\abovedisplayskip| 和 |\belowdisplayskip|;
否则 $g_a$ 和 $g_b$ 就变成对应于 |\abovedisplayshortskip| 和%
~|\belowdisplayshortskip| 的粘连项目。%
[转换:如果 predisplaysize 足够小,使得它与陈列公式不重叠,
那么陈列公式上下的粘连就比有重叠时的粘连要``小''。]
如果 $e=0$ 并且如果有 |\leqno|, 那么方程编号就就象一个 hbox 一样追加在自己后面,
向右平移 $s$ 并且在前面象通常那样放上行间粘连;
还要追加一个无限大惩罚,来防止这个编号与陈列公式之间断行。%
否则,就把粘连项目 $g_a$ 放在垂直列中。

%\ddanger (Continuation.) \ Now comes the displayed equation itself. If
%$e\ne0$, the equation number box~$a$ is combined with the formula box~$b$ as
%follows: Let~$k$ be a kern of width $z-w-e-d$. In the |\eqno| case, box~$b$
%is replaced by an hbox containing $(b,k,a)$; in the |\leqno| case, box~$b$
%is replaced by an hbox containing $(a,k,b)$, and $d$~is set to zero. In all
%cases, box~$b$ is then appended to the vertical list, shifted right by~$s+d$.
\ddanger (续) 现在出现的是陈列公式自己了。%
如果 $e\ne 0$, 那么方程编号的盒子 $a$ 就与公式的盒子 $b$ 如下组合起来:
设 $k$ 为宽度为 $z-w-e-d$ 的紧排。%
在 |\eqno| 的情况下,盒子 $b$ 由包含 $(b,k,a)$ 的 hbox 代替;
在 |\leqno| 的情况下,盒子 $b$ 由包含 $(a,k,b)$ 的 hbox 代替,
并且把 $d$ 设为零。%
在所有情形下,接下来都把盒子 $b$ 追加到垂直列中,向右平移 $s+d$。

%\ddanger (Continuation.) \ The final task is to append the glue or the
%equation number that follows the display. If there was an |\eqno| and if
%$e=0$, an infinite penalty is placed on the vertical list, followed by the
%equation number box~$a$ shifted right by $s+z$ minus its width, followed
%by a penalty item whose cost is the value of\/ ^|\postdisplaypenalty|.
%Otherwise a penalty item for the |\postdisplaypenalty| is appended
%first, followed by a glue item for~$g_b$ as specified above. \TeX\ now
%adds~3 to |\prevgraf| and returns to horizontal mode, ready to resume the
%paragraph.
\ddanger (续) 最后一个任务就是追加上粘连或跟在陈列公式下面的方程编号。%
如果有一个 |\eqno| 并且 $e=0$, 就在垂直列中放一个无限大惩罚;
后面是方程编号的盒子 $a$, 它向右平移 $s+z$ 减掉其宽度的距离;
再接下来是成本为 |\postdisplaypenalty| 的值的惩罚项目。%
否则,首先追加的是 |\postdisplaypenalty| 的惩罚项目,
接着是由上面给出的 $g_b$ 的粘连项目。%
现在, \TeX\ 把 |\prevgraf| 增加 3 并且返回水平模式,
就回复到段落。

%\ddanger One consequence of these rules is that you can force an equation
%number to appear on a line by itself by making its width zero, i.e.,
%by saying either `|\eqno|^|\llap||{$|\<formula>|$}|' or
%`|\leqno|^|\rlap||{$|\<formula>|$}|'. This makes $e=0$, and
%the condition $e=0$ controls \TeX's positioning logic, as explained
%in the rules just given.
\ddanger 这些规则的一个结果就是通过把方程编号盒子的宽度变成零,
即,使用`|\eqno||\llap||{$|\<formula>|$}|'%
或者`|\leqno||\rlap||{$|\<formula>|$}|', 你可以把它单独放在一行上。%
它使得 $e=0$, 并且条件 $e=0$ 在逻辑上控制着 \TeX\ 的放置方法,
就象刚刚讨论的规则那样。

%\ddanger Plain \TeX\ sets |\predisplaypenalty=10000|, because fine
%printers traditionally shun displayed formulas at the very top of a page.
%You can change |\predisplaypenalty| and
%|\postdisplaypenalty| if you want to encourage or discourage
%page breaks just before or just after a display. For example,
%`\hbox{|$$\postdisplaypenalty=|}\allowbreak\hbox{|-10000|\<formula>|$$|}'
%will force a page break, putting the formula at the bottom line. It is better
%to force a ^{page break} this way than to say ^|\eject| right after |$$...$$|;
%such an eject (which follows the |\belowdisplayskip| glue below the
%display) causes the page to be short, because it leaves unwanted glue at
%the bottom.
\ddanger Plain \TeX\ 设置 |\predisplaypenalty=10000|,
因为按照惯例,好的排版应该避免把陈列公式放在页面紧顶部的地方。%
如果你希望或者不希望正好在陈列公式前或后分页,
就可以改变 |\predisplaypenalty| 和 |\postdisplaypenalty|。%
例如,`\hbox{|$$\postdisplaypenalty=|}\allowbreak\hbox{|-10000|\<formula>|$$|}'
就强制分页,把公式放在最底的一行了。%
这种强制分页的方法比在 |$$...$$| 输入 |\eject| 要好;
这样的 eject~(它跟在陈列公式下面的 |\belowdisplayskip| 粘连后)%
得到一个短页面,因为它在底部留下了不想要的粘连。

%\ddangerexercise Read the rules carefully and deduce the final position
%of `$x=y$' in the formula
%\begintt
%$$\quad x=y \hskip10000pt minus 1fil \eqno(5)$$
%\endtt
%assuming that there is no hanging indentation. Also consider |\leqno|
%instead of\/ |\eqno|.
%\answer Assuming that |\hsize| is less than $10000\pt$, the natural width of
%this equation will be too large to fit on a line; also, |\quad| specifies
%glue at the left. Therefore `$x=y$' will appear exactly $1\em$ from
%the left, and `(5)' will appear flush right. \ (The widths will satisfy
%^^{displays, non-centered} $w=z-q$, $d=0$, $k=q-e=18\rm\,mu$.) \
%In the case of\/ |\leqno|, `(5)' will appear flush left, followed by
%one quad of space in |\textfont2|, followed by one quad of space in the
%current text font, followed by `$x=y$'.
\ddangerexercise \1详细阅读上面的规则,假定没有悬挂缩进,推导出公式
\begintt
$$\quad x=y \hskip10000pt minus 1fil \eqno(5)$$
\endtt
中`$x=y$'的最后位置。要是将 |\eqno| 改为 |\leqno| 呢?
\answer 假设 |\hsize| 小于 $10000\pt$,公式的自然宽度将太大而无法放在一行;
另外,|\quad| 在左边指定了一个粘连。
因此 `$x=y$' 将出现在离左侧 $1\em$ 的位置,而 `(5)' 将靠右对齐。%
(各宽度将满足 ^^{displays, non-centered}$w=z-q$,$d=0$,$k=q-e=18\rm\,mu$。)
改为 |\leqno| 以后,`(5)' 将靠左对齐,
后面跟着 |\textfont2| 字体的 1 quad 的空白,
后面跟着当前文本字体的 1 quad 空白,后面跟着 `$x=y$'。

%\ddanger \TeX\ also allows ``^{alignment displays},'' which are not
%processed in math mode because they contain no formulas at the outer
%level. An alignment display is created by commands of the general form
%\begindisplay
%|$$|\<assignments>|\halign{|\<alignment>|}|\<assignments>|$$|
%\enddisplay
%where the \<assignments> are optional things like parameter changes that do not
%produce any math lists. In such displays, the |\halign| is processed exactly
%as if it had appeared in vertical mode, and it will construct a vertical
%list~$v$ as usual, except that each row of the alignment will be shifted
%right by the ^|\displayindent|. After the alignment and the closing
%assignments have been processed, \TeX\ will put a ^|\predisplaypenalty|
%item and some ^|\abovedisplayskip| glue on the main vertical list, followed
%by~$v$, followed by a ^|\postdisplaypenalty| item and ^|\belowdisplayskip|
%glue. Thus, alignment displays are essentially like ordinary alignments,
%except that they can interrupt paragraphs; furthermore, they are embedded in
%glue and penalties just like other displays. The ^|\displaywidth| and
%^|\predisplaysize| do not affect the result, although you could use
%those parameters in your ^|\halign|. An entire alignment display is considered
%to be only three lines long, as far as ^|\prevgraf| is concerned.
\ddanger  \TeX\ 还允许``对齐的陈列公式'',
它们不在数学模式下处理,因为在外层没有包含公式。%
对齐的陈列公式由一般形式为
\begindisplay
|$$|\<assignments>|\halign{|\<alignment>|}|\<assignments>|$$|
\enddisplay
的命令生成,其中 \<assignments> 是象参数那样的可选内容,它不输出任何数学列。%
在这样的陈列公式中,|\halign| 就象它出现在垂直模式中那样处理,
并且构造出象通常那样的垂直列 $v$, 只是对齐的每行后向右平移了 |\displayindent|。%
在对齐和闭对齐都处理完后, \TeX\ 将把一个 |\predisplaypenalty| 项目%
和一些 |\abovedisplayskip| 粘连放在主垂直列,
接着是 $v$, 再接下来是一个 |\postdisplaypenalty| 项目和 |\belowdisplayskip| 粘连。%
这样,对齐的陈列公式本质上就象普通的对齐一样了,
只是它们可以插在段落中间;
还有,它们就象其它陈列公式那样嵌入在粘连和惩罚中。%
|\displaywidth| 和 |\predisplaysize| 不会影响结果,
虽然你可以在 |\halign| 使用它们。%
至于 |\prevgraf|, 这个对齐列表只看作三行长。

%\subsection Multi-line displays. OK, the use of displayed formulas is
%very nice. But when you try typing a lot of manuscripts you will run into
%some displays that don't fit the simple pattern of a one-line formula with
%or without an equation number. Plain \TeX\ provides special control
%sequences that will cover most of the remaining cases.
\subsection 多行陈列公式.
好,陈列公式非常有用。%
但是当你要排版许多文稿时,就会发现有些陈列公式只用有无编号的单行方程这种%
简单式样不合适。%
Plain \TeX\ 提供了特殊控制系列来解决大多数剩下的问题。

%Multi-line displays usually consist of several equations that should be
%lined up by their `$=$'~signs, as in
%$$\eqalign{X_1+\cdots+X_p&=m,\cr
%           Y_1+\cdots+Y_q&=n.\cr}$$
%The recommended procedure for such a display is to use ^|\eqalign|,
%which works with special markers |&| ^^{ampersand} and ^|\cr| that we
%have already encountered in connection with |\cases| and |\matrix|
%in Chapter~18. Here's how to type this particular one:
%\begintt
%$$\eqalign{X_1+\cdots+X_p&=m,\cr
%           Y_1+\cdots+Y_q&=n.\cr}$$
%\endtt
%There can be any number of equations in an |\eqalign|; the general pattern is
%$$\halign{\indent#&#\hfil\cr
%|\eqalign{|&\<left-hand side$_1$>|&|\<right-hand side$_1$>|\cr|\cr
%           &\<left-hand side$_2$>|&|\<right-hand side$_2$>|\cr|\cr
%\noalign{\vskip-2pt}
%           &\qquad\vdots\cr
%           &\<left-hand side$_n$>|&|\<right-hand side$_n$>|\cr}|\cr}$$
%where each \<right-hand side> starts with the symbol on which you want
%alignment to occur. For example, every right-hand side often begins
%with an $=$~sign. The equations will be typeset in display style.
多行陈列公式通常由几个方程组成,它们的`$=$'应当对齐,就象下面这样:
$$\eqalign{X_1+\cdots+X_p&=m,\cr
           Y_1+\cdots+Y_q&=n.\cr}$$
这样的陈列公式建议使用的命令是 |\eqalign|,
它用到特殊标记 |&| 和 |\cr|, 这些我们在第十八章中与 |\cases| 和 |\matrix|~%
有关的地方已经见过。%
下面就是上面公式的输入方法:
\begintt
$$\eqalign{X_1+\cdots+X_p&=m,\cr
           Y_1+\cdots+Y_q&=n.\cr}$$
\endtt
在 |\eqalign| 中可以有任意个方程;
一般样式为
$$\halign{\indent#&#\hfil\cr
|\eqalign{|&\<left-hand side$_1$>|&|\<right-hand side$_1$>|\cr|\cr
           &\<left-hand side$_2$>|&|\<right-hand side$_2$>|\cr|\cr
\noalign{\vskip-2pt}
           &\qquad\vdots\cr
           &\<left-hand side$_n$>|&|\<right-hand side$_n$>|\cr}|\cr}$$
其中每个 \<right-hand side> 以要对齐的符号开头。%
\1例如,每个右边都常常以 $=$ 开头。%
方程用陈列样式排版。

%\exercise In practice, the left-hand sides of aligned formulas are often
%blank, and the alignment is often done with respect to other symbols
%as well as~$=$. For example, the following display is typical; see if you
%can guess how the author typed it:
%$$\eqalign{T(n)\le T(2^{\lceil\lg n\rceil})
%    &\le c(3^{\lceil\lg n\rceil}-2^{\lceil\lg n\rceil})\cr
%    &<3c\cdot3^{\lg n}\cr
%    &=3c\,n^{\lg3}.\cr}$$ % from v2 p279
%\answer (Note in particular that the final `|.|'\ comes {\sl before\/} the
%final `|\cr|'.)
%\begintt
%$$\eqalign{T(n)\le T(2^{\lceil\lg n\rceil})
%    &\le c(3^{\lceil\lg n\rceil}-2^{\lceil\lg n\rceil})\cr
%    &<3c\cdot3^{\lg n}\cr
%    &=3c\,n^{\lg3}.\cr}$$
%\endtt
\exercise 在实践中,要对齐的公式的左边常常是空的,
对齐经常出现在其它符号以及 $=$ 上。
例如,下列陈列公式就比较典型;看看你是否知道作者是怎样输入的:
$$\eqalign{T(n)\le T(2^{\lceil\lg n\rceil})
    &\le c(3^{\lceil\lg n\rceil}-2^{\lceil\lg n\rceil})\cr
    &<3c\cdot3^{\lg n}\cr
    &=3c\,n^{\lg3}.\cr}$$ % from v2 p279
\answer (特别要注意最后的 `|.|' 要放在最后一个 `|\cr|'  {\sl 之前\/}。)
\begintt
$$\eqalign{T(n)\le T(2^{\lceil\lg n\rceil})
    &\le c(3^{\lceil\lg n\rceil}-2^{\lceil\lg n\rceil})\cr
    &<3c\cdot3^{\lg n}\cr
    &=3c\,n^{\lg3}.\cr}$$
\endtt

%The result of\/ |\eqalign| is a vertically centered box. This makes it easy to
%get a formula like
%$$\left\{
%\eqalign{\alpha&=f(z)\cr \beta&=f(z^2)\cr \gamma&=f(z^3)\cr}
%\right\}\qquad\left\{
%\eqalign{x&=\alpha^2-\beta\cr y&=2\gamma\cr}\right\}.$$ % meaningless
%You simply use |\eqalign| twice in the same line:
%\begintt
%$$\left\{
%\eqalign{\alpha&=f(z)\cr \beta&=f(z^2)\cr \gamma&=f(z^3)\cr}
%\right\}\qquad\left\{
%\eqalign{x&=\alpha^2-\beta\cr y&=2\gamma\cr}\right\}.$$
%\endtt
|\eqalign| 得到的结果是一个垂直居中的盒子。%
这使得很容易得到象下面这样的公式:
$$\left\{
\eqalign{\alpha&=f(z)\cr \beta&=f(z^2)\cr \gamma&=f(z^3)\cr}
\right\}\qquad\left\{
\eqalign{x&=\alpha^2-\beta\cr y&=2\gamma\cr}\right\}.$$ % meaningless
可以直接在同一行中使用 |\eqalign| 两次:
\begintt
$$\left\{
\eqalign{\alpha&=f(z)\cr \beta&=f(z^2)\cr \gamma&=f(z^3)\cr}
\right\}\qquad\left\{
\eqalign{x&=\alpha^2-\beta\cr y&=2\gamma\cr}\right\}.$$
\endtt

%\exercise Try your hand at the numbered two-line display % Polya/Szego V.29
%$$\eqalign{P(x)&=a_0+a_1x+a_2x^2+\cdots+a_nx^n,\cr
%   P(-x)&=a_0-a_1x+a_2x^2-\cdots+(-1)^na_nx^n.\cr}\eqno(30)$$
%[{\sl Hint:\/} Use the fact that |\eqalign| produces a vertically centered
%box; the equation number `(30)' is supposed to appear halfway between
%the two lines.]
%\answer |$$\eqalign{P(x)&=a_0+a_1x+a_2x^2+\cdots+a_nx^n,\cr|\parbreak
%        |   P(-x)&=a_0-a_1x+a_2x^2-\cdots+(-1)^na_nx^n.\cr}\eqno(30)$$|\par
\exercise 练练手,看看怎样得到带编号的双行陈列公式:
$$\eqalign{P(x)&=a_0+a_1x+a_2x^2+\cdots+a_nx^n,\cr
   P(-x)&=a_0-a_1x+a_2x^2-\cdots+(-1)^na_nx^n.\cr}\eqno(30)$$
[~{\KT{10}提示}:利用 |\eqalign| 得到垂直居中的盒子;
假定方程编号 `(30)' 出现在两行中间。]
\answer \1|$$\eqalign{P(x)&=a_0+a_1x+a_2x^2+\cdots+a_nx^n,\cr|\parbreak
        |   P(-x)&=a_0-a_1x+a_2x^2-\cdots+(-1)^na_nx^n.\cr}\eqno(30)$$|\par

%\exercise What happens if you forget the |&| in one equation of an |\eqalign|?
%\answer Both sides of that equation are considered to be on the left, so
%you get results that look like this:
%$$\openup-\jot
%\left\{\eqalign{\alpha&=f(z)\cr \beta&=f(z^2)\cr \gamma=f(z^3)\cr}
%  \right\}.$$
\exercise 如果在 |\eqalign| 的一个方程中你忘记输入 |&|,会出现什么情况?
\answer 方程两边的内容都被放在左边,因此你将得到类似下面的结果:
$$\openup-\jot
\left\{\eqalign{\alpha&=f(z)\cr \beta&=f(z^2)\cr \gamma=f(z^3)\cr}
  \right\}.$$

%\danger Multi-line formulas sometimes fit together in odd ways, and you'll
%find that every once in a~while you will want to move certain lines farther
%apart or closer together. If you type `^|\noalign||{|^|\vskip|\<glue>|}|'
%after any |\cr|, \TeX\ will insert the given amount of extra glue just
%after that particular line. For example,
%\begintt
%\noalign{\vskip3pt}
%\endtt
%will put $3\pt$ of additional space between lines. You can also change the
%amount of space before the first line, in the same way.
\danger 多行公式有时候用奇怪的方法放在一起,
并且偶尔你希望把行发散或紧凑一些。%
如果在任意 |\cr| 输入`|\noalign||{||\vskip|\<glue>|}|', 那么 \TeX\ 将%
在规定的行后面插入给定量的额外粘连。%
例如,
\begintt
\noalign{\vskip3pt}
\endtt
将把 $3\pt$ 的额外间距放在行之间。%
用同样的方法也可以改变第一行前面的间距。

%The next level of complexity occurs when you have several aligned
%equations with several equation numbers. Or perhaps some of the
%lines are numbered and others are not:
%$$\eqalignno{(x+y)(x-y)&=x^2-xy+yx-y^2\cr
%    &=x^2-y^2;&(4)\cr
%  (x+y)^2&=x^2+2xy+y^2.&(5)\cr}$$
%For this situation plain \TeX\ provides ^|\eqalignno|; you use it like
%|\eqalign|, but on each line that you want an equation number you add
%`|&|\<equation number>' just before the |\cr|. The example above was
%generated by
%\begintt
%$$\eqalignno{(x+y)(x-y)&=x^2-xy+yx-y^2\cr
%    &=x^2-y^2;&(4)\cr
%  (x+y)^2&=x^2+2xy+y^2.&(5)\cr}$$
%\endtt
%Notice that the second |&| is omitted unless there's an equation number.
\1当你有几个要对齐的方程和几个方程编号时,就出现更复杂的情况了。%
或许有些行有编号,有些又没有编号:
$$\eqalignno{(x+y)(x-y)&=x^2-xy+yx-y^2\cr
    &=x^2-y^2;&(4)\cr
  (x+y)^2&=x^2+2xy+y^2.&(5)\cr}$$
为此 plain \TeX\ 提供了 |\eqalignno|;
你可以象 |\eqalign| 那样使用它,但是在每个有方程编号的行上,只需要%
在 |\cr| 前面添加`|&|\<equation number>'即可。%
上面的例子由下面得到:
\begintt
$$\eqalignno{(x+y)(x-y)&=x^2-xy+yx-y^2\cr
    &=x^2-y^2;&(4)\cr
  (x+y)^2&=x^2+2xy+y^2.&(5)\cr}$$
\endtt
注意,如果没有方程编号,那么第二个 |&| 就可以省略。

%And there's also ^|\leqalignno|, which puts equation numbers at the left.
%In this case it is appropriate to move the `(4)' to the beginning
%of its equation:
%$$\leqalignno{(x+y)(x-y)&=x^2-xy+yx-y^2&(4)\cr
%    &=x^2-y^2;\cr
%  (x+y)^2&=x^2+2xy+y^2.&(5)\cr}$$
%Although the equation numbers appear at the left, you are still supposed to
%input them at the right, just as you do with |\leqno|; in other words,
%you should type
%`|$$\leqalignno{(x+y)(x-y)&...&(4)\cr...}$$|' to get the previous display.
还有一个 |\leqalignno|, 它把方程编号放在左边。%
在这种情况下,输出结果中`(4)'就相应地出现在方程前面:
$$\leqalignno{(x+y)(x-y)&=x^2-xy+yx-y^2&(4)\cr
    &=x^2-y^2;\cr
  (x+y)^2&=x^2+2xy+y^2.&(5)\cr}$$
虽然方程编号出现在左边,你却仍然要把它们输入在右边,就像 |\leqno| 那样;
换句话说,要得到前面的陈列公式,应该输入的是:
\begintt
$$\leqalignno{(x+y)(x-y)&...&(4)\cr...}$$
\endtt

%Caution: |\eqalignno| and |\leqalignno| both center the set of equations
%without regard to the widths of the equation numbers. If the equations or
%their numbers get too wide, they might overlap, yet no error message will
%be given.
注意:|\eqalignno| 和 |\leqalignno| 都把方程组居中,而不管方程编号的宽度。%
如果方程或它们的编号太宽,可能会重叠,但是却不给出错误信息。

%\exercise Typeset the following display: ^^|\gcd|
%$$\leqalignno{\gcd(u,v)&=\gcd(v,u);&(9)\cr
%    \gcd(u,v)&=\gcd(-u,v).&(10)\cr}$$ % v2 p316
%\answer |$$\leqalignno{\gcd(u,v)&=\gcd(v,u);&(9)\cr|\parbreak
%        |    \gcd(u,v)&=\gcd(-u,v).&(10)\cr}$$|
\exercise 排版下列陈列公式:
$$\leqalignno{\gcd(u,v)&=\gcd(v,u);&(9)\cr
    \gcd(u,v)&=\gcd(-u,v).&(10)\cr}$$ % v2 p316
\answer |$$\leqalignno{\gcd(u,v)&=\gcd(v,u);&(9)\cr|\parbreak
        |    \gcd(u,v)&=\gcd(-u,v).&(10)\cr}$$|

%\exercise And here's another one to try, just to keep in practice: ^^|\int|
%$$\vbox{
%\eqalignno{\biggl(\int_{-\infty}^\infty e^{-x^2}\,dx\biggr)^2
%  &=\int_{-\infty}^\infty\int_{-\infty}^\infty
%    e^{-(x^2+y^2)}\,dx\,dy\cr
%  &=\int_0^{2\pi}\int_0^\infty e^{-r^2}r\,dr\,d\theta\cr
%  &=\int_0^{2\pi}\biggl(-{e^{-r^2}\over2}
%    \bigg\vert_{r=0}^{r=\infty}\,\biggr)\,d\theta\cr
%  &=\pi.&(11)\cr}
%}$$ % cf Joy of TeX
%\answer %
%|$$\eqalignno{\biggl(\int_{-\infty}^\infty e^{-x^2}\,dx\biggr)^2|\parbreak
%        |  &=\int_{-\infty}^\infty\int_{-\infty}^\infty|\parbreak
%        |    e^{-(x^2+y^2)}\,dx\,dy\cr|\parbreak
%        |  &=\int_0^{2\pi}\int_0^\infty e^{-r^2}r\,dr\,d\theta\cr|\parbreak
%        |  &=\int_0^{2\pi}\biggl(-{e^{-r^2}\over2}|\parbreak
%        |    \bigg|\||_{r=0}^{r=\infty}\,\biggr)\,d\theta\cr|\parbreak
%        |  &=\pi.&(11)\cr}$$| ^^|\bigg|
\exercise 试试另一个,只是为了经常练习:^^|\int|
$$\vbox{
\eqalignno{\biggl(\int_{-\infty}^\infty e^{-x^2}\,dx\biggr)^2
  &=\int_{-\infty}^\infty\int_{-\infty}^\infty
    e^{-(x^2+y^2)}\,dx\,dy\cr
  &=\int_0^{2\pi}\int_0^\infty e^{-r^2}r\,dr\,d\theta\cr
  &=\int_0^{2\pi}\biggl(-{e^{-r^2}\over2}
    \bigg\vert_{r=0}^{r=\infty}\,\biggr)\,d\theta\cr
  &=\pi.&(11)\cr}
}$$ % cf Joy of TeX
\answer %
|$$\eqalignno{\biggl(\int_{-\infty}^\infty e^{-x^2}\,dx\biggr)^2|\parbreak
        |  &=\int_{-\infty}^\infty\int_{-\infty}^\infty|\parbreak
        |    e^{-(x^2+y^2)}\,dx\,dy\cr|\parbreak
        |  &=\int_0^{2\pi}\int_0^\infty e^{-r^2}r\,dr\,d\theta\cr|\parbreak
        |  &=\int_0^{2\pi}\biggl(-{e^{-r^2}\over2}|\parbreak
        |    \bigg|\||_{r=0}^{r=\infty}\,\biggr)\,d\theta\cr|\parbreak
        |  &=\pi.&(11)\cr}$$| ^^|\bigg|

%\danger Although |\eqalign| and |\eqalignno| look nearly the same, there's
%really a fundamental distinction between them: |\eqalign| makes a single,
%vertically centered box, which is no wider than it needs to be; but
%|\eqalignno| generates a set of lines that have the full display width
%(reaching all the way to both margins). Thus, for example, you can use
%|\eqalign| several times in a display, but |\eqalignno| can appear only
%once. If you try to use ^|\eqno| in conjunction with |\eqalign|,
%you get a decent result, but if you try to use |\eqno| in connection
%with |\eqalignno| you'll get some sort of weird error message(s).
\danger \1虽然 |\eqalign| 和 |\eqalignno| 看起来几乎一样,
但是它们之间有一个根本的不同:
|\eqalign| 得到的是一个垂直居中的盒子,宽度不会超过所需要的宽度;
但是 |\eqalignno| 生成了一组行,宽度为整个陈列公式的宽度(每行都延伸到两个边界)。%
因此,例如,可以在一个陈列公式中使用几次 |\eqalign|,
但是只能用一次 |\eqalignno|。%
如果把 |\eqno| 与 |\eqalign| 联合起来用,得到的结果还可以,
但是如果把 |\eqno| 与 |\eqalignno| 联合起来用,就会得到某些奇怪的错误信息。

%\ddanger The definitions in Appendix~B reveal why |\eqalign| and |\eqalignno|
%behave differently: |\eqalign| is an
%abbreviation for ^|\vcenter||{|^|\halign||{...}}|, while
%|\eqalignno| is an abbreviation for |\halign to\displaywidth{...}|;
%thus the |\eqalignno| macro generates an ``^{alignment display}.''
\ddanger 附录 B 中的定义揭示了 |\eqalign| 和 |\eqalignno| 的性质如此不同的原因:
因为 |\eqalign| 的定义是 |\vcenter||{||\halign||{...}}|,
而 |\eqalignno| 的定义是 |\halign to\displaywidth{...}|;
所以,宏 |\eqalignno| 得到的是``陈列的对齐阵列''。

%\ddanger This difference between |\eqalign| and |\eqalignno| has two
%interesting consequences. \ (1)~It's impossible to break an |\eqalign|
%between pages, but an |\eqalignno| can be broken. In fact, you can
%{\sl force\/} a ^{page break} after a particular line if you insert
%`^|\noalign||{|^|\break||}|'
%after the |\cr| for that line. You can prohibit {\sl all\/} breaks
%in an |\eqalignno| if you set ^|\interdisplaylinepenalty||=10000|; or you
%can enclose the whole works in a ^|\vbox|:
%\begintt
%$$\vbox{\eqalignno{...}}$$
%\endtt
%(2) You can also insert a line of text between two equations, without
%losing the alignment. For example, consider the two displays
%$$\eqalignno{x&=y+z\cr
%  \noalign{\hbox{and}}
%  x^2&=y^2+z^2.\cr}$$
%These were actually generated as a single display by typing
%\begintt
%$$\eqalignno{x&=y+z\cr
%  \noalign{\hbox{and}}
%  x^2&=y^2+z^2.\cr}$$
%\endtt
%Therefore the fact that their $=$ signs line up is not just a lucky
%coincidence. Sometimes you will want to adjust the spacing above or below
%such a line of inserted text, by putting a |\vskip| or two inside of the
%|\noalign{...}|.  Incidentally, this example also shows that it is
%possible to use |\eqalignno| without giving any equation numbers.
\ddanger |\eqalign| 和 |\eqalignno| 之间的这个差别有两个有意思的后果:
(1). 不可能在 |\eqalign| 中分页,但是在 |\eqalignno| 中可以。%
实际上,如果在某行的 |\cr| 后插入`|\noalign||{||\break||}|',
就可以在此行后{\KT{9}强制}分页。%
如果要禁止 |\eqalignno| 中的{\KT{9}所有}分页,可以设置%
~|\interdisplaylinepenalty||=10000|;
或者可以把整个公式组放在 |\vbox| 中:
\begintt
$$\vbox{\eqalignno{...}}$$
\endtt
(2). 还可以在两个方程之间插入一行文本,而不影响对齐结果。%
例如,看看下面两个陈列公式:
$$\eqalignno{x&=y+z\cr
  \noalign{\hbox{and}}
  x^2&=y^2+z^2.\cr}$$
它们实际上由一个陈列公式输入:
\begintt
$$\eqalignno{x&=y+z\cr
  \noalign{\hbox{and}}
  x^2&=y^2+z^2.\cr}$$
\endtt
因此,它们的 $=$ 对齐不是因为凑巧。%
有时候可能要调整所插入文本上下的间距,只要输入 |\vskip|~%
或者两边放上 |\noalign{...}| 即可。%
顺便说一句,整个例子还表明,可以在无方程编号时使用 |\eqalignno|。%

%\ddangerexercise What happens if\/ |\eqalign| is substituted for
%|\eqalignno| in this last example?
%\answer You get the displayed box
%$$\eqalign{x&=y+z\cr
%  \noalign{\hbox{and}}
%  x^2&=y^2+z^2.\cr}$$
%Reason: The `and' occurs at the left of the |\eqalign| box, not at the
%left of the whole display, and the |\eqalign| box is centered as usual.
\ddangerexercise 如果上一个例子中,
用 |\eqalign| 代替 |\eqalignno| 会出现什么情况?
\answer 你将会得到陈列的盒子
$$\eqalign{x&=y+z\cr
  \noalign{\hbox{and}}
  x^2&=y^2+z^2.\cr}$$
原因:`and' 出现在 |\eqalign| 盒子的左边,
而不是在整个陈列公式的左边,且 |\eqalign| 盒子通常是居中的。

%\ddangerexercise Our friend Ben ^{User} got into trouble again when he tried to
%move an equation number up higher than its usual position, by typing this:
%^^|\raise|
%\begintt
%$$\eqalignno{...&\raise6pt\hbox{(5)}\cr}$$
%\endtt
%What was his oversight, and what could he have done instead?
%\answer By raising the equation number, he increased the line height,
%so \TeX\ put extra space between that line and the previous line
%when it calculated the inter-line glue. If he had said
%`^|\smash||{\raise...}|', he wouldn't have had that problem.
\ddangerexercise 我们的朋友笨笨又遇到麻烦了,
当他要把方程编号从原来的位置上抬高时,他输入的是:
\begintt
$$\eqalignno{...&\raise6pt\hbox{(5)}\cr}$$
\endtt
他的失误在哪里?怎样才能实现他的愿望?
\answer 提升公式编号将增加行高,
因此在计算行间粘连时,\TeX\ 将在该行和前一行之间留下额外的间距。
如果他输入 `^|\smash||{\raise...}|',就不会有此问题。

%\danger For other types of displays, plain \TeX\ provides ^|\displaylines|,
%which lets you display any number of formulas in any way you want,
%without any alignment. The general form is
%$$\halign{\indent\hfil#&#\hfil\cr
%|$$\displaylines{|&\<displayed formula$_1$>|\cr|\cr
%    &\<displayed formula$_2$>|\cr|\cr
%\noalign{\vskip-2pt}
%    &\qquad\vdots\cr
%    &\<displayed formula$_n$>|\cr}$$|\cr}$$
%Each formula will be centered, because |\displaylines| puts ^|\hfil| at
%the left and the right of each line; you can override this centering to
%get things flush left or flush right by inserting ^|\hfill|, which takes
%precedence over |\hfil|.
\danger \1对其它类型的陈列公式,~plain \TeX\ 提供了 |\displaylines|,
用它可以用任意想要的方式输入任意个公式,
但是没有对齐。%
一般形式为
$$\halign{\indent\hfil#&#\hfil\cr
|$$\displaylines{|&\<displayed formula$_1$>|\cr|\cr
    &\<displayed formula$_2$>|\cr|\cr
\noalign{\vskip-2pt}
    &\qquad\vdots\cr
    &\<displayed formula$_n$>|\cr}$$|\cr}$$
每个公式都要居中,因为 |\displaylines| 在每行的两边都放上了 |\hfil|;
插入比 |\hfil| 更强的 |\hfill| 可以把它左对齐或右对齐。

%\dangerexercise Use |\displaylines| to typeset the three-line display
%$$\displaylines{\hfill x\equiv x;\hfill\llap{(1)}\cr
%   \hfill\hbox{if}\quad x\equiv y\quad\hbox{then}\quad
%      y\equiv x;\hfill\llap{(2)}\cr
%   \hfill\hbox{if}\quad x\equiv y\quad\hbox{and}\quad
%      y\equiv z\quad\hbox{then}\quad
%      x\equiv z.\hfill\llap{(3)}\cr}$$
%\answer |$$\displaylines{\hfill x\equiv x;\hfill\llap{(1)}\cr|\parbreak
%        |   \hfill\hbox{if}\quad x\equiv y\quad\hbox{then}\quad|\parbreak
%        |      y\equiv x;\hfill\llap{(2)}\cr|\parbreak
%        |   \hfill\hbox{if}\quad x\equiv y\quad\hbox{and}\quad|\parbreak
%        |      y\equiv z\quad\hbox{then}\quad|\parbreak
%        |      x\equiv z.\hfill\llap{(3)}\cr}$$|\par\medskip\noindent
%There's also a trickier solution, which begins with
%\begintt
%$$\displaylines{x\equiv x;\hfil\llap{(1)}\hfilneg\cr
%\endtt
\dangerexercise 用 |\displaylines| 来排版下列的三行陈列公式:
$$\displaylines{\hfill x\equiv x;\hfill\llap{(1)}\cr
   \hfill\hbox{if}\quad x\equiv y\quad\hbox{then}\quad
      y\equiv x;\hfill\llap{(2)}\cr
   \hfill\hbox{if}\quad x\equiv y\quad\hbox{and}\quad
      y\equiv z\quad\hbox{then}\quad
      x\equiv z.\hfill\llap{(3)}\cr}$$
\answer |$$\displaylines{\hfill x\equiv x;\hfill\llap{(1)}\cr|\parbreak
        |   \hfill\hbox{if}\quad x\equiv y\quad\hbox{then}\quad|\parbreak
        |      y\equiv x;\hfill\llap{(2)}\cr|\parbreak
        |   \hfill\hbox{if}\quad x\equiv y\quad\hbox{and}\quad|\parbreak
        |      y\equiv z\quad\hbox{then}\quad|\parbreak
        |      x\equiv z.\hfill\llap{(3)}\cr}$$|\par\medskip\noindent
还有一种更巧妙的解法,即将第一行改为
\begintt
$$\displaylines{x\equiv x;\hfil\llap{(1)}\hfilneg\cr
\endtt

%\danger If you look closely at the multi-line displays in this chapter,
%you'll see that the baselines are farther apart than they are in normal
%text; mathematics publishers generally do this in order to make the
%displays easier to read. In accordance with this tradition, |\eqalign| and
%its relatives automatically increase the ^|\baselineskip|. If~you are
%making a multi-line display with \TeX's primitive ^|\halign| command,
%instead of using one of the plain \TeX\ macros, you might want to make
%this same baseline adjustment, and you can do it easily by saying
%`|$$\openup1\jot \halign{...}$$|'. The ^|\openup| macro increases
%^|\lineskip| and ^|\lineskiplimit| as well as |\baselineskip|. If
%you say `|\openup2\jot|', the lines are spread apart 2 extra units, where
%plain \TeX\ opens things up in units of $3\pt$. Since |$$...$$| acts as a
%^{group}, the effect of\/ |\openup| will disappear when the display is
%finished. Any \<dimen> can follow |\openup|, but it's customary to express the
%amount symbolically in terms of a ^|\jot| instead of using absolute units;
%^^{generic coding}
%then your manuscript can be used with a variety of different formats.
\danger 如果你仔细观察本章中多行陈列公式,就会发现其基线比正常文本的基线要%
分开得更宽;
数学出版界这样做是为了使得陈列公式更易于阅读。%
遵循这个惯例,|\eqalign| 等自动增加其 |\baselineskip|。%
如果你用 \TeX\ 的原始命令 |\halign| 而不是 plain \TeX\ 的宏来制作多行陈列公式,
就可能要做同样的基线调整,这只需输入`|$$\openup1\jot \halign{...}$$|'即可。%
宏 |\openup| 增加 |\lineskip|, |\lineskiplimit| 以及 |\baselineskip|。%
如果输入的是`|\openup2\jot|', 基线就多分开 2 个单位,
~plain \TeX\ 的设置为增加 $3\pt$ 的单位。%
因为 |$$...$$| 有编组的作用,所以当陈列公式结束时 |\openup| 的影响就消失了。%
|\openup| 后面可以跟任何 \<dimen>, 但是习惯上用 |\jot| 符号来表示这个量,
而不用绝对单位;
这样你的文稿就可用于各种格式了。

%\ddanger Plain \TeX's ^|\displaylines|, ^|\eqalignno|, and ^|\leqalignno|
%macros begin with `|\openup1\jot|'. If you don't want the lines to be
%opened up, you can cancel this by saying, e.g.,
%`|$$\openup-1\jot \eqalignno{...}$$|', because |\openup| has a cumulative
%effect.
\ddanger Plain \TeX\ 宏 |\displaylines|、|\eqalignno| 和 |\leqalignno|~
以 `|\openup1\jot|' 开头。如果不希望行分开太多,可以取消它,
比如输入 `|$$\openup-1\jot \eqalignno{...}$$|',因为 |\openup| 有累加性。

%\ddanger Suppose that you have decided to make a homegrown display having
%the general form `|$$\openup1\jot \halign{...}$$|'; and for convenience,
%let's suppose that the normal conventions of plain \TeX\ are in force, so
%that |\jot=3pt| and |\baselineskip=12pt|. Then the |\openup| macro
%changes the baselineskip distance to $15\pt$. It follows that the baseline
%of the text line that immediately precedes the display will be $15\pt$
%above the topmost baseline of the display, plus the ^|\abovedisplayskip|.
%But when the paragraph resumes, its next baseline will be only $12\pt$
%below the bottom baseline of the display, plus the ^|\belowdisplayskip|,
%because the |\baselineskip| parameter will have reverted to its normal
%value. The |\eqalignno| and |\displaylines| macros say
%`|\noalign{\vskip|$-d$|}|' before their first lines, where $d$ is
%the net amount of opening-up, in order to compensate for this difference.
\ddanger 假定你要自制一个其一般形式为 `|$$\openup1\jot \halign{...}$$|' 的陈列公式;
并且为了使用方便,假定使用的是 plain \TeX\ 的正常约定,
即有 |\jot=3pt| 且 |\baselineskip=12pt|。
因此,宏 |\openup| 把 baselineskip 的距离变成 $15\pt$。
由此可知,陈列公式前面紧邻的文本行的基线与它的最高基线的距离将等于
$15\pt$ 加上 |\abovedisplayskip|。但是在回到段落时,
陈列公式后面紧邻的文本行的基线与它的最低基线的距离只有 $12\pt$ 加上
|\belowdisplayskip|,这是因为参数 |\baselineskip| 已经回到其正常值。
为了弥补这个差距,在宏 |\eqalignno| 和 |\displaylines|
的第一行前面添加上 `|\noalign{\vskip|$-d$|}|',其中 $d$ 是多分开的距离。

%\subsection Long formulas. Our discussion of mathematics typing is almost
%complete; we need to deal with just one more problem: What should be
%done when a formula is so long that it doesn't fit on a single line?
\subsection 长公式.
\1我们关于数学排版的讨论差不多结束了;
现在只想要再处理一个问题:
当公式太长在一行放不下时怎么办?

%For example, suppose that you encounter the equation
%$$\hfuzz=20pt % overfull box tolerated here
%\sigma(2^{34}-1,2^{35},1)=
%  -3+(2^{34}-1)/2^{35}+2^{35}\!/(2^{34}-1)+7/2^{35}(2^{34}-1)
%  -\sigma(2^{35},2^{34}-1,1).$$ % from v2, 1st ed, p76
%You'll have to break it up somehow; \TeX\ has done its best to squeeze
%everything together by shrinking the spaces next to the $+$ and~$-$ signs
%to zero, but still the line has come out overfull.
例如,假定你碰到方程
$$\hfuzz=20pt % overfull box tolerated here
\sigma(2^{34}-1,2^{35},1)=
  -3+(2^{34}-1)/2^{35}+2^{35}\!/(2^{34}-1)+7/2^{35}(2^{34}-1)
  -\sigma(2^{35},2^{34}-1,1).$$ % from v2, 1st ed, p76
那么就不得不把它裂分;
 \TeX\ 只是尽力通过把 $+$ 和 $-$ 间距收缩到零,从而把所有内容挤在一起,
但是这样仍得到溢出的行。

%Let's try to break that equation just before the `$+7$'. One common way to
%do this is to type
%\begintt
%$$\eqalign{\sigma(2^{34}-1,2^{35},1)
%&=-3+(2^{34}-1)/2^{35}+2^{35}\!/(2^{34}-1)\cr
%&\qquad+7/2^{35}(2^{34}-1)-\sigma(2^{35},2^{34}-1,1).\cr}$$
%\endtt
%which yields
%$$\eqalign{\sigma(2^{34}-1,2^{35},1)
%  &=-3+(2^{34}-1)/2^{35}+2^{35}\!/(2^{34}-1)\cr
%  &\qquad+7/2^{35}(2^{34}-1)-\sigma(2^{35},2^{34}-1,1).\cr}$$
%The idea is to treat a long one-line formula as a two-line formula,
%using |\qquad| on the second line so that the second part of the formula
%appears well to the right of the $=$~sign on the first line.
我们试着在`$+7$'前面把方程裂分。%
一个常用的方法是输入
\begintt
$$\eqalign{\sigma(2^{34}-1,2^{35},1)
&=-3+(2^{34}-1)/2^{35}+2^{35}\!/(2^{34}-1)\cr
&\qquad+7/2^{35}(2^{34}-1)-\sigma(2^{35},2^{34}-1,1).\cr}$$
\endtt
它得到的是
which yields
$$\eqalign{\sigma(2^{34}-1,2^{35},1)
  &=-3+(2^{34}-1)/2^{35}+2^{35}\!/(2^{34}-1)\cr
  &\qquad+7/2^{35}(2^{34}-1)-\sigma(2^{35},2^{34}-1,1).\cr}$$
这个方法是把单行长公式变成两行的公式,
在第二行用 |\qquad| 给第一行的 $=$ 留下对应的空白。

%\exercise Explain how to deal with the following display. % v2 p107
%$$\eqalignno{x_nu_1+\cdots+x_{n+t-1}u_t
%   &=x_nu_1+(ax_n+c)u_2+\cdots\cr
%   &\qquad+\bigl(a^{t-1}x_n+c(a^{t-2}+\cdots+1)\bigr)u_t\cr
%   &=(u_1+au_2+\cdots+a^{t-1}u_t)x_n+h(u_1,\ldots,u_t).
%     \quad&(47)\cr}$$
%\answer |$$\eqalignno{x_nu_1+\cdots+x_{n+t-1}u_t|\parbreak
%        |   &=x_nu_1+(ax_n+c)u_2+\cdots\cr|\parbreak
%        |   &\qquad+\bigl(a^{t-1}x_n+c(a^{t-2}+\cdots+1)\bigr)u_t\cr|\parbreak
%        |   &=(u_1+au_2+\cdots+a^{t-1}u_t)x_n+h(u_1,\ldots,u_t).|\parbreak
%        |     \quad&(47)\cr}$$|\par\noindent
%You weren't expected to insert the `|\quad|' on the last line; such
%refinements usually can't be anticipated until you see the first proofs.
%But without that |\quad| the `(47)' would occur half a quad closer to the
%formula.
\exercise 怎样输入下列陈列公式?
$$\eqalignno{x_nu_1+\cdots+x_{n+t-1}u_t
   &=x_nu_1+(ax_n+c)u_2+\cdots\cr
   &\qquad+\bigl(a^{t-1}x_n+c(a^{t-2}+\cdots+1)\bigr)u_t\cr
   &=(u_1+au_2+\cdots+a^{t-1}u_t)x_n+h(u_1,\ldots,u_t).
     \quad&(47)\cr}$$
\answer |$$\eqalignno{x_nu_1+\cdots+x_{n+t-1}u_t|\parbreak
        |   &=x_nu_1+(ax_n+c)u_2+\cdots\cr|\parbreak
        |   &\qquad+\bigl(a^{t-1}x_n+c(a^{t-2}+\cdots+1)\bigr)u_t\cr|\parbreak
        |   &=(u_1+au_2+\cdots+a^{t-1}u_t)x_n+h(u_1,\ldots,u_t).|\parbreak
        |     \quad&(47)\cr}$$|\par\noindent
不指望你会在最后一行插入 `|\quad|';这种细调在看到初校稿之前不大可能做到。
但是没有这个 |\quad|,`(47)' 和公式的距离将减少半个 quad。

%\danger It's quite an art to decide how to ^{break long displayed formulas}
%into several lines; \TeX\ never attempts to break them, because no set of
%rules is really adequate. The author of a mathematical manuscript is
%generally the best judge of what to do, since break positions depend on
%subtle factors of mathematical exposition. For example, it is often
%desirable to emphasize some of the symmetry or other structure that
%underlies a formula, and such things require a solid understanding of
%exactly what is going on in that formula.
\danger 怎样把长的陈列公式裂分成几行是一项技术;
 \TeX\ 从不试图把它们裂分,因为实在没有一组适当的规则。%
一般情况下,数学文稿的作者是最好的裁判,因为断点的位置%
依赖于数学含义的微妙因素。%
例如,常常希望强调某些对称性或者公式的其它潜在意义,
并且这样的东西要求确实理解了此公式中要表达的意思。

%\begingroup\ninepoint
%\danger Nevertheless, it is possible to state a few rules of thumb about
%how to deal with long formulas in displays, since there are some
%principles that the best mathematical typesetters tend to follow:\enddang
\begingroup\ninepoint
\danger 然而,对处理陈列公式中长公式,可能有几个笨规则,
因为有几个最好的数学排版者要遵循的原理:\enddanger

%\smallskip
%\textindent{a)}Although formulas within a paragraph always break {\sl after\/}
%binary operations and relations, displayed formulas always break {\sl before\/}
%binary operations and relations. Thus, we didn't end the first line of
%our $\sigma(\,\ldots\,)$ example with `|(2^{34}-1)+|'; we ended it with
%`|(2^{34}-1)|' and began the second line with `|+|'.
\smallskip
\textindent{a)}虽然段落中的公式总是在二元运算和关系符号{\KT{9}之后}断行,
但是陈列公式总是在它们{\KT{9}之前}断行。%
因此,不要在 $\sigma(\,\ldots\,)$ 这个例子的`|(2^{34}-1)+|'处结束第一行;
应该在`|(2^{34}-1)|'处结束它,并且以`|+|'开始第二行。

%\smallskip
%\textindent{b)}When an equation is broken before a binary operation, the second
%line should start at least two quads to the right of where the innermost
%subformula containing that binary operation begins on the first line.
%For example, if you wish to break
%\begindisplay
%|$$\sum_{0<k<n}\left(|\<formula$_1$>|+|\<formula$_2$>|\right)$$|
%\enddisplay
%at the plus sign between \<formula$_1$> and \<formula$_2$>, it is almost
%mandatory to have the plus sign on the second line appear somewhat to the
%right of the large left parenthesis that corresponds to `|\left(|'.
\smallskip
\textindent{b)}\1当方程在二元运算前被裂分时,第二行至少在%
包含第一行此二元运算的最内层子公式右边的两个 quad 后面才开始。%
例如,如果要在 \<formula$_1$> 和 \<formula$_2$> 之间的加号处裂分公式
\begindisplay
|$$\sum_{0<k<n}\left(|\<formula$_1$>|+|\<formula$_2$>|\right)$$|
\enddisplay
那么几乎是人为地把第二行的加号出现在左侧大圆括号(它由`|\left(|'得到)的右边。

\endgroup

%\danger In the example just considered, special care is needed to break the
%formula into two lines, because ^|\left| and ^|\right| delimiters cannot be
%used in isolation; you can't have only |\left| in one line of a formula
%and only |\right| in the second. Furthermore, you'll want the two delimiters
%to be of the same size, even though they occur in different lines. The best
%solution is usually to choose the delimiter size yourself; for example,
%you could type
%\begindisplay
%|$$\eqalign{\sum_{0<k<n}\biggl(&|\<formula$_1$>|\cr|\cr
%|             &\qquad{}+|\<formula$_2$>|\biggr)\cr}$$|\cr
%\enddisplay
%if\/ ^|\bigg| delimiters are best. Notice that the |&|~markers don't occur
%at $=$~signs in this example, they just mark a point of alignment.
\danger 在刚刚讨论的例子中,想要特别注意要把公式裂分为两行,
因为 |\left| 和 |\right| 分界符不能单独使用;
不能在第一行只用 |\left| 和在第二行只用 |\right|。%
还有,如果要得到同样大小的两个分界符——即使它们在不同的行,
那么最好的方法通常是自己选定分界符的尺寸;
例如,如果 |\bigg| 最适合,可以输入
\begindisplay
|$$\eqalign{\sum_{0<k<n}\biggl(&|\<formula$_1$>|\cr|\cr
|             &\qquad{}+|\<formula$_2$>|\biggr)\cr}$$|\cr
\enddisplay
注意,在本例中,标记 |&| 后面没有 $=$ 号,它们仅仅是标记要对齐的点。

%\danger There's another way to break long formulas, sometimes called the
%{\sl^{two-line}\/} form. The idea is to put the first part of the formula
%almost ^{flush left}, and to put the second part almost ^{flush right},
%where ``almost flush'' means ``one quad away.'' Thus, the two-line form of
%the long $\sigma(\,\ldots\,)$ equation considered earlier is
%$$\displaylines{\quad\sigma(2^{34}-1,2^{35},1)
%  =-3+(2^{34}-1)/2^{35}+2^{35}\!/(2^{34}-1)\hfill\cr
%\hfill{}+7/2^{35}(2^{34}-1)-\sigma(2^{35},2^{34}-1,1).\quad\cr}$$
%It isn't difficult to get this two-line effect with ^|\displaylines|:
%\begintt
%$$\displaylines{\quad\sigma(2^{34}-1,2^{35},1)
%  =-3+(2^{34}-1)/2^{35}+2^{35}\!/(2^{34}-1)\hfill\cr
%\hfill{}+7/2^{35}(2^{34}-1)-\sigma(2^{35},2^{34}-1,1).\quad\cr}$$
%\endtt
%An extra `|{}|' was typed on the second line here so that \TeX\ would know
%that the `|+|' is a binary operation.  The two-line form is especially
%recommended for equations that have a long left-hand side; in that case the
%break generally comes just before the~$=$~sign.
\danger 还有另一种方法了裂分长公式,有时候把它称为{\KT{9}双行}形式。%
其想法是,公式的第一部分几乎居左,第二部分几乎居右,其中``几乎居…''表示%
``差一个 quad''。%
这样,前面讨论的长方程 $\sigma(\,\ldots\,)$ 的双行形式为
$$\displaylines{\quad\sigma(2^{34}-1,2^{35},1)
  =-3+(2^{34}-1)/2^{35}+2^{35}\!/(2^{34}-1)\hfill\cr
\hfill{}+7/2^{35}(2^{34}-1)-\sigma(2^{35},2^{34}-1,1).\quad\cr}$$
用 |\displaylines| 不难得到这种双行的效果:
\begintt
$$\displaylines{\quad\sigma(2^{34}-1,2^{35},1)
  =-3+(2^{34}-1)/2^{35}+2^{35}\!/(2^{34}-1)\hfill\cr
\hfill{}+7/2^{35}(2^{34}-1)-\sigma(2^{35},2^{34}-1,1).\quad\cr}$$
\endtt
这里第二行的额外`|{}|'是为了告诉 \TeX, `|+|'是一个二元运算。%
对左边长的方程特别推荐使用双行形式;
在这种情况下,断点一般选在 $=$ 好前面。

%\dangerexercise Typeset the following display:
%$$\displaylines{\quad\sum_{1\le j\le n}{1\over
%    (x_j-x_1)\ldots(x_j-x_{j-1})(x-x_j)(x_j-x_{j+1})
%    \ldots(x_j-x_n)}\hfill\cr
%  \hfill={1\over(x-x_1)\ldots(x-x_n)}.\quad(27)\cr}$$ % v2 p80
%\answer |$$\displaylines{\quad\sum_{1\le j\le n}{1\over|\parbreak
%        |    (x_j-x_1)\ldots(x_j-x_{j-1})(x-x_j)(x_j-x_{j+1})|\parbreak
%        |    \ldots(x_j-x_n)}\hfill\cr|\parbreak
%        |  \hfill={1\over(x-x_1)\ldots(x-x_n)}.\quad(27)\cr}$$|
\dangerexercise 排版下列陈列公式:
$$\displaylines{\quad\sum_{1\le j\le n}{1\over
    (x_j-x_1)\ldots(x_j-x_{j-1})(x-x_j)(x_j-x_{j+1})
    \ldots(x_j-x_n)}\hfill\cr
  \hfill={1\over(x-x_1)\ldots(x-x_n)}.\quad(27)\cr}$$ % v2 p80
\answer \1|$$\displaylines{\quad\sum_{1\le j\le n}{1\over|\parbreak
        |    (x_j-x_1)\ldots(x_j-x_{j-1})(x-x_j)(x_j-x_{j+1})|\parbreak
        |    \ldots(x_j-x_n)}\hfill\cr|\parbreak
        |  \hfill={1\over(x-x_1)\ldots(x-x_n)}.\quad(27)\cr}$$|

%\ddangerexercise If it is necessary to typeset a huge fraction like
%^^{fraction, huge}
%$$\def\\#1;{(#1;q^2)_\infty}
%q^{{1\over2}n(n+1)}\\ea;\\eq/a;\\caq/e;\\cq^2\!/ae;
%\over(e;q)_\infty(cq/e;q)_\infty$$
%in a single narrow column, you might have to break up the numerator and
%resort to
%\begindisplay
%$\displaystyle{\def\\#1;{(#1;q^2)_\infty}
%\displaystyle{q^{{1\over2}n(n+1)}\\ea;\\eq/a;\qquad\atop
%  \hfill\\caq/e;\\cq^2\!/ae;}
%\over(e;q)_\infty(cq/e;q)_\infty}$
%\enddisplay
%How would you specify the latter fraction to \TeX?
%% cf SIAM J Math Anal 7 (1976) p333; even longer ones appear on p334
%\answer |$$\def\\#1;{(#1;q^2)_\infty} % to save typing|\parbreak
%        |\displaystyle{q^{{1\over2}n(n+1)}\\ea;\\eq/a;\qquad\atop|\parbreak
%        |  \hfill\\caq/e;\\cq^2\!/ae;}|\parbreak
%        |\over(e;q)_\infty(cq/e;q)_\infty$$|
\ddangerexercise 倘若你必须在窄栏中输入像下面这样的大分式
$$\def\\#1;{(#1;q^2)_\infty}
q^{{1\over2}n(n+1)}\\ea;\\eq/a;\\caq/e;\\cq^2\!/ae;
\over(e;q)_\infty(cq/e;q)_\infty$$
\1你也许只能把分子断开,变成
\begindisplay
$\displaystyle{\def\\#1;{(#1;q^2)_\infty}
\displaystyle{q^{{1\over2}n(n+1)}\\ea;\\eq/a;\qquad\atop
  \hfill\\caq/e;\\cq^2\!/ae;}
\over(e;q)_\infty(cq/e;q)_\infty}$
\enddisplay
怎样用 \TeX\ 得到后一个分式?
\answer |$$\def\\#1;{(#1;q^2)_\infty} % to save typing|\parbreak
        |\displaystyle{q^{{1\over2}n(n+1)}\\ea;\\eq/a;\qquad\atop|\parbreak
        |  \hfill\\caq/e;\\cq^2\!/ae;}|\parbreak
        |\over(e;q)_\infty(cq/e;q)_\infty$$|

\endchapter

When a formula is too long for the page-width
and has to be broken into successive lines
(and we are now, of course, speaking of displayed formulae),
it should be broken, if possible, at the end of a natural `phrase';
if, for example, it is a much-bracketed formula,
it should be broken at the end of one of the major brackets
and not at an inner symbol.
This natural phrasing (as in music or speech)
makes for intelligibility between writer and reader
and should not be left to the compositor.
An author, when he finds himself writing a longish formula,
should indicate a convenient point of fracture in case of need.
\author ^{CHAUNDY}, ^{BARRETT}, and ^{BATEY}, %
 {\sl The Printing of Mathematics\/} (1954) % p38

\bigskip

Some authors use display with discretion,
some run even extremely long, complicated equations into the text,
while others tend to display every equation in the paper.
The tendency to overdisplay is probably more predominant
than the tendency to underdisplay;
for this reason it is possible for the copy editor to shorten
(and even improve) papers by running displayed material into text. $\ldots$
On the other hand, there are occasions when the copy editor needs
to suggest the display of complicated expressions that have been run into text,
particularly when it would involve a bad break at the end of a text line.
\author ELLEN ^{SWANSON}, {\sl Mathematics into Type\/} (1971) % p41

\vfill\eject\byebye
