% -*- coding: utf-8 -*-

\input macros

%\beginchapter Appendix H. Hyphenation
\beginappendix Appendix H. 连字算法

It's better to break a word with a hyphen than to stretch interword spaces
too much. Therefore \TeX\ tries to divide words into syllables when
there's no good alternative available.

But computers are notoriously bad at ^{hyphenation}. When the type\-setting of
newspapers began to be fully automated, ^{jokes} about ``the-rapists who
pre-ached on wee-knights'' soon began to circulate.

It's not hard to understand why machines have behaved poorly at this task,
because hyphenation is quite a difficult problem. For example, the word
`record' is supposed to be broken as `rec-ord' when it is a noun, but
`re-cord' when it is a verb. The word `hyphenation' itself is somewhat
exceptional; if `hy-phen-a-tion' is compared to similar words like
`con-cat-e-na-tion', it's not immediately clear why the `n' should
be attached to the `e' in one case but not the other. Examples like
`dem-on-stra-tion' vs.~`de-mon-stra-tive' show that the alteration of
two letters can actually affect hyphens that are nine positions away.

A good solution to the problem was discovered by Frank M. ^{Liang} during
1980--1982, and \TeX\ incorporates the new method. Liang's algorithm
works quickly and finds nearly all of the legitimate places to insert
hyphens; yet it makes few if any errors, and it takes up comparatively
little space in the computer. Moreover, the method is flexible enough
to be adapted to any language, and it can also be used to hyphenate
words in two languages simultaneously. Liang's Ph.D. thesis,
published by Stanford University's Department of Computer Science in~1983,
explains how to take a dictionary of hyphenated words and teach it to \TeX;
i.e., it explains how to compute tables by which \TeX\ will be able to
reconstruct most of the hyphens in the given dictionary, without error.

\def\\#1{$_{\kern\scriptspace#1}$}
\TeX\ hyphenates a given word by first looking for it in an ``^{exception
dictionary},'' which specifies the hyphen positions for words that deserve
special treatment. If the word isn't there, \TeX\ looks for {\sl^{patterns}\/}
in the word, and this is the key idea underlying Liang's method. Here's how
it works, using the word `hyphenation' as an example, when \TeX\ is
operating with the English-oriented patterns of plain \TeX\ format:
The given word is first extended by special markers at either end; in this
case we obtain
\begintt
.hyphenation.
\endtt
if `|.|' denotes the special marker. The extended word has subwords
\begintt
. h y p h e n a t i o n .
\endtt
of length one,
\begintt
.h hy yp ph he en na at ti io on n.
\endtt
of length two,
\begintt
.hy hyp yph phe hen ena nat ati tio ion on.
\endtt
of length three,
and so on. Each subword of length $k$ is a pattern that defines $k+1$ small
integer values relating to the desirability of hyphens in the positions
between and adjacent to its letters. We can show these values by attaching
them as subscripts; for example,
`{\tt\\0h\\0e\\2n\\0}' means that the values corresponding to the
subword `|hen|' are 0,~0, 2, and~0, where the~2 relates to hyphens
between the `|e|' and the `|n|'. The interletter values are entirely zero
for all subwords except those that match an entry in \TeX's current
``^{pattern dictionary}''; and in this case, only the subwords
\begindisplay
\tt\\0h\\0y$_3$p\\0h\\0 \\0h\\0e\\2n\\0 \\0h\\0e\\0n\\0a\\4\cr
\qquad\tt\\0h\\0e\\0n\\5a\\0t\\0 \\1n\\0a\\0 \\0n\\2a\\0t\\0
  \\1t\\0i\\0o\\0 \\2i\\0o\\0 \\0o\\2n\\0\cr
\enddisplay
occur as special patterns. \TeX\ now computes the maximum interletter
value that occurs at each subword touching each interletter position.
For example, between `|e|' and `|n|' there are four relevant values
in this case (2~from {\tt\\0h\\0e\\2n\\0},
0~from {\tt\\0h\\0e\\0n\\0a\\4},
0~from {\tt\\0h\\0e\\0n\\5a\\0t\\0},
and 1~from {\tt\\1n\\0a\\0}); the maximum of these is~2.
The result of all the maximizations is
\begindisplay
\tt.\\0h\\0y$_3$p\\0h\\0e\\2n\\5a\\4t\\2i\\0o\\2n\\0.
\enddisplay
Now comes the final step: A hyphen is considered to be acceptable
between two letters if the associated interletter value is {\sl odd}.
Thus, two potential breakpoints have been found: `|hy-phen-ation|'.
Similarly, the word `|concatenation|' contains the patterns
\begindisplay
\tt\\0o\\2n\\0
\\0o\\0n\\1c\\0
\\1c\\0a\\0
\\1n\\0a\\0
\\0n\\2a\\0t\\0
\\1t\\0i\\0o\\0
\\2i\\0o\\0
\\0o\\2n\\0
\enddisplay
and this yields `{\tt\\0c\\0o\\2n\\1c\\0a\\0t\\0e\\1n\\2a\\1t\\2i\\0o\\2n\\0}',
i.e., `|con-cate-na-tion|'.

Let's try a 34-letter word:
`|supercalifragilisticexpialidocious|' matches the plain \TeX\ patterns
\begindisplay
\tt u\\1pe r\\1c \\1ca al\\1i ag\\1i gil\\4 il\\1i il\\4ist is\\1ti st\\2i\cr
\qquad\tt s\\1tic
  \\1exp x\\3p pi\\3a \\2i\\1a i\\2al \\2id \\1do \\1ci \\2io \\2us\cr
\enddisplay
(where subscripts that aren't shown are zero), and this yields
$$\centerline{%
\tt.\\0s\\0u\\1p\\0e\\0r\\1c\\0a\\0l\\1i\\0f\\0r\\0a\\0g\\1i\\0l\\4i%
\\0s\\1t\\2i\\0c\\1e\\0x\\3p\\2i\\3a\\0l\\2i\\1d\\0o\\1c\\2i\\0o\\2u\\0s\\0.}$$
The \stretch resulting \stretch hyphens \stretch
`|su-|\stretch|per-|\stretch|cal-|\stretch|ifrag-|\stretch|ilis-|\stretch
|tic-|\stretch|ex-|\stretch|pi-|\stretch|ali-|\stretch|do-|\stretch|cious|'
agree with Random House's {\sl Unabridged Dictionary\/} (which also
shows a few more: `|su-per-cal-i-frag-i-lis-tic-ex-pi-al-i-do-cious|').

Plain \TeX\ loads exactly 4447 patterns into \TeX's memory, beginning with
`{\tt\\0.\\0a\\0c\\0h\\4}' and ending with `{\tt\\4z\\1z\\2}' and
`{\tt\\0z\\4z\\0y$_0$}'. The interletter values in these patterns are all
between 0 and~5; a large odd value like the 5 in `{\tt\\0h\\5e\\0l\\0o\\0}'
forces desirable hyphen points in words like `|bach-e-lor|' and
`|ech-e-lon|', while a large even value like the 4 in `{\tt\\0h\\0a\\0c\\0h\\4}'
suppresses undesirable hyphens in words like `|tooth-aches|'. Liang
derived these patterns by preparing a special version of {\sl Webster's
Pocket Dictionary\/} (Merriam, 1966) that contains about 50,000 words including
derived forms. Then he checked a preliminary set of patterns obtained from
this data against an up-to-date hyphenation dictionary of about
115,000 words obtained from a publisher; errors found in this run
led to the addition of about 1000 words like |camp-fire|,
|Af-ghan-i-stan|, and |bio-rhythm| to the pocket dictionary list. He
weighted a few thousand common words more heavily so that they would be more
likely to be hyphenated; as a result, the patterns of plain \TeX\ guarantee
complete hyphenation of the 700 or so most common words of English,
as well as common technical words like |al-go-rithm|. These patterns
find 89.3\% of the hyphens in Liang's dictionary as a whole, and
they insert no hyphens that are not present.

Patterns derived from the common words of a language tend to work well
on uncommon or newly coined words that are not in the original dictionary.
For example, Liang's patterns find a correct subset of the hyphens in the
word that all of today's unabridged dictionaries agree is the longest
in English, namely
\begintt
pneu-monoul-tra-mi-cro-scop-ic-sil-i-co-vol-canoco-nio-sis.
\endtt
They even do fairly well on words from other languages that aren't
too distant from English; for example, the pseudo-^{German} utterances
of Mark ^{Twain}'s {\sl Connecticut Yankee\/} come out with only six or
seven bad hyphens:
\begintt
Con-stanti-nop-o-li-tanis-cher-
  dudel-sack-spfeifen-mach-ers-ge-sellschafft;
Ni-hilis-ten-dy-na-mitthe-
  aterkaestchensspren-gungsat-ten-taetsver-suchun-gen;
Transvaal-trup-pen-tropen-trans-port-
  tram-pelth-iertreib-er-trau-ungsthrae-nen-tra-goedie;
Mekka-musel-man-nen-massen-menchen-
  mo-er-der-mohren-mut-ter-mar-mor-mon-u-menten-machen.
\endtt
But when plain \TeX\ is tried on the name of a famous ^{Welsh} city,
^^{Llanfair P. G.}
\begintt
Llan-fair-p-wll-gwyn-gyll-gogerych-
  wyrn-drob-wl-l-l-lan-tysil-i-o-gogogoch,
\endtt
linguistic differences became quite evident, since the correct hyphens are
\begintt
Llan-fair-pwll-gwyn-gyll-go-ger-y-
  chwyrn-dro-bwll-llan-ty-sil-i-o-go-go-goch.
\endtt
Appropriate pattern values for other languages can be derived by applying
Liang's method to suitable dictionaries of hyphen points.

^{Dictionaries} of English do not always agree on where syllable boundaries
occur. For example, the {\sl American Heritage Dictionary\/} says
`|in-de-pend-ent|' while {\sl Webster's\/} says `|in-de-pen-dent|'.
Plain \TeX\ generally follows Webster except in a few cases where other
authorities seem preferable.

\begingroup
\hyphenpenalty=-1000 \pretolerance=-1 \tolerance=1000
\doublehyphendemerits=-100000 \finalhyphendemerits=-100000
[From here to the end of this appendix, \TeX\ will be typesetting with
\begindisplay
^|\hyphenpenalty||=-1000   |^|\pretolerance||=-1   |^|\tolerance||=1000|\cr
^|\doublehyphendemerits||=-100000  |^|\finalhyphendemerits||=-100000|\cr
\enddisplay
so that hyphens will be inserted much more often than usual.]

\looseness-1
The fact that plain \TeX\ finds only 90\% of the permissible hyphen points
in a large dictionary is, of course, no cause for alarm. When word
frequency is taken into account, the probability rises to well over 95\%.
Since \TeX's line-breaking algorithm often succeeds in finding a way to
break a paragraph without needing hyphens at all, and since there's a good
chance of finding a different hyphen point near to one that is missed by
\TeX's patterns, it is clear that manual intervention to correct or insert
hyphenations in \TeX\ output is rarely needed, and that such refinements
take a negligible amount of time compared to the normal work of
keyboarding and proofreading.

But you can always insert words into \TeX's ^{exception dictionary}, if
you find that the patterns aren't quite right for your application.
For example, this book was typeset with three exceptional words added:
The format in Appendix~E includes the command
\begintt
\hyphenation{man-u-script man-u-scripts ap-pen-dix}
\endtt
which tells \TeX\ how to hyphenate the words `manuscript', `manuscripts',
and `appendix'. Notice that both singular and plural forms of
`manuscript' were entered, since the exception dictionary affects
hyphenation only when a word agrees completely with an exceptional
entry. \ (Precise rules for the ^|\hyphenation| command are
discussed below.)

If you want to see all of the hyphens that plain \TeX\ will find in some
random text, you can say `^|\showhyphens||{|\<random text>|}|' and the
results will appear on your terminal (and in the log file). For example,
\begintt
*\showhyphens{random manuscript manuscripts appendix}
|smallskip
Underfull \hbox (badness 10000) detected at line 0
[] \tenrm ran-dom manuscript manuscripts ap-pendix
\endtt
shows the hyphen positions that would have been found in this book
without the addition of any ^|\hyphenation| exceptions. Somehow the word
`manuscript' slips through all of the ordinary patterns; the author added it
as an exception for this particular job because he used it 80~times (not
counting its appearances in this appendix).

The |\showhyphens| macro creates an hbox that is intentionally
^{underfull}, so you should ignore the warning about `|badness 10000|';
this spurious message comes out because \TeX\ displays hyphens in compact
form only when it is displaying the contents of anomalous
hboxes. \ (\TeX\ wizards may enjoy studying the way |\showhyphens|
is defined in Appendix~B.)

\danger If you want to add one or more words to the exception dictionary,
just say ^|\hyphenation||{|\<words>|}| where \<words> consists of one
or more \<word> items separated by spaces. A \<word> must consist entirely
of letters and hyphens; more precisely, a ``hyphen'' in this context is
the token |-|\\{12}. A ``letter'' in this context is a character token whose
category code is 11 or~12, or a control sequence defined by ^|\chardef|,
or ^|\char|\<8-bit number>, such that the corresponding character has a nonzero
^|\lccode|. \TeX\ uses the |\lccode| to convert each letter to ``^{lowercase}''
form; a word-to-be-hyphenated will match an entry in the exception
dictionary if and only if both words have the same lowercase form after
conversion to lowercase.

\danger \TeX\ will henceforth insert discretionary hyphens in the specified
positions, whenever it attempts to hyphenate a word that matches an entry
in the exception dictionary, except that hyphens are never inserted after the
very first letter or before the last or second-last letter of a word.
You must insert your own discretionary hyphens if you want to allow them
in such positions. A |\hyphenation| entry might contain no hyphens at all;
then \TeX\ will insert no hyphens in the word.

\danger The exception dictionary is global; i.e., exceptions do not disappear
at the end of a ^{group}. If you specify |\hyphenation| of the same word more
than once, its most recently specified hyphen positions are used.

\ddanger The exception dictionary is dynamic, but the pattern dictionary
is static: To change \TeX's current set of hyphenation patterns, you
must give an entirely new set, and \TeX\ will spend a little time putting them
into a form that makes the hyphenation algorithm efficient. The
command format is ^|\patterns||{|\<patterns>|}|, where \<patterns> is a
sequence of \<pattern> items separated by spaces. This command
is available only in ^|INITEX|, not in production versions of \TeX, since
the process of pattern compression requires extra memory that can be put
to better use in a production system. |INITEX| massages the patterns and
outputs a format file that production versions can load at high speed.

\ddanger A \<pattern> in the |\patterns| list has a more restricted form
than a \<word> in the |\hyphenation| list, since patterns are supposed to
be prepared by experts who are paid well for their expertise. Each
\<pattern> consists of one or more occurrences of \<value>\<letter>,
followed by \<value>. Here \<value> is either a digit (|0|\\{12} to |9|\\{12})
or empty; an empty \<value> stands for zero. For example, the pattern
`{\tt\\0a\\1b\\0}' can be represented as |0a1b0| or |a1b0| or |0a1b| or
simply |a1b|. A \<letter> is a character token of category 11 or~12 whose
|\lccode| is nonzero. If you want to use a ^{digit} as a \<letter>, you
must precede it by a nonempty \<value>; for example, if for some reason you
want the pattern `{\tt\\1a\\01\\2}' you can obtain it by typing
`|1a012|', assuming that |\lccode`1| is nonzero. Exception: The character
`|.|'~is treated as if it were a \<letter> of code~0
when it appears in a pattern. Code~0 (which obviously cannot match a nonzero
|\lccode|) is used by \TeX\ to represent the
left or right edge of a word when it is being hyphenated.

\ddanger Plain \TeX\ inputs a file called |hyphen.tex| that sets up the
pattern dictionary and the initial exception dictionary. The file has the
form
\begindisplay
|\patterns{.ach4 .ad4der .af1t .al3t|\quad$\cdots$\quad|zte4 4z1z2 z4zy}|\cr
|\hyphenation{as-so-ciate as-so-ciates dec-li-na-tion oblig-a-tory|\cr
|  phil-an-thropic present presents project projects reci-procity|\cr
|  re-cog-ni-zance ref-or-ma-tion ret-ri-bu-tion ta-ble}|\cr
\enddisplay
The first thirteen exceptions keep \TeX\ from inserting incorrect
hyphens; for example, `|pro-ject|' and `|pre-sent|' are words like
`|re-cord|', that cannot be hyphenated without knowing the context.
The other exception, `|ta-ble|', is included just to meet the claim
that plain \TeX\ fully hyphenates the 700 or so most common words
of English.

\looseness-1
\ddanger But how does \TeX\ decide what sequences of letters are ``words''
that should be hyphenated? Let's recall that \TeX\ is working on a
horizontal list that contains boxes, glue, rules, ^{ligatures}, ^{kerns},
discretionaries, marks, whatsits, etc., in addition to simple characters;
somehow it has to pick out things to hyphenate when it is unable to
find suitable breakpoints without hyphenation. The presence of
punctuation marks before and/or after a word should not make a word
unrecognizable or unhyphenatable; neither should the presence of
ligatures and kerns within a word. On the other hand, it is desirable to
do hyphenation quickly, not spending too much time trying to handle
unusual situations that might be hyphenatable but hard to recognize
\hbox{mechanically}.

\ddanger \TeX\ looks for potentially hyphenatable words by searching
ahead from each glue item that is not in a math formula. The search
bypasses characters whose |\lccode| is zero, or ligatures that begin
with such characters; it also bypasses whatsits and {\sl^{implicit kern}\/}
items, i.e., kerns that were inserted by \TeX\ itself because of information
stored with the font. If the search
finds a character with nonzero |\lccode|, or if it finds a ligature
that begins with such a character, that character is called the
{\sl starting letter}. But if any other type of item occurs before a suitable
starting letter is found, hyphenation is abandoned (until after the next
glue item). Thus, a box or rule or mark, or a kern that was explicitly
inserted by ^|\kern| or~|\/|, must not intervene between glue and a
hyphenatable word. If the starting letter is not lowercase (i.e.,
if it doesn't equal its own |\lccode|), hyphenation is abandoned unless
^|\uchyph| is positive.

\ddanger If a suitable starting letter is found, let it be in font~$f$.
Hyphenation is abandoned unless the ^|\hyphenchar| of~$f$ is between
0 and~255, and unless a character of that number exists in the font.
If this test is passed, \TeX\ continues to scan forward
until coming to something that's not one of the following three
``admissible items'': (1)~a character in font~$f$ whose |\lccode|
is nonzero; (2)~a ligature formed entirely from characters of type~(1);
(3)~an implicit kern. The first inadmissible item terminates this part of
the process; the trial word consists of all the letters found in admissible
items. Notice that all of these letters are in font~$f$.

\ddanger If a trial word $l_1\ldots l_n$ has been found by this process,
hyphenation will still be abandoned unless $n\ge\lambda+\rho$, where
$\lambda=\max(1,\hbox{|\lefthyphenmin|})$ and
$\rho=\max(1,\hbox{|\righthyphenmin|})$.
^^|\lefthyphenmin| ^^|\righthyphenmin|
\ (Plain \TeX\ takes $\lambda=2$ and $\rho=3$.) \ Furthermore, the items
immediately following the trial word must consist of zero or more
characters, ligatures, and implicit kerns, followed immediately by
either glue or an explicit kern or a penalty item or a whatsit or an
item of vertical mode material from ^|\mark|, ^|\insert|, or ^|\vadjust|.
Thus, a box or rule or math formula or discretionary following too closely
upon the trial word will inhibit hyphenation. (Since \TeX\ inserts
empty discretionaries after ^{explicit hyphens}, these rules imply that
already-hyphenated compound words will not be further hyphenated by
the algorithm.)

\ddanger Trial words $l_1\ldots l_n$ that pass all these tests are submitted
to the hyphenation algorithm described earlier. Hyphens are not inserted
before $l_\lambda$ or after $l_{n+1-\rho}$.
If other hyphenation points are found,
one or more discretionary items are inserted in the word; ligatures
and implicit kerns are reconstituted at the same time.

\ddanger Since ligatures and kerns are treated in quite a general manner,
it's possible that one hyphenation point might preclude another because the
ligatures that occur with hyphenation might overlap the ligatures
that occur without hyphenation. This anomaly probably won't occur in real-life
situations; therefore \TeX's interesting approach to the problem will
not be discussed here.

\ddanger According to the rules above, there's an important distinction
between implicit and explicit kerns, because \TeX\ recomputes implicit
kerns when it finds at least one hyphen point in a word. You can
see the difference between these two types of kerns when \TeX\ displays
lists of items in its ^{internal format}, if you look closely:
`|\kern2.0|' denotes an implicit kern of $2\pt$, and `|\kern 2.0|'
denotes an explicit kern of the same magnitude. The ^{italic correction}
command~|\/| inserts an explicit kern.

\ddanger The control sequence ^|\-| is equivalent to
|\discretionary{\char|$\,h$|}{}{}|, where $h$ is the
^|\hyphenchar| of the current font, provided that $h$ lies
between 0 and~255. Otherwise |\-| is equivalent to
|\discretionary{}{}{}|.

\ddanger \looseness=-1
So far we have assumed that \TeX\ knows only one style of
hyphenation at a time; but in fact \TeX\ can remember up to 256
distinct sets of rules, if you have enough memory in your computer. An
integer parameter called ^|\language| selects the rules actually used;
every ^|\hyphenation| and ^|\patterns| specification appends new rules
to those previously given for the current value of\/ |\language|.
\ (If\/ |\language| is negative or greater than 255, \TeX\ acts as if
|\language|$\null=0$.) \ All |\patterns| for all languages must be
given before a paragraph is typeset, if |INITEX| is used for
typesetting.

\ddanger \TeX\ is able to work with several languages in the same
paragraph, because it operates as follows. At the beginning of a
paragraph the ``current language'' is defined to be~0. Whenever a
character is added to the current paragraph (i.e., in unrestricted
horizontal mode), the current language is compared to |\language|; if
they differ, the current language is reset and a whatsit node
specifying the new current language is inserted before the character.
Thus, if you say `|\def\french{\language1...}|' and `|mix| |{\french
franc/ais}| |with| |English|', \TeX\ will put whatsits before the |f|
and the~|w|; hence it will use language~1 rules when hyphenating
|franc/ais|, after which it will revert to language~0. You can insert
the whatsit yourself (even in restricted horizontal mode) by saying
^|\setlanguage|\<number>; this changes the current language but it
does not change |\language|. Each whatsit records the current
|\lefthyphenmin| and |\righthyphenmin|.
^^{French}

\endgroup
\endchapter

If all problems of hyphenation have not been solved,
at least some progress has been made since that night,
when according to legend, an RCA Marketing Manager received
a phone call from a disturbed customer. His 301 had just hyphenated ``God.''
\author PAUL E. ^{JUSTUS}, {\sl There's More to Typesetting Than %
  Setting Type\/} (1972)
% in {\sl IEEE Transactions on Professional Commun. vol PC-15, pp. 13-15

\bigskip

The committee skeptically re-
commended more study for a bill
to require warning labels on rec-
ords with subliminal messages re-
corded backward.
\author THE PENINSULA ^{TIMES TRIBUNE} (April 28, 1982)

%\eject
\eject\byebye
