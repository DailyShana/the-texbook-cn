% -*- coding: utf-8 -*-

\input macros

%\beginchapter Chapter 6. Running\\\TeX
\beginchapter Chapter 6. 运行程序

\origpageno=23

%The best way to learn how to use \TeX\ is to use it. Thus, it's high time
%for you to sit down at a computer terminal and interact with the \TeX\
%system, trying things out to see what happens. Here are some small but
%complete examples suggested for your first encounter.
%^^{Running the program}
\1掌握 \TeX\ 的最好方法就是使用它。%
因此,对你而言,该坐在终端前与 \TeX\ 系统交流交流,试一下看看到底怎么样。%
首先为你准备的是一些短小而全面的例子。

%Caution: This chapter is rather a long one. Why don't you stop reading
%now, and come back fresh tomorrow?
注意:本章相当长。要不要现在休息一下,明天接着看?

%\smallskip OK, let's suppose that you're rested and excited about having a
%trial run of \TeX\null. Step-by-step instructions for using it appear in this
%chapter. First do this: Go to the lab where the graphic output device is,
%since you will be wanting to see the output that you get---it won't really
%be satisfactory to run \TeX\ from a remote location, where you can't hold
%the generated documents in your own hands. Then log in; and start \TeX. \
%(You may have to ask somebody how to do this on your local computer. Usually
%the operating system prompts you for a command and you type `|tex|' or
%`|run| |tex|' or something like that.)
\smallskip
现在,假定你已经精力充沛,并且急于进行 \TeX\ 实战。%
逐步使用它的指导都在本章中。%
首先要做的是:
去有图像输出设备的实验室,因为你将要看到所得到的结果——%
从远程运行 \TeX\ 不那么过瘾,因为你不能亲手拿到生成的文档。%
于是,登录;运行 \TeX。%
(可能你必须问问别人看看怎样做。%
通常操作系统提示符出现后,键入`|TeX|'或`|run| |tex|'或类似的命令。)

%When you're successful, \TeX\ will welcome you with a message such as
%\begintt
%This is TeX, Version 3.141 (preloaded format=plain 89.7.15)
%**
%\endtt
%The `^|**|' is \TeX's way of asking you for an input file name.
%% Incidentally, 89.7.15 was Jill's 50th birthday.
当你成功运行后,\TeX\ 将出现如下欢迎信息:
\begintt
This is TeX, Version 3.14 (preloaded format=plain 89.7.15)
**
\endtt
`|**|'是 \TeX\ 要求你输入文件名。

%Now type `^|\relax|' (including the backslash), and ^\<return>
%(or whatever is used to mean ``end-of-line'' on your terminal).
%\TeX\ is all geared up for action, ready to read a long manuscript; but
%you're saying that it's all right to take things easy, since this is
%going to be a real simple run. In fact, |\relax| is a control sequence
%that means ``do nothing.''
现在,键入`|\relax|'(要包含反斜线)和 \<return>~(或者其它在你的终端上表示%
一行结束的符\hbox{号)。}%
\TeX\ 已经做好准备以读入一篇长文稿;
但是你刚才的意思是简单做点事,因为仅仅运行了一下。%
实际上,|relax| 是一个控制系列,意思是``do nothing''。

%The machine will type another asterisk at you. This time type something
%like `|Hello?|'\ and wait for another ^{asterisk}. Finally type `^|\end|',
%and stand back to see what happens.
计算机将输出一个星号来回应你。%
现在键入象`|Hello?|'这样的内容并且等待下一个星号。
最好键入`|\end|'且等一下看看有什么情况。

%\TeX\ should respond with `^|[1]|' (meaning that it has finished page~1
%of your output); then the program will halt, probably with some
%indication that it has created a file called `|texput.dvi|'. \ (\TeX\
%uses the name ^|texput| for its output when you haven't specified any
%better name in your first line of input; and ^|dvi| stands for
%``^{device independent},'' since |texput.dvi| is capable of
%being printed on almost any kind of typographic output device.)
\TeX\ 将回应出`|[1]|'(意思是它完成了一页输出);
接着程序将停止,可能还有一些显示说它生成了一个叫`|texput.dvi|'的文件。%
(如果你在输入的第一行没有给定名字,\TeX\ 将使用 |texput| 这个名字;
还有,|dvi| 的意思是``不依赖于设备'', 因为 |texput.dvi| 可以在几乎%
所有的印刷输出设备上打印出来。)

%Now you're going to need some help again from your friendly local
%computer hackers. They will tell you how to produce hardcopy from
%|texput.dvi|. And when you see the hardcopy---Oh, glorious day!---you
%will see a magnificent `Hello?'\ and the page number `1' at the bottom.
%Congratulations on your first masterpiece of fine printing.
现在你又需要周围友好的计算机高手的帮助了。
他们会告诉你怎样把 |texput.dvi| 打印出来。
当你看到结果——噢,太漂亮了!——你会看到一个漂亮的 `Hello?',
并且在底部中间有页码`1'。为你的精美印刷的处女作干杯。

%\smallbreak
%The point is, you understand now how to get something through the whole cycle.
%It only remains to do the same thing with a somewhat longer document.
%So our next experiment will be to work from a file instead of typing
%the input online.
\smallbreak
关键是你现在知道了实现它的整个过程。%
剩下的就是用同样的方法处理更长的文稿。%
所以我们接下来用文件举例,而不是逐行输入。

%Use your favorite text editor to create a file called ^|story.tex|
%that contains the following 18 lines of text (no more, no less):
%$$\halign{\hbox to\parindent{\hfil\sevenrm#\ \ }&#\hfil\cr
%1&|\hrule|\cr\noalign{^^|\hrule|}
%2&|\vskip 1in|\cr\noalign{^^|\vskip|^^{leading, see vskip}}
%3&|\centerline{\bf A SHORT STORY}|\cr\noalign{^^|\centerline|}
%4&|\vskip 6pt|\cr
%5&|\centerline{\sl by A. U. Thor}|\cr\noalign{^^{Thor}}
%6&|\vskip .5cm|\cr
%7&|Once upon a time, in a distant|\cr
%8&|  galaxy called \"O\"o\c c,|\cr\noalign{^^|\"|^^|\c|}
%9&|there lived a computer|\cr
%10&|named R.~J. Drofnats.|\cr\noalign{^^{Drofnats}}
%11&||\cr
%12&|Mr.~Drofnats---or ``R. J.,'' as|\cr
%13&|he preferred to be called---|\cr
%14&|was happiest when he was at work|\cr
%15&|typesetting beautiful documents.|\cr
%16&|\vskip 1in|\cr
%17&|\hrule|\cr
%18&|\vfill\eject|\cr\noalign{^^|\vfill|^^|\eject|}}$$
%\write16{\ifnum\pageno=\storypage
%  \else Redefine \string\storypage to be \the\pageno\fi}%
%(Don't type the numbers at the left of these lines, of course; they are present
%only for reference.) \ This example is a bit long, and more than a bit silly;
%but it's no trick for a good typist like you and it will give you some
%worthwhile experience, so do it. For your own good. And think about what
%you're typing, as you go; the example introduces a few important features
%of \TeX\ that you can learn as you're making the file.
\1用你喜欢的文本编辑器建立一个文件,名字为 |story.tex|,
它包含下列 18 行文本(不多不少):
$$\halign{\hbox to\parindent{\hfil\sevenrm#\ \ }&#\hfil\cr
1&|\hrule|\cr\noalign{}
2&|\vskip 1in|\cr\noalign{}
3&|\centerline{\bf A SHORT STORY}|\cr\noalign{}
4&|\vskip 6pt|\cr
5&|\centerline{\sl by A. U. Thor}|\cr\noalign{}
6&|\vskip .5cm|\cr
7&|Once upon a time, in a distant|\cr
8&|  galaxy called \"O\"o\c c,|\cr\noalign{}
9&|there lived a computer|\cr
10&|named R.~J. Drofnats.|\cr\noalign{}
11&||\cr
12&|Mr.~Drofnats---or ``R. J.,'' as|\cr
13&|he preferred to be called---|\cr
14&|was happiest when he was at work|\cr
15&|typesetting beautiful documents.|\cr
16&|\vskip 1in|\cr
17&|\hrule|\cr
18&|\vfill\eject|\cr\noalign{}}$$
\write16{\ifnum\pageno=\storypage
  \else Redefine \string\storypage to be \the\pageno\fi}%
(当然,别键入这些行左边的数字;
它们只起参照的作用。)
这个例子有点长,而且还有点无聊;
对象你这样优秀的排版者没有窍门,并且它将为你提供宝贵的经验,所以要完成它。%
为了自己!
键入时想一想;随着你键入这个文件,这个例子出现了你可能已经掌握的几个 \TeX\ 的%
重要特征。

%Here is a brief explanation of what you have just typed: Lines 1 and~17
%put a horizontal ^{rule} (a thin line) across the page. Lines 2 and~16
%skip past one inch of space; `|\vskip|' means ``vertical skip,'' and this
%extra space will separate the horizontal rules from the rest of the copy.
%Lines 3 and~5 produce the title and the author name, centered, in boldface
%and in slanted type. Lines 4 and~6 put extra white space between those
%lines and their successors. \ (We shall discuss units of measure like
%`|6pt|' and `|.5cm|' in Chapter~10.)
这里对你刚刚键入的内容做一个简要的解释:
第 1 和 17 行放了一根贯穿整页的水平线(细\hbox{线)。}%
第 2 和 16 行跳过了一英寸的空白;`|\vskip|'表示``垂直跳跃'',
并且这个额外的空白将把水平线与文档的其它内容分开。%
第 3 和 5 行生成了题目和作者的名字,居中且用 bold 字体和用 slanted 字体。%
第 4 和 6 行在相应行和随后的行之间放上空白。%
(我们将在第十章讨论象`|6pt|'和`|.5cm|'这样的测量单位。

%The main bulk of the story appears on lines 7--15, and it consists of
%two ^{paragraphs}. The fact that line~11 is blank informs \TeX\ that
%^^{blank line} ^^{empty line}
%line~10 is the end of the first paragraph; and the `|\vskip|' on line~16
%implies that the second paragraph ends on line~15, because vertical
%skips don't appear in paragraphs. Incidentally, this example seems
%to be quite full of \TeX\ commands; but it is atypical in that respect,
%because it is so short and because it is supposed to be teaching things.
%Messy constructions like |\vskip| and |\centerline| can be expected at the
%very beginning of a manuscript, unless you're using a canned format, but
%they don't last long; most of the time you will find yourself typing
%straight text, with relatively few control sequences.
本 story 的主体在第 7--15 行,并且它包含两个段落。%
第 11 行是一个空行就告诉 \TeX, 第 10 行是第一个段落的最后一行;
还有,第 16 行的`|\vskip|'意味着第二段到第 15 行结束,
因为垂直跳跃不出现在段落中。%
顺便说一下,本例子看起来含有丰富的 \TeX\ 命令;
但是在那方面它不是典型的,因为它太短以及被用来做教学。%
象 |\vskip| 和|\centerline|这样的结构可能应该放在文稿的开头,除非你在用一个%
刻板的格式,但是它们不会太多;
在大部分的时间你都在键入规整的文本以及相对少量的控制系列。

%And now comes the good news, if you haven't used computer typesetting
%before:  You don't have to worry about where to break lines in a paragraph
%(i.e., where to stop at the right margin and to begin a new line), because
%\TeX\ will do that for you. Your manuscript file can contain long lines or
%short lines, or both; it doesn't matter. This is especially helpful when
%you make changes, since you don't have to retype anything except the words
%that changed. {\sl Every time you begin a new line in your manuscript file
%it is essentially the same as typing a space.} When \TeX\ has read an
%entire paragraph---in this case lines 7 to~11---it will try to break up
%the text so that each line of output, except the last, contains about the
%same amount of copy; and it will hyphenate words if necessary to keep the
%spacing consistent, but only as a last resort.
\1现在,如果你以前不曾用过计算机排版,那么你有福了:
在一个段落中,你不必关心在什么地方断行(即在右边的哪里结束本行并重新开始一行),
因为 \TeX\ 将替你做。%
你的文稿文件可以包含长行和/或短行;这不是问题。%
当你修改时这特别有用,因为除了单词外你不必改动任何东\hbox{西。}%
{\KT{10}在你的文稿中,每个新行本质上相相当于一个空格}。
当 \TeX\ 读完整段时——在本例中就是第 7 到 11 行——它将设法把文本断开,
使得除最后一行外的每行输出包含同样多的材料;
并且为了保持间距的一致性,在有必要的地方加入连字符,但这只是最后一招。

%Line 8 contains the strange concoction
%\begintt
%\"O\"o\c c
%\endtt
%and you already know that |\"| stands for an ^{umlaut} accent. The
%|\c| stands for a ``^{cedilla},'' so you will get `\"O\"o\c c' as the
%name of that distant galaxy.
第 8 行含有怪怪的东西
\begintt
\"O\"o\c c
\endtt
并且你已经知道,|\"| 表示一个变音的重音。%
|\c|表示一个``变音符号'', 所以你得到了`\"O\"o\c c'来作为那个遥远银河的名字。

%The remaining text is simply a review of the conventions that we discussed
%long ago for dashes and quotation marks, except that the `|~|' signs in
%lines 10 and~12 are a new wrinkle. These are called {\sl ^{ties}}, because
%they tie words together; i.e., \TeX\ is supposed to treat `|~|' as a
%normal space but not to break between lines there.
%A good typist will use ties within names, as shown in our
%example; further discussion of ties appears in Chapter~14. ^^{tilde}
剩下的文本就是以前我们讨论过的破折号和引号的用法,只有在第 10 和 12 行的%
`|~|'符号是个的波纹号。%
它们称为{\KT{10}带子}, 因为它们的单词绑在一起;
即 \TeX\ 把`|~|'看作正常的空格,但是不能在那里断行。%
好的排版应该在名字之间用带子,就象我们的例子一样;
关于带子更多的讨论见第十四章。

%Finally, line~18 tells \TeX\ to `^|\vfill|', i.e., to fill the rest of
%the page with white space; and to `^|\eject|' the page, i.e., to send it
%to the output file.
最后,第 18 行告诉 \TeX\ 要`|\vfill|', 即,用空白把本页其余的部分填满;
接着`|\eject|'本页\hbox{面,} 即把它送到输出文件。

%\smallskip Now you're ready for Experiment~2: Get \TeX\ going again.
%This time when the machine says `|**|' you should answer `|story|', since
%that is the name of the file where your input resides. \ (The file
%could also be called by its full name `|story.tex|', but \TeX\ automatically
%supplies the suffix `|.tex|' if no suffix has been specified.)
%^^{file names}
\smallskip 现在进行第二次实战:
再次运行 \TeX。%
这次当计算机出现`|**|'时,你应该键入`|stroy|',
因为那是你的输入资料所在的文件的名字。%
(文件也可以用它的全名`|story.tex|', 但是如果后缀没有给出,
\TeX\ 将自动加上后缀`|.tex|'。)

%You might wonder why the first prompt was `^|**|', while the subsequent
%ones are `^|*|'; the reason is simply that the first thing you type to
%\TeX\ is slightly different from the rest: If the first character of your
%response to `|**|' is not a backslash, \TeX\ automatically inserts
%`^|\input|'. Thus you can usually run \TeX\ by merely naming your input
%file. \ (Previous \TeX\ systems required you to start by typing `|\input
%story|' instead of `|story|', and you can still do that; but most \TeX\
%users prefer to put all of their commands into a file instead of typing
%them online, so \TeX\ now spares them the nuisance of starting out with
%|\input| each time.) \ Recall that in Experiment~1 you typed `|\relax|';
%that started with a backslash, so |\input| was not implied.
你可能想知道为什么第一个提示符是`|**|', 而后面的是`|*|';
原因很简单,你所键入到 \TeX\ 的第一个东西与其余的略有不同:
如果在`|**|'后你键入的第一个字符不是反斜线,那么 \TeX\ 将自动插入`|\input|'。%
因此,通常只要命名了输入文件,你就可以运行 \TeX\ 了。%
(目前的 \TeX\ 系统要求键入`|\input story|'而不是`|story|'来开始处理,
并且你仍然可以那样做;
但是大多数 \TeX\ 用户喜欢把它们所有的命令都放在一个文件中,而不是逐行键入,
所有现在 \TeX\ 不需要每次开始都键入 |\input| 来惹人厌烦。)
回想一下,在第一次实战中,你键入了`|\relax|', 它以反斜线开头,
所以 |\input| 没有暗中插入。

%\danger There's actually another difference between `|**|' and `|*|': If the
%first character after |**| is an ^{ampersand} (\thinspace`|&|'\thinspace),
%\TeX\ will replace its memory with a precomputed ^{format file} before
%proceeding. Thus, for example, you can type `|&plain \input story|' or
%even `|&plain story|' in response to `|**|', if you are running some
%version of \TeX\ that might not have the plain format preloaded.
%^^{preloaded formats}
\danger 实际上在`|**|'和`|*|'之间有另外一个差别:
如果 |**| 后的第一个字符是 ampersand~(\thinspace`|&|'\thinspace),
\TeX\ 将在处理之前用预编译的格式文件来替换其内存。%
于是,比如,你可以在`|**|'后键入`|&plain \input story|'或仅仅是`|&plain story|',
如果你正在运行没有预载入的 plain 格式的某些 \TeX\ 版本。

%\danger Incidentally, many systems allow you to invoke \TeX\ by typing a
%one-liner like `|tex story|' instead of waiting for the `|**|'; similarly,
%`|tex \relax|' works for Experiment~1, and `|tex &plain story|' loads the
%plain format before inputting the |story| file.  You might want to try
%this, to see if it works on your computer, or you might ask somebody if
%there's a similar shortcut.
\danger \1顺便说一下,很多系统都允许你通过在命令行键入象`|tex story|'这样的命令%
来调用 \TeX, 而不用等`|**|';
类似地,`|tex \relax|'就是实战一,`|tex &plain story|'在输入 |story| 文件前%
载入了 plain 格式。
你可能想试试,看看它在你的计算机上是否可行,或者你可能会问别人是否有一个%
类似的快捷方法。

%As \TeX\ begins to read your story file, it types `|(story.tex|', possibly
%with a version number for more precise identification, depending on your
%local operating system. Then it types `|[1]|', meaning that page~1 is done;
%and `|)|', meaning that the file has been entirely input.
随着 \TeX\ 开始读入你的 story 文件,它在终端输出了`|(story.tex|',
可能会有精确识别的版本号,这依赖于你的操作系统。%
接着,它输出了`|[1]|', 意思是第 1 页处理完毕;
接着是`|)|', 意味着文件被全部读入了。

%\TeX\ will now prompt you with `|*|', because the file did not contain
%`^|\end|'. Enter |\end| into the computer now, and you should get a file
%|story.dvi| containing a typeset version of Thor's story. As in Experiment~1,
%you can proceed to convert |story.dvi| into hardcopy; go ahead and do that now.
%The typeset output won't be shown here, but you can see the results by
%doing the experiment personally. Please do so before reading on.
\TeX\ 现在又用`|*|'提示你了,因为文件中不包含`|\end|'。%
现在输入 |\end|, 就得到一个文件 |story.dvi|, 它包含 Thor 的 story 排版后的结果。%
就象在实战一中一样,你可以把 |story.dvi| 打印出来;现在去做吧。%
这里没有给出排版输出,但是你可以通过亲自实战得到结果。%
请在继续之前完成它。

%\exercise Statistics show that only 7.43 of 10 people who read this manual
%actually type the |story.tex| file as recommended, but that those people
%learn \TeX\ best. So why don't you join them?
%\answer Laziness and/or obstinacy.
\exercise 统计表明,在读本手册的人中,10 个中有 7.43 个实际上按照建议键入了%
~|story.tex| 文件,但是那些人掌握 \TeX\ 却最好。%
为何你不想变成他们那样?
\answer \1懒惰和/或固执。

%\exercise Look closely at the output of Experiment~2, and compare it to
%|story.tex|\thinspace: If you followed the instructions carefully, you
%will notice a typographical error. What is it, and why did it sneak in?
%\answer There's an unwanted space after `called---', because (as the book
%says) \TeX\ treats the end of a line as if it were a blank space. That
%blank space is usually what you want, except when a line ends with a
%hyphen or a dash; so you should {\sc WATCH OUT} for lines that end with
%hyphens or dashes.
\exercise 仔细观察实战二的输出,把它与 |story.tex| 相比较:
如果你认真阅读了用法说明,你会发现一个排版错误。%
它是什么?为什么它会偷偷出现在那里?
\answer 在 `called---' 之后有个多余空格,
这是因为(如同本书所说)\TeX\ 将行尾符视为空格符。
这样的空白通常是你所需的,除非某行以连字符或横线符结尾;
因此你应当{\bf 留意}以连字符或横线符结尾的行。

%With Experiment 2 under your belt, you know how to make a document from a
%file. The remaining experiments in this chapter are intended to help you
%cope with the inevitable anomalies that you will run into later; we will
%intentionally do things that will cause \TeX\ to ``squeak.''
现在,你知道怎样从一个文件得到文档了。%
本章剩下的实战要帮助你应付后面运行中不可避免的异常情况;
我们将有意用一些方法使 \TeX\ ``难受得只叫''。

%But before going on, it's best to fix the error revealed by the previous
%output (see exercise 6.2): Line~13 of the |story.tex| file should be changed to
%\begintt
%he preferred to be called---% error has been fixed!
%\endtt
%The `|%|' sign here ^^{percent} is a feature of plain \TeX\ that we haven't
%discussed before: It effectively terminates a line of your input file,
%without introducing the blank space that \TeX\ ordinarily inserts when
%moving to the next line of input. Furthermore, \TeX\ ignores everything
%that you type following a |%|, up to the end of that line in the file;
%you can therefore put ^{comments} into your manuscript, knowing that the
%comments are for your eyes only.
但是在继续之前先把前面输出的错误纠正过来(见 exercise 6.2):
文件 story.tex 的第 13 行应当改为
\begintt
he preferred to be called---% error has been fixed!
\endtt
这里的`|%|'%
号是 plain \TeX\ 中我们以前没讨论过的特征:
它使得输入文件的行有效地终止而不引入换行时 \TeX\ 通常要插入的空格。%
还有,\TeX\ 将忽略掉 |%| 后的任何内容,直到文件的那行的结尾,
这样就可以在文稿中加入注释,这些注释只是为了阅读方便。

%Experiment 3 will be to make \TeX\ work harder, by asking it to set the
%story in narrower and narrower columns. Here's how: After starting the
%program, type
%\begintt
%\hsize=4in \input story
%\endtt
%in response to the `|**|'. This means, ``Set the story in a 4-inch column.''
%More precisely, ^|\hsize| is a primitive of \TeX\ that specifies the
%horizontal size, i.e., the width of each normal line in the output when a
%paragraph is being typeset; and ^|\input| is a primitive that causes \TeX\
%to read the specified file. Thus, you are instructing the machine to
%change the normal setting of\/ |\hsize| that was defined by plain \TeX, and
%then to process |story.tex| under this modification.
实战三将让 \TeX\ 处理更吃力,这是通过要求它把 story 放在越来越窄的栏中。%
如下操作:
启动程序后,在`|**|'后键入
\begintt
\hsize=4in \input story
\endtt
\1这意味着``把 story 放在 4 英寸的栏中''。%
更准确地说,|\hsize| 是 \TeX\ 的一个原始控制系列,它给出了水平宽度,
即当排版段落时输出中每个正常行的宽度;
还有,|\input| 也是一个原始控制系列,它使 \TeX\ 读入给定的文件。%
于是,你告诉计算机改变由 plain \TeX\ 定义的 |\hsize| 的正常设置,
并且在这种修改后的设置下处理 |story.tex|。

%\TeX\ should respond by typing something like `|(story.tex [1])|' as
%before, followed by `|*|'. Now you should type
%\begintt
%\hsize=3in \input story
%\endtt
%and, after \TeX\ says `|(story.tex [2])|' asking for more, type three more lines
%\begintt
%\hsize=2.5in \input story
%\hsize=2in \input story
%\end
%\endtt
%to complete this four-page experiment.
如前 \TeX\ 输出类似`|(story.tex [1])|'后,接着出现`|*|'。%
现在你应该键入
\begintt
\hsize=3in \input story
\endtt
接着,\TeX\ 输出`|(story.tex [2])|'后又出现`|*|', 键入下面三行
\begintt
\hsize=2.5in \input story
\hsize=2in \input story
\end
\endtt
这就完成了这个四页的实战。

%Don't be alarmed when \TeX\ screams `|Overfull| |\hbox|' several times as
%it works at the 2-inch size; that's what was supposed to go wrong during
%Experiment~3. There simply is no good way to break the given
%paragraphs into lines that are exactly two inches wide, without making
%the spaces between words come out too large or too small. Plain \TeX\
%has been set up to ensure rather strict tolerances on all of the lines it
%produces:
%\begindisplay
%\hbox spread-1em{You don't get spaces between words narrower than this,\ and}\cr
%\hbox spread+1.679895em{you don't get spaces between words wider than this.}\cr
%\enddisplay
%If there's no way to meet these restrictions, you get an ^{overfull box}.
%And with the overfull box you also get (1)~a warning message, printed
%on your terminal, and (2)~a big black bar inserted at the right of the
%offending box, in your output. \ (Look at page~4 of the output from
%Experiment~3; the overfull boxes should stick out like sore thumbs.
%On the other hand, pages 1--3 should be perfect.)
在处理 2-inch 宽度时,\TeX\ 会发出几次`|Overfull| |\hbox|'警告,别慌;
这就是在实战三中设定的问题。%
在把给定的段落分解为正好 2 英寸宽度时,如果不允许单词间的间距太大或太小,
是没有简单的好办法的。%
Plain \TeX\ 已经对所有行设置了一个相当严格的公差,由它得到
\begindisplay
%\hbox spread-1em{你所得到的字间距不能比它 窄,}\cr
%\hbox spread+1.679895em{也不能比它 宽。}\cr
\hbox spread-1em{you don't get spaces between words narrower than this,\ and}\cr
\hbox spread+1.679895em{you don't get spaces between words wider than this.}\cr
\enddisplay
如果没有办法满足这些限制,就得到了一个溢出的盒子。%
对于这个溢出的盒子,还出现 (1) 警告信息,出现在终端上,和 (2) 在溢出盒子的右边%
插入一个大黑块,出现在输出结果中。%
(看看实战三输出的第四页;溢出的盒子象大拇指一样伸出来。
另一方面,第一到三页就很漂亮。)

%Of course you don't want overfull boxes in your output, so \TeX\ provides
%several ways to remove them; that will be the subject of our Experiment~4.
%But first let's look more closely at the results of Experiment~3, since
%\TeX\ reported some potentially valuable information when it was forced
%to make those boxes too full; you should learn how to read this data:
%\begintt
%Overfull \hbox (0.98807pt too wide) in paragraph at lines 7--11
%\tenrm tant galaxy called []O^^?o^^Xc, there lived||
%Overfull \hbox (0.4325pt too wide) in paragraph at lines 7--11
%\tenrm a com-puter named R. J. Drof-nats. ||
%Overfull \hbox (5.32132pt too wide) in paragraph at lines 12--16
%\tenrm he pre-ferred to be called---was hap-||
%\endtt
%Each overfull box is correlated with its location in your input file
%(e.g., the first two were generated when processing the paragraph on
%lines 7--11 of |story.tex|), and you also learn by how much the copy
%sticks out (e.g., 0.98807 points).
当然,你不想在输出中有溢出的盒子,所有 \TeX\ 给出几种方法来去掉它们;
这就是我们实战四的主题。%
但是首先要仔细观察实战三的结果,因为当 \TeX\ 被迫使用溢出的盒子时,
给出了一些有价值的信息;你应该知道怎样分析这些数据:
\begintt
Overfull \hbox (0.98807pt too wide) in paragraph at lines 7--11
\tenrm tant galaxy called []O^^?o^^Xc, there lived||
Overfull \hbox (0.4325pt too wide) in paragraph at lines 7--11
\tenrm a com-puter named R. J. Drof-nats. ||
Overfull \hbox (5.32132pt too wide) in paragraph at lines 12--16
\tenrm he pre-ferred to be called---was hap-||
\endtt
每个溢出的盒子都与输入文件中的相应位置有关(比如,当处理 |story.tex| 的第 7--11~%
行时出现前两个警告), 并且还可以知道伸出了多少宽度(比如,0.98807 points)。

%Notice that \TeX\ also shows the contents of the overfull boxes in
%abbreviated form. For example, the last one has the words `he preferred
%to be called---was hap-', set in font |\tenrm| (10-point roman type);
%the first one has a somewhat curious rendering of `\"O\"o\c c', because the
%accents appear in strange places within that font. In general, when you
%see `^|[]|' in one of these messages, it stands either for
%the paragraph indentation or for some sort of complex construction;
%in this particular case it stands for an umlaut that has been raised
%up to cover an `O'.
\1注意,\TeX\ 还用简略的形式给出了溢出的盒子的内容。%
例如,最后一个警告有这样的信息:`he preferred
to be called---was hap-', 设定字体为 |\tenrm| (10-point roman 字体);
第一个所给出的内容是有些古怪的`\"O\"o\c c'的表述,
因为重音出现在字体的奇怪的地方。%
一般地,当你在这些信息中发现`|[]|'时,就表示段落缩进或某些复杂的指令;
在这种特殊情况下,它表示一个`O'上的重音符号。

%\dangerexercise Can you explain the `\|' that appears after
%`|lived|' in that message?
%\answer It represents the heavy bar that shows up in
%your output. \ (This bar wouldn't be present if\/ ^|\overfullrule| had been
%set to |0pt|, nor is it present in an underfull box.)
\dangerexercise 请解释一下出现在那个信息中`|lived|'后面的那个`\|'。
\answer 它表示出现在你的输出中的宽竖线。%
(此竖线在设定 ^|\overfullrule| 为 |0pt| 时不会出现,在未满盒子中也不会出现。)

%\ddangerexercise Why is there a space before the `\|' in `|Drof-nats.
%|\|'\thinspace?
%\answer This is the ^|\parfillskip| space that ends the paragraph.
%In plain \TeX\ the parfillskip is zero when the last line of the paragraph
%is full; hence no space actually appears before the rule in the output
%of Experiment~3. But all hskips show up as spaces in an overfull box
%message, even if they're zero.
\ddangerexercise 为什么在`|Drof-nats. |\|'中的`\|'前有一个空格?
\answer 这是结束段落的 ^|\parfillskip| 间距。
在 plain \TeX\ 中,当段落最后一行已填满时 parfillskip 等于零;
从而在实战三的输出的标尺之前实际上没有空白。
但是在过满盒子的信息中,所有水平间距都显示为空格,即使它们等于零。

%You don't have to take out pencil and paper in order to write down the
%overfull box messages that you get before they disappear from view, since
%\TeX\ always writes a ``^{transcript}'' or ``^{log file}'' that records what
%happened during each session. For example, you should now have a file
%called |story.log| containing the transcript of Experiment~3, as well
%as a file called |texput.log| containing the transcript of Experiment~1. \
%(The transcript of Experiment~2 was probably overwritten when you did
%number~3.) \ Take a look at |story.log| now; you will see that the overfull
%box messages are accompanied not only by the abbreviated box contents,
%but also by some strange-looking data about hboxes and glue and kerns and
%such things. This data gives a precise description of what's in that
%overfull box; \TeX\ wizards will find such
%listings important, if they are called upon to diagnose some mysterious
%error, and you too may want to understand \TeX's internal code some day.
你不必在溢出盒子的信息在屏幕上消失前在纸上记下这些信息,因为 \TeX\ 总是%
把它们写成记录或``log 文件'', 它们记下了每个活动期间所发生的事情。%
例如,现在你有一个文件叫 |story.log|, 它包含实战三的记录,以及文件%
~|texput.log|, 它包含实战一的记录。%
(当你进行实战三时,实战二的记录可能已经被覆盖掉了。)
现在看一下 |story.log|; 就会看到伴随溢出盒子的信息的不但有简略的盒子的内容,
而且有某些与 hbox 和 glue 以及 kern 等等有关的奇怪的信息。%
这些资料给出了溢出盒子的精确描述;
在\TeX\ 奇才们被请求诊断某些不可理解的错误时,将发现这些罗列信息的重要性,
并且你可能也有一天想要懂得 \TeX\ 的内部代码。

%The abbreviated forms of overfull boxes show the hyphenations that
%\TeX\ tried before it resorted to overfilling. The ^{hyphenation} algorithm,
%which is described in Appendix~H\null, is excellent but not perfect; for
%example, you can see from the messages in |story.log| that \TeX\ finds the
%hyphen in `pre-ferred', and it can even hyphenate `Drof-nats'. Yet it
%discovers no hyphen in `galaxy', and every once in
%a~while an overfull box problem can be cured simply by giving \TeX\ a hint
%about how to hyphenate some word more completely. \ (We will see later that
%there are two ways to do this, either by inserting ^{discretionary hyphens}
%each time as in `\hbox{|gal\-axy|}', or by saying
%`\hbox{|\hyphenation{gal-axy}|}' once at the beginning of your manuscript.)
溢出盒子的简略形式给出了 \TeX\ 在溢出之前在连字化方面的尝试。%
连字算法阐述在附录 H, 它是优秀的但并不完美;
例如,从 |story.log| 的信息中你可以看到,\TeX\ 找到了`pre-ferred'中的连字,
并且它可以把`Drof-nats'连字化。%
但是它发现`galaxy'中没有连字符,有时候,简单地通过给 \TeX\ %
一个线索,让它知道怎样更完整地把某些单词连字化,可以解决盒子的溢出问题。%
(后面我们将看到,有两种方法来实现这个功能:象在`\hbox{|gal\-axy|}'中每次%
插入任意连字符,或者通过在文稿开始声明一次`\hbox{|\hyphenation{gal-axy}|}'。)

%In the present example, hyphenation is not a problem, since \TeX\ found
%and tried all the hyphens that could possibly have helped. The only way to
%get rid of the overfull boxes is to change the tolerance, i.e., to allow
%wider spaces between words. Indeed, the tolerance that plain \TeX\ uses
%for wide lines is completely inappropriate for 2-inch columns; such narrow
%columns simply can't be achieved without loosening the constraints, unless
%you rewrite the copy to fit.
在当前的例子中,连字符不是一个问题,因为 \TeX\ 找到并尝试了所有可能有用的%
连字符。%
解决盒子溢出的唯一方法是改变公差,即在单词之间允许更大的间距。%
的确,plain \TeX\ 对宽行设定的公差与 2-inch 的栏完全不相称;
不放松这些限制,这么窄的栏就不能完成任务,除非你为适应它而重新改写文稿。

%\TeX\ assigns a numerical value called ``^{badness}'' to each line that
%it sets, in order to assess the quality of the spacing. The exact rules
%for badness are different for different fonts, and they will be discussed
%in Chapter~14; but here is the way badness works for the roman font
%of plain \TeX:
%\begindisplay \hbadness10000
%\hbox spread-.666667em{The badness of this line is 100.}&
%  \quad(very tight)\cr
%\hbox spread-.333333em{The badness of this line is 12.}&
%  \quad(somewhat tight)\cr
%\hbox{The badness of this line is 0.}&
%  \quad(perfect)\cr
%\hbox spread.5em{The badness of this line is 12.}&
%  \quad(somewhat loose)\cr
%%\hbox spread 1em{The badness of this line is 100.}&
%% \quad(loose)\cr % then "looser"
%\hbox spread 1.259921em{The badness of this line is 200.}&
%  \quad(loose)\cr
%%\hbox spread 1.713em{The badness of this line is 500.}&
%% \quad(bad)\cr % then "worse"
%\hbox spread 2.155em{The badness of this line is 1000.}&
%  \quad(bad)\cr
%\hbox spread 3.684em{The badness of this line is 5000.}& % actually 4995!
%  \quad(awful)\cr
%\enddisplay
%Plain \TeX\ normally stipulates that no line's badness should exceed 200;
%but in our case, the task would be impossible since
%\begindisplay \hbadness 10000
%`\hbox to 2in{tant galaxy called \"O\"o\c c, there}'\hskip 3em
%  has badness 1521;\cr
%`\hbox to 2in{he preferred to be called---was}'\hskip 3em
%  has badness 568.\cr
%\enddisplay
%So we turn now to Experiment~4, in which spacing variations that are
%more appropriate to narrow columns will be used.
为了评估间距的品质,\TeX\ 为它设置的每行规定了一个数值叫做``丑度''。%
``丑度''的严格规则因字体不同而不同,它们将在第十四章中讨论;
\1但是,这里给出了 plain \TeX\ 中 roman 字体的丑度的情况:
\begindisplay \hbadness10000
\hbox spread-.666667em{The badness of this line is 100.}&
  \quad(太紧)\cr
\hbox spread-.333333em{The badness of this line is 12.}&
  \quad(有点紧)\cr
\hbox{The badness of this line is 0.}&
  \quad(正好)\cr
\hbox spread.5em{The badness of this line is 12.}&
  \quad(有点松散)\cr
%\hbox spread 1em{The badness of this line is 100.}&
% \quad(loose)\cr % then "looser"
\hbox spread 1.259921em{The badness of this line is 200.}&
  \quad(松散)\cr
%\hbox spread 1.713em{The badness of this line is 500.}&
% \quad(bad)\cr % then "worse"
\hbox spread 2.155em{The badness of this line is 1000.}&
  \quad(难看)\cr
\hbox spread 3.684em{The badness of this line is 5000.}& % actually 4995!
  \quad(丑陋)\cr
\enddisplay
Plain \TeX\ 一般规定,行的丑度不可大于 200;
但是在现在的情形下,因为
\begindisplay \hbadness 10000
`\hbox to 2in{tant galaxy called \"O\"o\c c, there}'\hskip 3em
  的丑度为 1521;\cr
`\hbox to 2in{he preferred to be called---was}'\hskip 3em
  的丑度为 568.\cr
\enddisplay
所以无法满足要求。%
因此我们现在进行实战四,其中对窄栏采用更适当的间距的变化。

%Run \TeX\ again, and begin this time by saying
%\begintt
%\hsize=2in \tolerance=1600 \input story
%\endtt
%so that lines with badness up to 1600 will be tolerated. Hurray! There are
%^^|\tolerance|
%no overfull boxes this time. \ (But you do get a message about an {\sl
%underfull\/} box, since \TeX\ reports all boxes whose badness exceeds
%a certain threshold called ^|\hbadness|; plain \TeX\ sets |\hbadness=1000|.) \
%^^{underfull box}
%Now make \TeX\ work still harder by trying
%\begintt
%\hsize=1.5in \input story
%\endtt
%(thus leaving the tolerance at 1600 but making the ^{column width} still
%^^{measure, see hsize}
%skimpier). Alas, overfull boxes return; so try typing
%\begintt
%\tolerance=10000 \input story
%\endtt
%in order to see what happens. \TeX\ treats 10000 as if it were ``infinite''
%tolerance, allowing arbitrarily wide space; thus, a tolerance of 10000 will
%{\sl never\/} produce an overfull box, unless something strange occurs like
%an unhyphenatable word that is wider than the column itself.
再次运行 \TeX, 这次在开头键入
\begintt
\hsize=2in \tolerance=1600 \input story
\endtt
使得要容许的丑度放宽到 1600。%
万岁! 这次没有盒子溢出了。%
(但是你却得到一个{\KT{10}松散}的盒子,因为如果盒子的丑度超过%
一个叫 |\hbadness| 的阈值,\TeX\ 也给出警告;
plain \TeX\ 设置 |\hbadness=1000|。)
现在键入
\begintt
\hsize=1.5in \input story
\endtt
让 \TeX\ 更难处理,
(这样就是保持容许度为 1600 而栏的宽度更窄)。%
唉,盒子又溢出了;
为了看看出现什么问题,那么再试试
\begintt
\tolerance=10000 \input story
\endtt
\TeX\ 把 10000 看作``无穷大''的容许度,允许任意宽的间距;
因此,容许度为 10000 将{\KT{10}永远不}出现盒子溢出,除非比栏自己的宽度%
还宽且无连字化的单词出现。

%The underfull box that \TeX\ produces in the 1.5-inch case is really bad;
%with such narrow limits, an occasional wide space is unavoidable. But try
%\begintt
%\raggedright \input story
%\endtt
%for a change. \ ^^|\raggedright|(This tells \TeX\ not to worry about keeping
%the right margin straight, and to keep the spacing uniform within each line.) \
%Finally, type
%\begintt
%\hsize=.75in \input story
%\endtt
%followed by `|\end|', to complete Experiment 4. This makes the columns
%almost impossibly narrow.
在 1.5-inch 情形下,\TeX\ 出现的松散盒子实在难看;
在这么窄的极限下,偶尔出现宽间距是不可避免的。%
而思路一变,试试
\begintt
\raggedright \input story
\endtt
(它告诉 \TeX\ 不用对齐右页边,且在每行中保持间距一致。)
最后,键入
\begintt
\hsize=.75in \input story
\endtt
再接着键入`|\end|'以结束实战四。%
这使得栏变得出奇地窄。

%\danger The output from this experiment will give you some feeling for the
%problem of ^{breaking a paragraph} into approximately equal lines. When the
%lines are relatively wide, \TeX\ will almost always find a good solution.
%But otherwise you will have to figure out some compromise, and several
%options are possible. Suppose you want to ensure that no lines have
%badness exceeding~500. Then you could set |\tolerance| to some high
%number, and |\hbadness=500|; \TeX\ would not produce overfull boxes, but
%it would warn you about the underfull ones. Or you could set
%|\tolerance=500|; then \TeX\ might produce overfull boxes. If you really
%want to take corrective action, the second alternative is better, because
%you can look at an overfull box to see how much sticks out; it becomes
%graphically clear what remedies are possible.  On the other hand, if you
%don't have time to fix bad spacing---if you just want to know how bad it
%is---then the first alternative is better, although it may require more
%computer time.
\danger \1在把段落分成大致相等的行这个问题上,本实战的输出结果会让你有所体会。%
当行相对宽时,\TeX\ 几乎总能找到好的解决方法。%
否则你必须给出一些折衷的方案,可能有几种可选择的方法。%
假设你要保证行的丑度不超过 500。%
那么你可以把 |\tolerance| 设为较大的值,以及 |\hbadness=100|;
\TeX\ 不出现溢出的盒子,但是会对松散的盒子发出警告。%
或者你设置 |\tolerance=500|; 那么 \TeX\ 可能出现溢出的盒\hbox{子。}%
如果你真希望起到提示纠正的作用,第二个办法更好,因为你可以看到溢出的盒子%
伸出了多少;
对可能的补救就一目了然了。%
另一方面,如果你没有时间来处理这些难看的间距——如果你只想看看%
它难看在哪里——那么第一种方法更好,只是会花费计算机更多的时间。

%\dangerexercise When |\raggedright| has been specified, badness reflects
%the amount of space at the right margin, instead of the spacing between
%words. Devise an experiment by which you can easily determine what
%badness \TeX\ assigns to each line, when the |story| is set ragged-right
%in 1.5-inch columns.
%\answer Run \TeX\ with \hbox{|\hsize=1.5in|} \hbox{|\tolerance=10000|}
%\hbox{|\raggedright|} \hbox{|\hbadness=-1|} and then |\input story|. \TeX\ will
%report the badness of all lines (except the final lines of paragraphs, where
%fill glue makes the badness zero).
\dangerexercise 当设定为 |\raggedright| 后,丑度反映的是右页边的空白的量,
而不是单词间的间距。%
设计一个方法,在 |story| 设置为 1.5-inch 栏且左对齐时,
通过它可以很容易得到 \TeX\ 给每行计算出的丑度。%
\answer 运行 \TeX ,接着依次输入 \hbox{|\hsize=1.5in|} \hbox{|\tolerance=10000|}
\hbox{|\raggedright|} \hbox{|\hbadness=-1|},然后 |\input story|。
\TeX\ 将报告各行的劣度(段落的最后一行除外,因为 fill 粘连将使得劣度为零)。

%\danger A parameter called ^|\hfuzz| allows you to ignore boxes that are only
%slightly overfull. For example, if you say |\hfuzz=1pt|, a box must stick
%out more than one point before it is considered erroneous. Plain \TeX\
%sets |\hfuzz=0.1pt|.
\danger 有一个叫 |\hfuzz| 的参数,它允许不理会那些只略微溢出的盒子。%
例如,如果设置 |\hfuzz=1pt|,
那么只有超出部分大于 1 point 的盒子才被认为是不正确的。%
Plain \TeX\ 的设置是 |\hfuzz=0.1pt|。

%\ddangerexercise Inspection of the output from Experiment~4, especially
%page~3, shows that with narrow columns it would be better to allow white
%space to appear before and after a dash, whenever other spaces in the
%same line are being stretched. Define a ^|\dash| macro that does this.
%\answer |\def\extraspace{\nobreak \hskip 0pt plus .15em\relax}|\parbreak
%|\def\dash{\unskip\extraspace---\extraspace}|\par\nobreak\smallskip\noindent
%(If you try this with the story at 2-inch and 1.5-inch sizes, you will
%notice a substantial improvement. The |\unskip| allows people to leave a
%space before typing |\dash|.  \TeX\ will try to hyphenate before |\dash|,
%but not before `|---|'; cf.\ Appendix~H\null. The ^|\relax| at the end of
%|\extraspace| is a precaution in case the next word is `|minus|'.)
\ddangerexercise 仔细观察实战四的输出,特别是第三页,可以发现,在窄栏情况下,
只要同一行的其它间距变大,在破折号前后允许出现空白效果会更好。%
定义一个叫 |\dash| 的宏来实现本要求。
\answer |\def\extraspace{\nobreak \hskip 0pt plus .15em\relax}|\parbreak
|\def\dash{\unskip\extraspace---\extraspace}|\par\nobreak\smallskip\noindent
(如果在 2-inch 和 1.5-inch 宽度中测试 story 文档,你会注意到实质性的改善。
|\unskip| 允许人们在键入 |\dash| 之前留出空格。
\TeX\ 将试图在 |\dash| 之前连字化,但不会在 `|---|' 之前;参考附录 H。
|\extraspace| 定义末尾的 ^|\relax| 用于预防下一个单词是 `|minus|' 的情形。)

%You were warned that this is a long chapter. But take heart: There's only
%one more experiment to do, and then you will know enough about \TeX\ to
%run it fearlessly by yourself forever after. The only thing you are still
%missing is some information about how to cope with ^{error messages}---i.e.,
%not just with warnings about things like overfull boxes, but with cases
%where \TeX\ actually stops and asks you what to do next.
已经提醒过你:本章很长。%
但是要记住:还只有一个实战,以后在运行 \TeX\ 时你就胸有成竹了。%
你还不知道怎样处理错误信息——即,不仅仅是象盒子溢出这样的警告,
而是 \TeX\ 停下来要求你干预的情况。

%Error messages can be terrifying when you aren't prepared for them;
%but they can be fun when you have the right attitude. Just remember that
%you really haven't hurt the computer's feelings, and that nobody will
%hold the errors against you. Then you'll find that running \TeX\ might
%actually be a creative experience instead of something to dread.
当你没有思想准备时,错误信息可能有些可怕;
但是当你端正了态度后,它们可能还很有意思。%
只是要记住,你其实别怨计算机,而且没人用这些错误来攻击你。%
接下来你会发现,运行 \TeX\ 实际上可能是一次创造体验,而不是可怕的老虎。

%The first step in Experiment 5 is to plant two intentional mistakes in the
%|story.tex| file. Change line~3 to
%\begintt
%\centerline{\bf A SHORT \ERROR STORY}
%\endtt
%and change `|\vskip|' to `|\vship|' on line~2.
实战五的第一步是在文件 |story.tex| 中故意放两个错误。%
把第 3 行改为
\begintt
\centerline{\bf A SHORT \ERROR STORY}
\endtt
并且把第 2 行的`|\vskip|'改为`|\vship|'。

%Now run \TeX\ again; but instead of `|story|' type `|sorry|'. The computer
%should respond by saying that it can't find file |sorry.tex|, and it will
%ask you to try again. Just hit \<return> this time; you'll see
%that you had better give the name of a real file. So type `|story|' and
%wait for \TeX\ to find one of the {\sl faux pas\/} in that file.
再次运行 \TeX; 但是不键入`|story|', 而键入`|sorry|'。%
计算机就回答说它找不到文件 |sorry.tex|,
并且它会要求你再尝试一下。%
这次,只单击 \<return>;
将看到你最好给出一个真实文件的名字。%
因此,键入`|story|'并等 \TeX\ 找到那个文件中的{\KT{10}过失}中的一个。

%Ah yes, the machine will soon stop,\footnote*{Some installations of \TeX\ do
%not allow interaction. In such cases all you can do is look at the error
%messages in your log file, where they will appear together with the ``help''
%information.} after typing something like this:
%\begintt
%! Undefined control sequence.
%l.2 \vship
%           1in
%?
%\endtt
%\write16{\ifnum\pageno=\vshippage
%  \else Redefine \string\vshippage to be \the\pageno\fi}%
%\TeX\ begins its error messages with `|!|', and it shows what it was
%reading at the time of the error by displaying two lines of context. The
%top line of the pair (in this case `|\vship|'\thinspace) shows what \TeX\
%has looked at so far, and where it came from (`|l.2|', i.e., line number~2);
%the bottom line (in this case `|1in|'\thinspace) shows what \TeX\ has yet
%to read.
\1噢,计算机很快停下来,(注:某些 \TeX\ 的安装不支持交互模式。%
在这种情况下,你只能在 log 文件中查看错误信息,在那里,这些错误信息和``help''%
信息一起出现。)
输出如下信息:
\begintt
! Undefined control sequence.
l.2 \vship
           1in
?
\endtt
\def\storypage{24} % listing of story.tex
\write16{\ifnum\pageno=\vshippage
  \else Redefine \string\vshippage to be \the\pageno\fi}%
\TeX\ 以`|!|'开始错误信息,并且它通过给出两行上下文来提示错误出现时的内容。%
上面一行(在本例是`|\vship|'\thinspace)给出了 \TeX\ 目前读的内容,并且告诉了它的%
位置(`|l.2|',即第 2 行);
下面一行(本例中的`|1in|'\thinspace)是 \TeX\ 将要读入的内容。

%The `^|?|'\ that appears after the context display means that \TeX\ wants
%advice about what to do next. If you've never seen an error message before,
%or if you've forgotten what sort of response is expected, you can type
%`|?|'\ now (go ahead and try it!); \TeX\ will respond as follows:
%\begintt
%Type <return> to proceed, S to scroll future error messages,
%R to run without stopping, Q to run quietly,
%I to insert something, E to edit your file,
%1 or ... or 9 to ignore the next 1 to 9 tokens of input,
%H for help, X to quit.
%\endtt
%This is your menu of options. You may choose to continue in various ways:
在显示的上下文后面出现的`|?|'意味着 \TeX\ 要求给出下一步怎样做。%
如果你以前没有见过错误信息,或者你忘记应该用什么来应答,现在你可以键入`|?|'%
(继续来试试吧!);
\TeX\ 将给出下列信息:
\begintt
Type <return> to proceed, S to scroll future error messages,
R to run without stopping, Q to run quietly,
I to insert something, E to edit your file,
1 or ... or 9 to ignore the next 1 to 9 tokens of input,
H for help, X to quit.
\endtt
这是供你选择的清单。%
你可以用各种方法来选择进行运行:

%\smallskip\item{1.}
%Simply type \<return>. \TeX\ will resume its processing, after
%attempting to recover from the error as best it can.
\smallskip\item{1.}
直接键入 \<return>. 在尽可能修复错误后,\TeX\ 将恢复运行。

%\smallbreak\item{2.} Type `|S|'. \TeX\ will proceed without
%pausing for instructions if further errors arise. Subsequent error messages
%will flash by on your terminal, possibly faster than you can read them, and
%they will appear in your log file where you can scrutinize them at your
%leisure. Thus, `|S|' is sort of like typing \<return> to every message.
\smallbreak\item{2.} 键入`|S|'。如果还有错误出现,\TeX\ 将直接运行,不再提示。%
随后的错误信息将在你的屏幕上闪过,快到你来不及看它们,
而且它们将出现在你的 log 文件中,在那里你可以静下心来细细查看。%
所以,`|S|'就象对每个错误都键入 \<return>。

%\smallbreak\item{3.} Type `|R|'. This is like `|S|' but even stronger,
%since it tells \TeX\ not to stop for any reason, not even if a file name
%can't be found.
\smallbreak\item{3.} 键入`|R|'。这类似于`|S|', 但是更强大,因为它告诉 \TeX,
不管什么原因都不停止,即使文件名都找不见。

%\smallbreak\item{4.} Type `|Q|'. This is like `|R|' but even more so,
%since it tells \TeX\ not only to proceed without stopping but also to
%suppress all further output to your terminal. It is a fast, but somewhat
%reckless, way to proceed (intended for running \TeX\ with no operator in
%attendance).
\smallbreak\item{4.} 键入`|Q|'。它类似于`|R|', 但还要强大,
因为它不但告诉 \TeX\ 无停顿地运行,而且不再在屏幕上输出信息。%
它是一种快速但不考虑后果的运行方式(就是放任 \TeX\ 去运行而不进行任何干预)。

%\smallbreak\item{5.} Type `|I|', followed by some text that you want to
%insert. \TeX\ will read this line of text before encountering what it
%would ordinarily see next. Lines inserted in this way are not assumed to
%end with a blank space. ^^{inserting text online}
%^^{online interaction, see interaction} ^^{interacting with TeX}
\smallbreak\item{5.} 键入`|I|', 后面跟一些你要插入的文本。%
\TeX\ 将首先读入此行再继续读入原来要读入的内\hbox{容。}%
用这种方法插入的行不以空格为结尾。

%\smallbreak\item{6.} Type a small number (less than 100). \TeX\ will
%delete this many characters and control sequences from whatever it is
%about to read next, and it will pause again to give you another chance to
%look things over.  ^^{deleting tokens}
\smallbreak\item{6.} \1键入一个数(小于 100)。%
\TeX\ 将从接下来读入的字符和控制系列中删除这么多字节,
并且做完后它会再停下来等待干预。

%\smallbreak\item{7.} Type `|H|'. This is what you should do now and whenever
%you are faced with an error message that you haven't seen for a~while. \TeX\
%has two messages built in for each perceived error: a formal one and an
%informal one. The formal message is printed first (e.g., `|! Undefined
%control sequence.|'\thinspace); the informal one is printed if you request
%more help by typing `|H|', and it also appears in your log file if you
%are scrolling error messages. The informal message tries to complement the
%formal one by explaining what \TeX\ thinks the trouble is, and often
%by suggesting a strategy for recouping your losses.^^{help messages}
\smallbreak\item{7.} 键入`|H|'。这就是你现在所做的事情,对你暂时没搞清楚%
的任何错误信息都可以使用。%
对发现的每个错误,\TeX\ 都给出两个信息:正式的和非正式的。
正式的信息首先输出(比如,`|! Undefined
control sequence.|'\thinspace); 如果你键入`|H|'请求更多帮助,非正式的信息才输出,
而且,如果错误信息翻滚过去了,你还可以在 log 文件中查看。%
非正式的信息尝试着给出 \TeX\ 所认为的问题所在来补充正式的信息,
并且通常为修补错误而提出一个方法。

%\smallbreak\item{8.} Type `|X|'. This stands for ``exit.'' It causes \TeX\
%to stop working on your job, after putting the finishing touches on your
%|log| file and on any pages that have already been output to your |dvi|
%file.  The current (incomplete) page will not be output.
\smallbreak\item{8.} 键入`|X|'。它表示要``exit''。%
在把已完成的相关任务写入 |log| 文件和已输出的页面后,它把 \TeX\ 终止。%
当前(未完成的)页面不产生输出。

%\smallbreak\item{9.} Type `|E|'. This is like `|X|', but it also prepares
%the computer to edit the file that \TeX\ is currently reading, at the
%current position, so that you can conveniently make a change before
%trying again.
\smallbreak\item{9.} 键入`|E|'。它类似于`|X|', 但是它还告诉计算机对当前文件进行%
编辑,位置就是当前停止的位置,这样你就可以很方便修改后再次运行。

%\smallbreak\noindent
%After you type `|H|' (or `|h|', which also works), you'll get a message
%that tries to explain that the control sequence just read by \TeX\
%(i.e., |\vship|) has never been assigned a meaning, and that you should
%either insert the correct control sequence or you should go on as if the
%offending one had not appeared.
\smallbreak\noindent
当你键入`|H|'(或者`|h|')后,就得到一个提示,它尝试着解释说,
\TeX\ 刚刚读入的控制系列(即,|\vship|)没有给出定义,
并且你要么插入一个正确的控制系列,要么直接继续,就象这个讨厌的东西没有出现过一样。

%In this case, therefore, your best bet is to type
%\begintt
%I\vskip
%\endtt
%(and \<return>), with no space after the `|I|'; this effectively replaces
%|\vship| by |\vskip|. \ (Do it.)
在本情形下,你最好的选择是键入
\begintt
I\vskip
\endtt
(并且 \<return>), 注意,在`|I|'后面没有空格;
这就把 |\vship| 替换成了 |\vskip|。(试试吧!)

%If you had simply typed \<return> instead of
%inserting anything, \TeX\ would have gone ahead and read `|1in|', which
%it would have regarded as part of a paragraph to be typeset. Alternatively,
%you could have typed `|3|'\thinspace; that would have deleted
%`|1in|' from \TeX's input. Or you could have typed `|X|' or `|E|' in
%order to correct the spelling error in your file. But it's usually
%best to try to detect as many errors as you can, each time you run \TeX,
%since that increases your productivity while decreasing your computer bills.
%Chapter~27 explains more about the art of steering \TeX\ through
%troubled text.
如果你直接键入 \<return> 而不是其它选项,那么 \TeX\ 将继续运行并读入`|1in|',
它将被看作要排版的段落的一部分。%
你也可以键入`|3|'\thinspace; 它将从 \TeX\ 的输入中删除掉`|1in|'。%
另外你也可以键入`|X|'或`|E|'以改正文件中的拼写错误。%
但是一般最好是每次运行 \TeX\ 时检出尽可能多的错误,这样可以提高效率和减少成本。%
第二十七章阐述了引导 \TeX\ 处理这些出问题的文本的更多\hbox{技术。}

%\dangerexercise What would have happened if you had typed `|5|' after
%the |\vship| error?
%\answer \TeX\ would have deleted five tokens: |1|, |i|, |n|, \],
%|\centerline|.  (The space was at the end of line~2, the |\centerline| at the
%beginning of line~3.)
\dangerexercise 如果你在 |\vship| 这个错误后键入`|5|', 会出现什么情况?
\answer \TeX\ 将删除五个记号:|1|、|i|、|n|、\] 和 |\centerline|。%
(空格在第 2 行的行尾,而 |\centerline| 在第 3 行的开头。)

%\danger You can control the level of interaction by giving commands
%in your file as well as online: The \TeX\ primitives ^|\scrollmode|,
%^|\nonstopmode|, and ^|\batchmode| correspond respectively to typing
%`|S|', `|R|', or `|Q|' in response to an error message, and
%^|\errorstopmode| puts you back into the normal level of interaction. \
%(Such changes are global, whether or not they appear inside a group.) \
%Furthermore, many installations have implemented a way to ^{interrupt}
%\TeX\ while it is running; such an interruption causes the program to
%revert to |\errorstopmode|, after which it pauses and waits for
%further instructions.
\danger 通过在文件中以及在交互时给定命令,你可以控制交互的等级:
\TeX\ 的原始 |\scrollmode|、|\nonstopmode| 以及 |\batchmode|
分别对应于对错误信息键入 `|S|'、`|R|' 或 `|Q|',
而且 |\errorstopmode| 转换回正常的交互等级。%
(这样的改变是全局的,不管它们是否出现在组中。)
还有,很多已安装好的系统已经设置了 \TeX\ 运行时中断它的方法;
\1这样的中断使程序转换回 |\errorstopmode|,在此处 \TeX\ 停下来等待干预。

%What happens next in Experiment 5? \TeX\ will hiccup on the other bug that
%we planted in the file. This time, however, the error message is more
%elaborate, since the context appears on six lines instead of two:
%\begintt
%! Undefined control sequence.
%<argument> \bf A SHORT \ERROR
%                              STORY
%\centerline #1->\line {\hss #1
%                              \hss }
%l.3 \centerline{\bf A SHORT \ERROR STORY}
%|null
%?
%\endtt
%You get multiline error messages like this when the error is detected
%while \TeX\ is processing some higher-level commands---in this case,
%while it is trying to carry out |\centerline|, which is not a primitive
%operation (it is defined in plain \TeX). At first, such error
%messages will appear to be complete nonsense to you, because much of what
%you see is low-level \TeX\ code that you never wrote. But you can overcome
%this hangup by getting a feeling for the way \TeX\ operates.
实战五接下来会出现什么呢?
\TeX\ 将被噎在我们故意设置的另一个缺陷上。%
但是这次,错误信息更详细,因为出现了 6 行而不是两行:
\begintt
! Undefined control sequence.
<argument> \bf A SHORT \ERROR
                              STORY
\centerline #1->\line {\hss #1
                              \hss }
l.3 \centerline{\bf A SHORT \ERROR STORY}
|null
?
\endtt
当 \TeX\ 在处理某些高级命令而检测到的错误时,你就会得到类似的多行错误信息%
——在本例中,它正在处理非原始的定义 |\centerline|(它定义在 plain \TeX\ 中)。%
首先,这样的错误信息对你毫无用处,因为你所看到的是不曾写过的低级 \TeX\ 代码。%
但是通过体会 \TeX\ 的运作方式,你会处理好这个错误。

%First notice that the context information always appears in pairs of lines.
%As before, the top line shows what \TeX\ has just read (\thinspace
%`|\bf A SHORT \ERROR|'\thinspace), then comes what it is about to read
%(\thinspace`|STORY|'\thinspace). The next pair of lines shows the context
%of the first two; it indicates what \TeX\ was doing just before it began to
%read the others. In this case, we see that \TeX\ has just read `|#1|', which
%is a special code that tells the machine to ``read the first ^{argument} that
%is governed by the current control sequence''; i.e., ``now read the stuff that
%^|\centerline| is supposed to center on a line.'' The definition in Appendix~B
%says that |\centerline|, when applied to some text, is supposed to be carried
%out by sticking that text in place of the `|#1|' in `|\line{\hss#1\hss}|'.
%So \TeX\ is in the midst of this expansion of\/ |\centerline|, as well as being
%in the midst of the text that is to be centered.
首先注意到,上下文的内容总出现在两行中。%
象以前一样,上一行表示 \TeX\ 刚刚读入的(\thinspace
`|\bf A SHORT \ERROR|'\thinspace), 接着出现的是将要读入的%
(\thinspace`|STORY|'\thinspace)。
接下来的两行是上面两行的上下文;
它表明,在开始读入上面两行前 \TeX\ 所做的事情。%
在本例我们看到,\TeX\ 刚刚读入`|#1|', 它是一个特殊代码,来告诉计算机%
``读入由当前控制系列处理的第一个参量'';
即``现在读入 |\centerline| 要居中的行的内容''。%
在附录 B 中的定义声明,当应用到某些文本时,|\centerline| 要做的是用那些文本来%
代替`|\line{\hss#1\hss}|'中的`|#1|'。%
所以 \TeX\ 运行到了 |\centerline| 的展开代码的中间代码,即要居中的文本的内容。

%\looseness-1
%The bottom line shows how far \TeX\ has gotten until now in the |story| file.
%\ (Actually the bottom line is blank in this example; what appears to be
%the bottom line is really the first of two lines of context, and it
%indicates that \TeX\ has read everything including the `|}|' in line~3 of
%the file.) \ Thus, the context in this error message gives us a glimpse of
%how \TeX\ went about its business. First, it saw |\centerline| at the
%beginning of line~3. Then it looked at the definition of\/ |\centerline| and
%noticed that |\centerline| takes an ``argument,'' i.e., that |\centerline|
%applies to the next character or control sequence or group that follows.
%So \TeX\ read~on, and filed `|\bf A SHORT \ERROR STORY|' away as the
%argument to |\centerline|.  Then it began to read the expansion, as
%defined in Appendix~B\null.  When it reached the |#1|, it began to read
%the argument it had saved.  And when it reached |\ERROR|, it complained
%about an undefined control sequence.
\looseness-1
底部的行表示到现在为止 \TeX\ 得到的文件 |story| 中的进度。%
(实际上,本例的底行是空的;
出现在底行的实际上是第一个两行的上下文,并且它表明 \TeX\ 已经读入包括`|}|'的文件的%
第 3 行的所有内容。)
因此,在本错误信息中的内容让我们一览 \TeX\ 的处理方式。%
首先,它遇见第 3 行开头的 |\centerline|。%
接着它得到 |\centerline| 的定义,并注意到 |\centerline| 有一个``参量'',
即,|\centerline| 要应用在跟着的下一个字符或者控制系列,或者组上。%
因此 \TeX\ 继续读入,把`|\bf A SHORT \ERROR STORY|'提出来作为 |\centerline| 的参量。%
接着它开始展开定义,正如附录 B 所定义的一样。%
当它读到 |#1|, 就开始读入存储的参量。%
并且当遇到 |\ERROR| 时就给出未定义的控制系列这个错误信息。

%\dangerexercise Why didn't \TeX\ complain about |\ERROR| being undefined
%when |\ERROR| was first encountered, i.e., before reading `|STORY}|' on line~3?
%\answer A control sequence like |\centerline| might well define a control
%sequence like |\ERROR| before telling \TeX\ to look at |#1|. Therefore
%\TeX\ doesn't interpret control sequences when it scans an argument.
\dangerexercise \1为什么 \TeX\ 第一次遇到 |\ERROR| 时(即在读第 3 行的`|STORY|'之前)%
不认为它是一个未定义的控制\hbox{系列呢?}
\answer 类似 |\centerline| 的控制系列很可能%
在让 \TeX\ 查看 |#1| 之前就定义了类似 |\ERROR| 的控制系列。
\TeX\ 在扫描参量时并不解释其中的控制系列。

%When you get a multiline error message like this, the best clues about the
%source of the trouble are usually on the bottom line (since that is what
%you typed) and on the top line (since that is what triggered the error
%message). Somewhere in there you can usually spot the problem.
当你看到类似的错误信息时,找到问题根源的最好思路通常是底行(因为它是你键入的内\hbox{容)}%
和顶行(因为那里是错误发生的地方)。%
在那里的某个地方你通常可以发现问题。

%Where should you go from here? If you type `|H|' now, you'll just get
%the same help message about undefined control sequences that you saw
%before. If you respond by typing \<return>, \TeX\ will go
%on and finish the run, producing output virtually identical to that in
%Experiment~2. In other words, the conventional responses won't teach you
%anything new. So type `|E|' now; this terminates the run and prepares
%the way for you to fix the erroneous file. \ (On some systems, \TeX\ will
%actually start up the standard text editor, and you'll be positioned at
%the right place to delete `|\ERROR|'. On other systems, \TeX\ will simply
%tell you to edit line~3 of file |story.tex|.) ^^{editing}
现在你该怎么办?
如果现在键入`|H|', 得到的是与你以前看到的一样的关于未定义控制系列的帮助信息。%
如果键入 \<return>, \TeX\ 将继续运行,得到与实战二相同的输出结果。%
换句话说,一般的干预没什么新的东西了。%
那么,键入`|E|'; 它将终止运行并打开有错误的文件。%
(在某些系统上,\TeX\ 实际上启动一个标准的文本编辑器,
并且定位在要删除`|\ERROR|'的地方。%
在其它系统上,\TeX\ 将直接告诉你编辑文件 |story.tex| 的第 3 行。

%When you edit |story.tex| again, you'll notice that line~2 still contains
%|\vship|; the fact that you told \TeX\ to insert |\vskip| doesn't mean
%that your file has changed in any way. In general, you should correct all
%errors in the input file that were spotted by \TeX\ during a run; the
%log file provides a handy way to remember what those errors were.
当你再次编辑 |story.tex| 时,会注意到第 2 行包含的还是 |\vship|;
事实上,你要 \TeX\ 插入 |\vskip| 并不意味着文件有任何改变。%
一般地,在运行时,你应该改正 \TeX\ 发现的所有错误;
log 文件为记录那些出现错误提供了一种方便的方法。

%\smallskip
%Well, this has indeed been a long chapter, so let's summarize what has
%been accomplished. By doing the five experiments you have learned at first
%hand (1)~how to get a job printed via \TeX; (2)~how to make a file that
%contains a complete \TeX\ manuscript; (3)~how to change the plain \TeX\
%format to achieve columns with different widths; and (4)~how to avoid
%panic when \TeX\ issues stern~warnings.
\smallskip
好了,本章的确比较长,那么我们总结一下。%
通过上面的五个实战,你已经直接掌握了:
(1). 怎样通过 \TeX\ 输出一个的文稿;
(2). 怎样制作包含这个 \TeX\ 文稿的文件;
(3). 怎样改变 plain \TeX\ 的格式以得到不同宽度的栏;
(4). 当 \TeX\ 给出警告时怎样有条不紊地修正。

%So you could now stop reading this book and go on to print a bunch of
%documents. It is better, however, to continue bearing with the author
%(after perhaps taking another rest), since you're just at the threshold
%of being able to do a lot more. And you ought to read Chapter~7
%at least, because it warns you about certain symbols that you must not
%type unless you want \TeX\ to do something special. While reading the
%remaining chapters it will, of course, be best for you to continue making
%trial runs, using experiments of your own design.
那么,你现在可以不再阅读本书,并且可输出很多文档了。%
但是最好继续下去,因为你正处在深造的关头。%
而且至少你应该读读第七章,
因为它给出了不必键入的一些符号,除非你想要 \TeX\ 做某些特殊的事情。%
当阅读剩下的章节时,当然最好做一些实战练习,你自己去设计\hbox{实验吧。}

%\ddanger If you use \TeX\ format packages designed by others, your
%error messages may involve many inscrutable two-line levels of macro
%context. By setting ^|\errorcontextlines||=0| at the beginning of your file,
%you can reduce the amount of information that is reported;
%\TeX\ will show only the top and bottom pairs of context lines
%together with up to |\errorcontextlines| additional two-line items. \ (If
%anything has thereby been omitted, you'll also see `|...|'.) \ Chances
%are good that you can spot the source of an error even when most of a
%large context has been suppressed; if not, you can say
%`|I\errorcontextlines=100\oops|' and try again. \ (That will usually
%give you an undefined control sequence error and plenty of context.) \
%Plain \TeX\ sets |\errorcontextlines=5|.
\ddanger 如果你使用别人设计的 \TeX\ 格式包,
错误信息可能包括许多不可预测的双行的宏的内容。
通过在你的文件开头设置 |\errorcontextlines||=0|,可以减少输出的信息的量;
\TeX\ 将只输出最上面两行和最下面两行内容,
以及与 |\errorcontextlines| 有关的附加的两行内容。
(如果因此忽略了一些内容,你还会看到 `|...|'。)
如果当大量错误信息被去掉时仍能找到错误所在,就太幸运了;
否则,你可以键入 `|I\errorcontextlines=100\oops|' 并且再次运行。%
(它一般给出了未定义控制系列的错误和大量上下文。)

\endchapter

What we have to learn to do we learn by doing.
\author ^{ARISTOTLE}, {\sl Ethica Nicomachea\/} II (c.~325 B.C.)

\bigskip

He may run who reads.
\author ^{HABAKKUK} 2\thinspace:\thinspace2 (c.~600 B.C.) % RSV
\smallskip
He that runs may read.
\author WILLIAM ^{COWPER}, {\sl Tirocinium\/} (1785)

\vfill\eject\byebye
